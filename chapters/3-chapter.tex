\newthought{Secondly, history is necessary} to the man of conservative and
reverent nature, who looks back to the origins of his existence with
love and trust; through it, he gives thanks for life. He is careful
to preserve what survives from ancient days, and will reproduce the
conditions of his own upbringing for those who come after him; thus
he does life a service. The possession of his ancestors' furniture
changes its meaning in his soul: for his soul is rather possessed by
it. All that is small and limited, mouldy and obsolete, gains a worth
and inviolability of its own from the conservative and reverent soul
of the antiquary migrating into it, and building a secret nest there.
The history of his town becomes the history of himself; he looks on
the walls, the turreted gate, the town council, the fair, as an
illustrated diary of his youth, and sees himself in it all -- his
strength, industry, desire, reason, faults and follies. \enquote{Here one
could live,} he says, \enquote{as one can live here now -- and will go on
living; for we are tough folk, and will not be uprooted in the
night.} And so, with his \enquote{we,} he surveys the marvellous individual
life of the past and identifies himself with the spirit of the house,
the family and the city. He greets the soul of his people from afar
as his own, across the dim and troubled centuries: his gifts and his
virtues lie in such power of feeling and divination, his scent of a
half-vanished trail, his instinctive correctness in reading the
scribbled past, and understanding at once its palimpsests -- nay, its
polypsests. Goethe stood with such thoughts before the monument of
Erwin von Steinbach: the storm of his feeling rent the historical
cloud-veil that hung between them, and he saw the German work for the
first time \enquote{coming from the stern, rough, German soul.} This was the
road that the Italians of the Renaissance travelled, the spirit that
reawakened the ancient Italic genius in their poets to \enquote{a wondrous
echo of the immemorial lyre,} as Jacob Burckhardt\footnote{Carl Jacob Christoph Burckhardt ( 1818 –- 1897) was a Swiss historian known as one of the major founders of the field of cultural history.} says. But the
greatest value of this antiquarian spirit of reverence lies in the
simple emotions of pleasure and content that it lends to the drab,
rough, even painful circumstances of a nation's or individual's life:
Niebuhr confesses that he could live happily on a moor among free
peasants with a history, and would never feel the want of art. How
could history serve life better than by anchoring the less gifted
races and peoples to the homes and customs of their ancestors, and
keeping them from ranging far afield in search of better, to find
only struggle and competition? The influence that ties men down to
the same companions and circumstances, to the daily round of toil, to
their bare mountain-side, -- seems to be selfish and unreasonable: but
it is a healthy unreason and of profit to the community; as every one
knows who has clearly realised the terrible consequences of mere
desire for migration and adventure, -- perhaps in whole peoples, -- or
who watches the destiny of a nation that has lost confidence in its
earlier days, and is given up to a restless cosmopolitanism and an
unceasing desire for novelty. The feeling of the tree that clings to
its roots, the happiness of knowing one's growth to be one not merely
arbitrary and fortuitous, but the inheritance, the fruit and blossom
of a past, that does not merely justify but crown the present -- this
is what we nowadays prefer to call the real historical sense.

These are not the conditions most favourable to reducing the past to
pure science: and we see here too, as we saw in the case of
monumental history, that the past itself suffers when history serves
life and is directed by its end. To vary the metaphor, the tree feels
its roots better than it can see them: the greatness of the feeling
is measured by the greatness and strength of the visible branches.
The tree may be wrong here; how far more wrong will it be in regard
to the whole forest, which it only knows and feels so far as it is
hindered or helped by it, and not otherwise! The antiquarian sense of
a man, a city or a nation has always a very limited field. Many
things are not noticed at all; the others are seen in isolation, as
through a microscope. There is no measure: equal importance is given
to everything, and therefore too much to anything. For the things of
the past are never viewed in their true perspective or receive their
just value; but value and perspective change with the individual or
the nation that is looking back on its past.

There is always the danger here, that everything ancient will be
regarded as equally venerable, and everything without this respect
for antiquity, like a new spirit, rejected as an enemy. The Greeks
themselves admitted the archaic style of plastic art by the side of
the freer and greater style; and later, did not merely tolerate the
pointed nose and the cold mouth, but made them even a canon of taste.
If the judgment of a people harden in this way, and history's service
to the past life be to undermine a further and higher life; if the
historical sense no longer preserve life, but mummify it: then the
tree dies, unnaturally, from the top downwards, and at last the roots
themselves wither. Antiquarian history degenerates from the moment
that it no longer gives a soul and inspiration to the fresh life of
the present. The spring of piety is dried up, but the learned habit
persists without it and revolves complaisantly round its own centre.
The horrid spectacle is seen of the mad collector raking over all the
dust-heaps of the past. He breathes a mouldy air; the antiquarian
habit may degrade a considerable talent, a real spiritual need in
him, to a mere insatiable curiosity for everything old: he often
sinks so low as to be satisfied with any food, and greedily devour
all the scraps that fall from the bibliographical table.

Even if this degeneration do not take place, and the foundation be
not withered on which antiquarian history can alone take root with
profit to life: yet there are dangers enough, if it become too
powerful and invade the territories of the other methods. It only
understands how to preserve life, not to create it; and thus always
undervalues the present growth, having, unlike monumental history, no
certain instinct for it. Thus it hinders the mighty impulse to a new
deed and paralyses the doer, who must always, as doer, be grazing
some piety or other. The fact that has grown old carries with it a
demand for its own immortality. For when one considers the
life-history of such an ancient fact, the amount of reverence paid to
it for generations -- whether it be a custom, a religious creed, or a
political principle, -- it seems presumptuous, even impious, to replace
it by a new fact, and the ancient congregation of pieties by a new
piety.

Here we see clearly how necessary a third way of looking at the past
is to man, beside the other two. This is the \enquote{critical} way; which is
also in the service of life. Man must have the strength to break up
the past; and apply it too, in order to live. He must bring the past
to the bar of judgment, interrogate it remorselessly, and finally
condemn it. Every past is worth condemning: this is the rule in
mortal affairs, which always contain a large measure of human power
and human weakness. It is not justice that sits in judgment here; nor
mercy that proclaims the verdict; but only life, the dim, driving
force that insatiably desires -- itself. Its sentence is always
unmerciful, always unjust, as it never flows from a pure fountain of
knowledge: though it would generally turn out the same, if Justice
herself delivered it. \enquote{For everything that is born is \textit{worthy} of
being destroyed: better were it then that nothing should be born.}\footnote{Perhaps echoing Ecclesiastes 4:2-3, \enquote{Therefore I praised the dead who were already dead, More than the living who are still alive. Yet, better than both is he who has never existed, Who has not seen the evil work that is done under the sun.}} It
requires great strength to be able to live and forget how far life
and injustice are one. Luther himself once said that the world only
arose by an oversight of God; if he had ever dreamed of heavy
ordnance, he would never have created it. The same life that needs
forgetfulness, needs sometimes its destruction; for should the
injustice of something ever become obvious -- a monopoly, a caste, a
dynasty for example -- the thing deserves to fall. Its past is
critically examined, the knife put to its roots, and all the
\enquote{pieties} are grimly trodden under foot. The process is always
dangerous, even for life; and the men or the times that serve life in
this way, by judging and annihilating the past, are always dangerous
to themselves and others. For as we are merely the resultant of
previous generations, we are also the resultant of their errors,
passions, and crimes: it is impossible to shake off this chain.
Though we condemn the errors and think we have escaped them, we
cannot escape the fact that we spring from them. At best, it comes to
a conflict between our innate, inherited nature and our knowledge,
between a stern, new discipline and an ancient tradition; and we
plant a new way of life, a new instinct, a second nature, that
withers the first. It is an attempt to gain a past \textit{a posteriori}\footnote{A priori ('from the earlier') and a posteriori ('from the later') are Latin phrases used in philosophy to distinguish types of knowledge or argument by their reliance on experience. \textit{A priori} knowledge is independent from any experience (for example, mathematical). \textit{A posteriori} knowledge depends on empirical evidence.}
from which we might spring, as against that from which we do spring;
always a dangerous attempt, as it is difficult to find a limit to the
denial of the past, and the second natures are generally weaker than
the first. We stop too often at knowing the good without doing it,
because we also know the better but cannot do it. Here and there the
victory is won, which gives a strange consolation to the fighters, to
those who use critical history for the sake of life. The consolation
is the knowledge that this \enquote{first nature} was once a second, and that
every conquering \enquote{second nature} becomes a first.

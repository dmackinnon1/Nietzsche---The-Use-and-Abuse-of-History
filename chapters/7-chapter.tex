\newthought{The unrestrained historical sense}, pushed to its logical extreme,
uproots the future, because it destroys illusions and robs existing
things of the only atmosphere in which they can live. Historical
justice, even if practised conscientiously, with a pure heart, is
therefore a dreadful virtue, because it always undermines and ruins
the living thing: its judgment always means annihilation. If there be
no constructive impulse behind the historical one, if the clearance
of rubbish be not merely to leave the ground free for the hopeful
living future to build its house, if justice alone be supreme, the
creative instinct is sapped and discouraged. A religion, for example,
that has to be turned into a matter of historical knowledge by the
power of pure justice, and to be scientifically studied throughout,
is destroyed at the end of it all. For the historical audit brings so
much to light which is false and absurd, violent and inhuman, that
the condition of pious illusion falls to pieces. And a thing can only
live through a pious illusion. For man is creative only through love
and in the shadow of love's illusions, only through the unconditional
belief in perfection and righteousness. Everything that forces a man
to be no longer unconditioned in his love, cuts at the root of his
strength: he must wither, and be dishonoured. Art has the opposite
effect to history: and only perhaps if history suffer transformation
into a pure work of art, can it preserve instincts or arouse them.
Such history would be quite against the analytical and inartistic
tendencies of our time, and even be considered false. But the history
that merely destroys without any impulse to construct, will in the
long-run make its instruments tired of life; for such men destroy
illusions, and \enquote{he who destroys illusions in himself and others is
punished by the ultimate tyrant, Nature.} For a time a man can take
up history like any other study, and it will be perfectly harmless.
Recent theology seems to have entered quite innocently into
partnership with history, and scarcely sees even now that it has
unwittingly bound itself to the Voltairean \textit{écrasez}! No one need
expect from that any new and powerful constructive impulse: they
might as well have let the so-called Protestant Union serve as the
cradle of a new religion, and the jurist Holtzendorf, the editor of
the far more dubiously named Protestant Bible, be its John the
Baptist. This state of innocence may be continued for some time by
the Hegelian philosophy\footnote{Georg Wilhelm Friedrich Hegel (1770 –- 1831) was a German philosopher and one of the most influential figures of German idealism and 19th-century philosophy.}, -- still seething in some of the older
heads, -- by which men can distinguish the \enquote{idea of Christianity} from
its various imperfect \enquote{manifestations}; and persuade themselves that
it is the \enquote{self-movement of the Idea} that is ever particularising
itself in purer and purer forms, and at last becomes the purest, most
transparent, in fact scarcely visible form in the brain of the
present \textit{theologus liberalis vulgaris}\footnote{\textit{theologus liberalis vulgaris} -- \enquote{popular liberal theologian}}. But to listen to this pure
Christianity speaking its mind about the earlier impure Christianity,
the uninitiated hearer would often get the impression that the talk
was not of Christianity at all but of ... -- what are we to think? if
we find Christianity described by the \enquote{greatest theologians of the
century} as the religion that claims to \enquote{find itself in all real
religions and some other barely possible religions,} and if the \enquote{true
church} is to be a thing \enquote{which may become a liquid mass with no
fixed outline, with no fixed place for its different parts, but
everything to be peacefully welded together} -- what, I ask again, are
we to think?

Christianity has been denaturalised by historical treatment -- which in
its most complete form means \enquote{just} treatment -- until it has been
resolved into pure knowledge and destroyed in the process. This can
be studied in everything that has life. For it ceases to have life if
it be perfectly dissected, and lives in pain and anguish as soon as
the historical dissection begins. There are some who believe in the
saving power of German music to revolutionise the German nature. They
angrily exclaim against the special injustice done to our culture,
when such men as Mozart and Beethoven are beginning to be spattered
with the learned mud of the biographers and forced to answer a
thousand searching questions on the rack of historical criticism. Is
it not premature death, or at least mutilation, for anything whose
living influence is not yet exhausted, when men turn their curious
eyes to the little minutiæ of life and art, and look for problems of
knowledge where one ought to learn to live, and forget problems? Set
a couple of these modern biographers to consider the origins of
Christianity or the Lutheran reformation: their sober, practical
investigations would be quite sufficient to make all spiritual
\enquote{action at a distance} impossible: just as the smallest animal can
prevent the growth of the mightiest oak by simply eating up the
acorn. All living things need an atmosphere, a mysterious mist,
around them. If that veil be taken away and a religion, an art, or a
genius condemned to revolve like a star without an atmosphere, we
must not be surprised if it becomes hard and unfruitful, and soon
withers. It is so with all great things \enquote{that never prosper without
some illusion,} as Hans Sachs\footnote{Hans Sachs (1494 –- 1576) was a German poet.} says in the Meistersinger.\footnote{\textit{Die Meistersinger von Nürnberg} (\enquote{The Master-Singers of Nuremberg}), an opera by Richard Wagner in which Hans Sachs is a character.}

Every people, every man even, who would become ripe, needs such a
veil of illusion, such a protecting cloud. But now men hate to become
ripe, for they honour history above life. They cry in triumph that
\enquote{science is now beginning to rule life.} Possibly it might; but a
life thus ruled is not of much value. It is not such true \textit{life}, and
promises much less for the future than the life that used to be
guided not by science, but by instincts and powerful illusions. But
this is not to be the age of ripe, alert and harmonious
personalities, but of work that may be of most use to the
commonwealth. Men are to be fashioned to the needs of the time, that
they may soon take their place in the machine. They must work in the
factory of the \enquote{common good} before they are ripe, or rather to
prevent them becoming ripe; for this would be a luxury that would
draw away a deal of power from the \enquote{labour market.} Some birds are
blinded that they may sing better; I do not think men sing to-day
better than their grandfathers, though I am sure they are blinded
early. But light, too clear, too sudden and dazzling, is the infamous
means used to blind them. The young man is kicked through all the
centuries: boys who know nothing of war, diplomacy, or commerce are
considered fit to be introduced to political history. We moderns also
run through art galleries and hear concerts in the same way as the
young man runs through history. We can feel that one thing sounds
differently from another, and pronounce on the different \enquote{effects.}
And the power of gradually losing all feelings of strangeness or
astonishment, and finally being pleased at anything, is called the
historical sense, or historical culture. The crowd of influences
streaming on the young soul is so great, the clods of barbarism and
violence flung at him so strange and overwhelming, that an assumed
stupidity is his only refuge. Where there is a subtler and stronger
self-consciousness we find another emotion too -- disgust. The young
man has become homeless: he doubts all ideas, all moralities. He
knows \enquote{it was different in every age, and what you are does not
matter.} In a heavy apathy he lets opinion on opinion pass by him,
and understands the meaning of Hölderlin's\footnote{Johann Christian Friedrich Hölderlin (1770 –- 1843) was a German poet and philosopher.} words when he read the
work of Diogenes Laertius\footnote{Diogenes Laërtius (3rd century AD) was a biographer of the Greek philosophers; his \textit{Lives and Opinions of Eminent Philosophers} is a foundational source for the history of ancient Greek philosophy.} on the lives and doctrines of the Greek
philosophers: \enquote{I have seen here too what has often occurred to me,
that the change and waste in men's thoughts and systems is far more
tragic than the fates that overtake what men are accustomed to call
the only realities.} No, such study of history bewilders and
overwhelms. It is not necessary for youth, as the ancients show, but
even in the highest degree dangerous, as the moderns show. Consider
the historical student, the heir of ennui, that appears even in his
boyhood. He has the \enquote{methods} for original work, the \enquote{correct ideas}
and the airs of the master at his fingers' ends. A little isolated
period of the past is marked out for sacrifice. He cleverly applies
his method, and produces something, or rather, in prouder phrase,
\enquote{creates} something. He becomes a \enquote{servant of truth} and a ruler in
the great domain of history. If he was what they call ripe as a boy,
he is now over-ripe. You only need shake him and wisdom will rattle
down into your lap; but the wisdom is rotten, and every apple has its
worm. Believe me, if men work in the factory of science and have to
make themselves useful before they are really ripe, science is ruined
as much as the slaves who have been employed too soon. I am sorry to
use the common jargon about slave-owners and taskmasters in respect
of such conditions, that might be thought free from any economic
taint: but the words \enquote{factory, labour-market, auction-sale, practical
use,} and all the auxiliaries of egoism, come involuntarily to the
lips in describing the younger generation of savants. Successful
mediocrity tends to become still more mediocre, science still more
\enquote{useful.} Our modern savants are only wise on one subject, in all the
rest they are, to say the least, different from those of the old
stamp. In spite of that they demand honour and profit for themselves,
as if the state and public opinion were bound to take the new coinage
for the same value as the old. The carters have made a trade-compact
among themselves, and settled that genius is superfluous, for every
carrier is being re-stamped as one. And probably a later age will see
that their edifices are only carted together and not built. To those
who have ever on their lips the modern cry of battle and
sacrifice -- \enquote{Division of labour! fall into line!} we may say roundly:
\enquote{If you try to further the progress of science as quickly as
possible, you will end by destroying it as quickly as possible; just
as the hen is worn out which you force to lay too many eggs.} The
progress of science has been amazingly rapid in the last decade; but
consider the savants, those exhausted hens. They are certainly not
\enquote{harmonious} natures: they can merely cackle more than before,
because they lay eggs oftener: but the eggs are always smaller,
though their books are bigger. The natural result of it all is the
favourite \enquote{popularising} of science (or rather its feminising and
infantising), the villainous habit of cutting the cloth of science to
fit the figure of the \enquote{general public.} Goethe saw the abuse in this,
and demanded that science should only influence the outer world by
way of \textit{a nobler ideal of action}. The older generation of savants
had good reason for thinking this abuse an oppressive burden: the
modern savants have an equally good reason for welcoming it, because,
leaving their little corner of knowledge out of account, they are
part of the \enquote{general public} themselves, and its needs are theirs.
They only require to take themselves less seriously to be able to
open their little kingdom successfully to popular curiosity. This
easy-going behaviour is called \enquote{the modest condescension of the
savant to the people}; whereas in reality he has only \enquote{descended} to
himself, so far as he is not a savant but a plebeian. Rise to the
conception of a people, you learned men; you can never have one noble
or high enough. If you thought much of the people, you would have
compassion towards them, and shrink from offering your historical
aquafortis as a refreshing drink. But you really think very little of
them, for you dare not take any reasonable pains for their future;
and you act like practical pessimists, men who feel the coming
catastrophe and become indifferent and careless of their own and
others' existence. \enquote{If only the earth last for us: and if it do not
last, it is no matter.} Thus they come to live an \textit{ironical}
existence.

\newthought{This is how history can serve life.} Every man and nation needs a
certain knowledge of the past, whether it be through monumental,
antiquarian, or critical history, according to his objects, powers,
and necessities. The need is not that of the mere thinkers who only
look on at life, or the few who desire knowledge and can only be
satisfied with knowledge; but it has always a reference to the end of
life, and is under its absolute rule and direction. This is the
natural relation of an age, a culture and a people to history; hunger
is its source, necessity its norm, the inner plastic power assigns
its limits. The knowledge of the past is only desired for the service
of the future and the present, not to weaken the present or undermine
a living future. All this is as simple as truth itself, and quite
convincing to any one who is not in the toils of \enquote{historical
deduction.}

And now to take a quick glance at our time! We fly back in
astonishment. The clearness, naturalness, and purity of the
connection between life and history has vanished; and in what a maze
of exaggeration and contradiction do we now see the problem! Is the
guilt ours who see it, or have life and history really altered their
conjunction and an inauspicious star risen between them? Others may
prove we have seen falsely; I am merely saying what we believe we
see. There is such a star, a bright and lordly star, and the
conjunction is really altered -- by science, and the demand for history
to be a science. Life is no more dominant, and knowledge of the past
no longer its thrall: boundary marks are overthrown everything bursts
its limits. The perspective of events is blurred, and the blur
extends through their whole immeasurable course. No generation has
seen such a panoramic comedy as is shown by the \enquote{science of universal
evolution,} history; that shows it with the dangerous audacity of its
motto -- \enquote{Fiat veritas, pereat vita.}\footnote{\textit{Fiat veritas, pereat vita} -- \enquote{Let truth be, let life perish.} Similar to the more frequently cited \textit{Fiat iustitia, et pereat mundus}, a Latin phrase, meaning \enquote{Let justice be done, though the world perish}.}

Let me give a picture of the spiritual events in the soul of the
modern man. Historical knowledge streams on him from sources that are
inexhaustible, strange incoherencies come together, memory opens all
its gates and yet is never open wide enough, nature busies herself to
receive all the foreign guests, to honour them and put them in their
places. But they are at war with each other: violent measures seem
necessary, in order to escape destruction one's self. It becomes
second nature to grow gradually accustomed to this irregular and
stormy home-life, though this second nature is unquestionably weaker,
more restless, more radically unsound than the first. The modern man
carries inside him an enormous heap of indigestible knowledge-stones
that occasionally rattle together in his body, as the fairy-tale has
it. And the rattle reveals the most striking characteristic of these
modern men, the opposition of something inside them to which nothing
external corresponds; and the reverse. The ancient nations knew
nothing of this. Knowledge, taken in excess without hunger, even
contrary to desire, has no more the effect of transforming the
external life; and remains hidden in a chaotic inner world that the
modern man has a curious pride in calling his \enquote{real personality.} He
has the substance, he says, and only wants the form; but this is
quite an unreal opposition in a living thing. Our modern culture is
for that reason not a living one, because it cannot be understood
without that opposition. In other words, it is not a real culture but
a kind of knowledge about culture, a complex of various thoughts and
feelings about it, from which no decision as to its direction can
come. Its real motive force that issues in visible action is often no
more than a mere convention, a wretched imitation, or even a
shameless caricature. The man probably feels like the snake that has
swallowed a rabbit whole and lies still in the sun, avoiding all
movement not absolutely necessary. The \enquote{inner life} is now the only
thing that matters to education, and all who see it hope that the
education may not fail by being too indigestible. Imagine a Greek
meeting it; he would observe that for modern men \enquote{education} and
\enquote{historical education} seem to mean the same thing, with the
difference that the one phrase is longer. And if he spoke of his own
theory, that a man can be very well educated without any history at
all, people would shake their heads and think they had not heard
aright. The Greeks, the famous people of a past still near to us, had
the \enquote{unhistorical sense} strongly developed in the period of the
greatest power. If a typical child of this age were transported to
that world by some enchantment, he would probably find the Greeks
very \enquote{uneducated.} And that discovery would betray the closely
guarded secret of modern culture to the laughter of the world. For we
moderns have nothing of our own. We only become worth notice by
filling ourselves to overflowing with foreign customs, arts,
philosophies, religions and sciences: we are wandering encyclopædias,
as an ancient Greek who had strayed into our time would probably call
us. But the only value of an encyclopædia lies in the inside, in the
contents, not in what is written outside, in the binding or the
wrapper. And so the whole of modern culture is essentially internal;
the bookbinder prints something like this on the cover: \enquote{Manual of
internal culture for external barbarians.} The opposition of inner
and outer makes the outer side still more barbarous, as it would
naturally be, when the outward growth of a rude people merely
developed its primitive inner needs. For what means has nature of
repressing too great a luxuriance from without? Only one, -- to be
affected by it as little as possible, to set it aside and stamp it
out at the first opportunity. And so we have the custom of no longer
taking real things seriously, we get the feeble personality on which
the real and the permanent make so little impression. Men become at
last more careless and accommodating in external matters, and the
considerable cleft between substance and form is widened; until they
have no longer any feeling for barbarism, if only their memories be
kept continually titillated, and there flow a constant stream of new
things to be known, that can be neatly packed up in the cupboards of
their memory. The culture of a people as against this barbarism, can
be, I think, described with justice as the \enquote{unity of artistic style
in every outward expression of the people's life.} This must not be
misunderstood, as though it were merely a question of the opposition
between barbarism and \enquote{fine style.} The people that can be called
cultured, must be in a real sense a living unity, and not be
miserably cleft asunder into form and substance. If one wish to
promote a people's culture, let him try to promote this higher unity
first, and work for the destruction of the modern educative system
for the sake of a true education. Let him dare to consider how the
health of a people that has been destroyed by history may be
restored, and how it may recover its instincts with its honour.

I am only speaking, directly, about the Germans of the present day,
who have had to suffer more than other people from the feebleness of
personality and the opposition of substance and form. \enquote{Form}
generally implies for us some convention, disguise or hypocrisy, and
if not hated, is at any rate not loved. We have an extraordinary fear
of both the word convention and the thing. This fear drove the German
from the French school; for he wished to become more natural, and
therefore more German. But he seems to have come to a false
conclusion with his \enquote{therefore.} First he ran away from his school of
convention, and went by any road he liked: he has come ultimately to
imitate voluntarily in a slovenly fashion, what he imitated painfully
and often successfully before. So now the lazy fellow lives under
French conventions that are actually incorrect: his manner of walking
shows it, his conversation and dress, his general way of life. In the
belief that he was returning to Nature, he merely followed caprice
and comfort, with the smallest possible amount of self-control. Go
through any German town; you will see conventions that are nothing
but the negative aspect of the national characteristics of foreign
states. Everything is colourless, worn out, shoddy and ill-copied.
Every one acts at his own sweet will -- which is not a strong or
serious will -- on laws dictated by the universal rush and the general
desire for comfort. A dress that made no head ache in its inventing
and wasted no time in the making, borrowed from foreign models and
imperfectly copied, is regarded as an important contribution to
German fashion. The sense of form is ironically disclaimed by the
people -- for they have the \enquote{sense of substance}: they are famous for
their cult of \enquote{inwardness.}

But there is also a famous danger in their \enquote{inwardness}: the internal
substance cannot be seen from the outside, and so may one day take
the opportunity of vanishing, and no one notice its absence, any more
than its presence before. One may think the German people to be very
far from this danger: yet the foreigner will have some warrant for
his reproach that our inward life is too weak and ill-organised to
provide a form and external expression for itself. It may in rare
cases show itself finely receptive, earnest and powerful, richer
perhaps than the inward life of other peoples; but, taken as a whole,
it remains weak, as all its fine threads are not tied together in one
strong knot. The visible action is not the self-manifestation of the
inward life, but only a weak and crude attempt of a single thread to
make a show of representing the whole. And thus the German is not to
be judged on any one action, for the individual may be as completely
obscure after it as before. He must obviously be measured by his
thoughts and feelings, which are now expressed in his books; if only
the books did not, more than ever, raise the doubt whether the famous
inward life is still really sitting in its inaccessible shrine. It
might one day vanish and leave behind it only the external
life, -- with its vulgar pride and vain servility, -- to mark the German.
Fearful thought! -- as fearful as if the inward life still sat there,
painted and rouged and disguised, become a play-actress or something
worse; as his theatrical experience seems to have taught the quiet
observer Grillparzer\footnote{Franz Seraphicus Grillparzer (1791 -– 1872) was considered to be the leading Austrian dramatist of the 19th century.}, standing aside as he did from the main press.
\enquote{We feel by theory,} he says. \enquote{We hardly know any more how our
contemporaries give expression to their feelings: we make them use
gestures that are impossible nowadays. Shakespeare has spoilt us
moderns.}

This is a single example, its general application perhaps too hastily
assumed. But how terrible it would be were that generalisation
justified before our eyes! There would be then a note of despair in
the phrase, \enquote{We Germans feel by theory, we are all spoilt by
history;} -- a phrase that would cut at the roots of any hope for a
future national culture. For every hope of that kind grows from the
belief in the genuineness and immediacy of German feeling, from the
belief in an untarnished inward life. Where is our hope or belief,
when its spring is muddied, and the inward quality has learned
gestures and dances and the use of cosmetics, has learned to express
itself \enquote{with due reflection in abstract terms,} and gradually lose
itself? And how should a great productive spirit exist among a nation
that is not sure of its inward unity and is divided into educated men
whose inner life has been drawn from the true path of education, and
uneducated men whose inner life cannot be approached at all? How
should it exist, I say, when the people has lost its own unity of
feeling, and knows that the feeling of the part calling itself the
educated part and claiming the right of controlling the artistic
spirit of the nation, is false and hypocritical? Here and there the
judgment and taste of individuals may be higher and finer than the
rest, but that is no compensation: it tortures a man to have to speak
only to one section and be no longer in sympathy with his people. He
would rather bury his treasure now, in disgust at the vulgar
patronage of a class, though his heart be filled with tenderness for
all. The instinct of the people can no longer meet him half-way; it
is useless for them to stretch their arms out to him in yearning.
What remains but to turn his quickened hatred against the ban, strike
at the barrier raised by the so-called culture, and condemn as judge
what blasted and degraded him as a living man and a source of life?
He takes a profound insight into fate in exchange for the godlike
desire of creation and help, and ends his days as a lonely
philosopher, with the wisdom of disillusion. It is the painfullest
comedy: he who sees it will feel a sacred obligation on him, and say
to himself, -- \enquote{Help must come: the higher unity in the nature and soul
of a people must be brought back, the cleft between inner and outer
must again disappear under the hammer of necessity.} But to what
means can he look? What remains to him now but his knowledge? He
hopes to plant the feeling of a need, by speaking from the breadth of
that knowledge, giving it freely with both hands. From the strong
need the strong action may one day arise. And to leave no doubt of
the instance I am taking of the need and the knowledge, my testimony
shall stand, that it is German unity in its highest sense which is
the goal of our endeavour, far more than political union: it is the
unity of the German spirit and life after the annihilation of the
antagonism between form and substance, inward life and convention.

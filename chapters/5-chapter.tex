\newthought{An excess of history} seems to be an enemy to the life of a time, and
dangerous in five ways. Firstly, the contrast of inner and outer is
emphasised and personality weakened. Secondly, the time comes to
imagine that it possesses the rarest of virtues, justice, to a higher
degree than any other time. Thirdly, the instincts of a nation are
thwarted, the maturity of the individual arrested no less than that
of the whole. Fourthly, we get the belief in the old age of mankind,
the belief, at all times harmful, that we are late survivals, mere
Epigoni\footnote{In Greek mythology, the Epigoni (\enquote{offspring}) are the sons of the Argive heroes, who had been killed in the first Theban war. The second Theban war (the war of the Epigoni), occurred ten years later, when the Epigoni, wishing to avenge the death of their fathers, attacked Thebes.}. Lastly, an age reaches a dangerous condition of irony with
regard to itself, and the still more dangerous state of cynicism,
when a cunning egoistic theory of action is matured that maims and at
last destroys the vital strength.

To return to the first point: the modern man suffers from a weakened
personality. The Roman of the Empire ceased to be a Roman through the
contemplation of the world that lay at his feet; he lost himself in
the crowd of foreigners that streamed into Rome, and degenerated amid
the cosmopolitan carnival of arts, worships and moralities. It is the
same with the modern man, who is continually having a world-panorama
unrolled before his eyes by his historical artists. He is turned into
a restless, dilettante spectator, and arrives at a condition when
even great wars and revolutions cannot affect him beyond the moment.
The war is hardly at an end, and it is already converted into
thousands of copies of printed matter, and will be soon served up as
the latest means of tickling the jaded palates of the historical
gourmets. It seems impossible for a strong full chord to be
prolonged, however powerfully the strings are swept: it dies away
again the next moment in the soft and strengthless echo of history.
In ethical language, one never succeeds in staying on a height; your
deeds are sudden crashes, and not a long roll of thunder. One may
bring the greatest and most marvellous thing to perfection; it must
yet go down to Orcus\footnote{Orcus was a god of the underworld, punisher of broken oaths in Etruscan and Roman mythology. As with Hades, the name of the god was also used for the underworld itself.} unhonoured and unsung. For art flies away when
you are roofing your deeds with the historical awning. The man who
wishes to understand everything in a moment, when he ought to grasp
the unintelligible as the sublime by a long struggle, can be called
intelligent only in the sense of Schiller's epigram on the "reason of
reasonable men." There is something the child sees that he does not
see; something the child hears that he does not hear; and this
something is the most important thing of all. Because he does not
understand it, his understanding is more childish than the child's
and more simple than simplicity itself; in spite of the many clever
wrinkles on his parchment face, and the masterly play of his fingers
in unravelling the knots. He has lost or destroyed his instinct; he
can no longer trust the "divine animal" and let the reins hang loose,
when his understanding fails him and his way lies through the desert.
His individuality is shaken, and left without any sure belief in
itself; it sinks into its own inner being, which only means here the
disordered chaos of what it has learned, which will never express
itself externally, being mere dogma that cannot turn to life. Looking
further, we see how the banishment of instinct by history has turned
men to shades and abstractions: no one ventures to show a
personality, but masks himself as a man of culture, a savant, poet or
politician.

If one take hold of these masks, believing he has to do with a
serious thing and not a mere puppet-show -- for they all have an
appearance of seriousness -- he will find nothing but rags and coloured
streamers in his hands. He must deceive himself no more, but cry
aloud, "Off with your jackets, or be what you seem!" A man of the
royal stock of seriousness must no longer be Don Quixote, for he has
better things to do than to tilt at such pretended realities. But he
must always keep a sharp look about him, call his "Halt! who goes
there?" to all the shrouded figures, and tear the masks from their
faces. And see the result! One might have thought that history
encouraged men above all to be honest, even if it were only to be
honest fools: this used to be its effect, but is so no longer.
Historical education and the uniform frock-coat of the citizen are
both dominant at the same time. While there has never been such a
full-throated chatter about "free personality," personalities can be
seen no more (to say nothing of free ones); but merely men in
uniform, with their coats anxiously pulled over their ears.
Individuality has withdrawn itself to its recesses; it is seen no
more from the outside, which makes one doubt if it be possible to
have causes without effects. Or will a race of eunuchs prove to be
necessary to guard the historical harem of the world? We can
understand the reason for their aloofness very well. Does it not seem
as if their task were to watch over history to see that nothing comes
out except other histories, but no deed that might be historical; to
prevent personalities becoming "free," that is, sincere towards
themselves and others, both in word and deed? Only through this
sincerity will the inner need and misery of the modern man be brought
to the light, and art and religion come as true helpers in the place
of that sad hypocrisy of convention and masquerade, to plant a common
culture which will answer to real necessities, and not teach, as the
present "liberal education" teaches, to tell lies about these needs,
and thus become a walking lie one's self.

In such an age, that suffers from its "liberal education," how
unnatural, artificial and unworthy will be the conditions under which
the sincerest of all sciences, the holy naked goddess Philosophy,
must exist! She remains, in such a world of compulsion and outward
conformity, the subject of the deep monologue of the lonely wanderer
or the chance prey of any hunter, the dark secret of the chamber or
the daily talk of the old men and children at the university. No one
dare fulfil the law of philosophy in himself; no one lives
philosophically, with that single-hearted virile faith that forced
one of the olden time to bear himself as a Stoic, wherever he was and
whatever he did, if he had once sworn allegiance to the Stoa. All
modern philosophising is political or official, bound down to be a
mere phantasmagoria of learning by our modern governments, churches,
universities, moralities and cowardices: it lives by sighing "if
only...." and by knowing that "it happened once upon a time...."
Philosophy has no place in historical education, if it will be more
than the knowledge that lives indoors, and can have no expression in
action. Were the modern man once courageous and determined, and not
merely such an indoor being even in his hatreds, he would banish
philosophy. At present he is satisfied with modestly covering her
nakedness. Yes, men think, write, print, speak and teach
philosophically: so much is permitted them. It is only otherwise in
action, in "life." Only one thing is permitted there, and everything
else quite impossible: such are the orders of historical education.
"Are these human beings," one might ask, "or only machines for
thinking, writing and speaking?"

Goethe says of Shakespeare: "No one has more despised correctness of
costume than he: he knows too well the inner costume that all men
wear alike. You hear that he describes Romans wonderfully; I do not
think so: they are flesh-and-blood Englishmen; but at any rate they
are men from top to toe, and the Roman toga sits well on them." Would
it be possible, I wonder, to represent our present literary and
national heroes, officials and politicians as Romans? I am sure it
would not, as they are no men, but incarnate compendia, abstractions
made concrete. If they have a character of their own, it is so deeply
sunk that it can never rise to the light of day: if they are men,
they are only men to a physiologist. To all others they are something
else, not men, not "beasts or gods," but historical pictures of the
march of civilisation, and nothing but pictures and civilisation,
form without any ascertainable substance, bad form unfortunately, and
uniform at that. And in this way my thesis is to be understood and
considered: "only strong personalities can endure history, the weak
are extinguished by it." History unsettles the feelings when they are
not powerful enough to measure the past by themselves. The man who
dare no longer trust himself, but asks history against his will for
advice "how he ought to feel now," is insensibly turned by his
timidity into a play-actor, and plays a part, or generally many
parts, -- very badly therefore and superficially. Gradually all
connection ceases between the man and his historical subjects. We see
noisy little fellows measuring themselves with the Romans as though
they were like them: they burrow in the remains of the Greek poets,
as if these were \textit{corpora}\footnote{\textit{corpora}  --  body} for their dissection -- and as \textit{vilia}\footnote{\textit{vilia} -- cheap} as
their own well-educated \textit{corpora} might be. Suppose a man is working
at Democritus. The question is always on my tongue, why precisely
Democritus? Why not Heraclitus, or Philo, or Bacon, or Descartes? And
then, why a philosopher? Why not a poet or orator? And why especially
a Greek? Why not an Englishman or a Turk? Is not the past large
enough to let you find some place where you may disport yourself
without becoming ridiculous? But, as I said, they are a race of
eunuchs: and to the eunuch one woman is the same as another, merely a
woman, "woman in herself," the Ever-unapproachable. And it is
indifferent what they study, if history itself always remain
beautifully "objective" to them, as men, in fact, who could never
make history themselves. And since the Eternal Feminine could never
"draw you upward," you draw it down to you, and being neuter
yourselves, regard history as neuter also. But in order that no one
may take any comparison of history and the Eternal Feminine too
seriously, I will say at once that I hold it, on the contrary, to be
the Eternal Masculine: I only add that for those who are
"historically trained" throughout, it must be quite indifferent which
it is; for they are themselves neither man nor woman, nor even
hermaphrodite, but mere neuters, or, in more philosophic language,
the Eternal Objective.

If the personality be once emptied of its subjectivity, and come to
what men call an "objective" condition, nothing can have any more
effect on it. Something good and true may be done, in action, poetry
or music: but the hollow culture of the day will look beyond the work
and ask the history of the author. If the author have already created
something, our historian will set out clearly the past and the
probable future course of his development, he will put him with
others and compare them, and separate by analysis the choice of his
material and his treatment; he will wisely sum the author up and give
him general advice for his future path. The most astonishing works
may be created; the swarm of historical neuters will always be in
their place, ready to consider the authors through their long
telescopes. The echo is heard at once: but always in the form of
"criticism," though the critic never dreamed of the work's
possibility a moment before. It never comes to have an influence, but
only a criticism: and the criticism itself has no influence, but only
breeds another criticism. And so we come to consider the fact of many
critics as a mark of influence, that of few or none as a mark of
failure. Actually everything remains in the old condition, even in
the presence of such "influence": men talk a little while of a new
thing, and then of some other new thing, and in the meantime they do
what they have always done. The historical training of our critics
prevents their having an influence in the true sense, an influence on
life and action. They put their blotting paper on the blackest
writing, and their thick brushes over the gracefullest designs; these
they call "corrections"; -- and that is all. Their critical pens never
cease to fly, for they have lost power over them; they are driven by
their pens instead of driving them. The weakness of modern
personality comes out well in the measureless overflow of criticism,
in the want of self-mastery, and in what the Romans called
\textit{impotentia}.


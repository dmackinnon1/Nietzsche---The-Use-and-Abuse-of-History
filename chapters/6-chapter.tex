\newthought{But leaving these weaklings,} let us turn rather to a point of
strength for which the modern man is famous. Let us ask the painful
question whether he has the right in virtue of his historical
\enquote{objectivity} to call himself strong and just in a higher degree than
the man of another age. Is it true that this objectivity has its
source in a heightened sense of the need for justice? Or, being
really an effect of quite other causes, does it only have the
appearance of coming from justice, and really lead to an unhealthy
prejudice in favour of the modern man? Socrates thought it near
madness to imagine one possessed a virtue without really possessing
it. Such imagination has certainly more danger in it than the
contrary madness of a positive vice. For of this there is still a
cure; but the other makes a man or a time daily worse, and therefore
more unjust.

No one has a higher claim to our reverence than the man with the
feeling and the strength for justice. For the highest and rarest
virtues unite and are lost in it, as an unfathomable sea absorbs the
streams that flow from every side. The hand of the just man, who is
called to sit in judgment, trembles no more when it holds the scales:
he piles the weights inexorably against his own side, his eyes are
not dimmed as the balance rises and falls, and his voice is neither
hard nor broken when he pronounces the sentence. Were he a cold demon
of knowledge, he would cast round him the icy atmosphere of an awful,
superhuman majesty, that we should fear, not reverence. But he is a
man, and has tried to rise from a careless doubt to a strong
certainty, from gentle tolerance to the imperative \enquote{thou must}; from
the rare virtue of magnanimity to the rarest, of justice. He has come
to be like that demon without being more than a poor mortal at the
outset; above all, he has to atone to himself for his humanity and
tragically shatter his own nature on the rock of an impossible
virtue. -- All this places him on a lonely height as the most reverend
example of the human race. For truth is his aim, not in the form of
cold intellectual knowledge, but the truth of the judge who punishes
according to law; not as the selfish possession of an individual, but
the sacred authority that removes the boundary stones from all
selfish possessions; truth, in a word, as the tribunal of the world,
and not as the chance prey of a single hunter. The search for truth
is often thoughtlessly praised: but it only has anything great in it
if the seeker have the sincere unconditional will for justice. Its
roots are in justice alone: but a whole crowd of different motives
may combine in the search for it, that have nothing to do with truth
at all; curiosity, for example, or dread of ennui, envy, vanity, or
amusement. Thus the world seems to be full of men who \enquote{serve truth}:
and yet the virtue of justice is seldom present, more seldom known,
and almost always mortally hated. On the other hand a throng of sham
virtues has entered in at all times with pomp and honour.

Few in truth serve truth, as only few have the pure will for justice;
and very few even of these have the strength to be just. The will
alone is not enough: the impulse to justice without the power of
judgment has been the cause of the greatest suffering to men. And
thus the common good could require nothing better than for the seed
of this power to be strewn as widely as possible, that the fanatic
may be distinguished from the true judge, and the blind desire from
the conscious power. But there are no means of planting a power of
judgment: and so when one speaks to men of truth and justice, they
will be ever troubled by the doubt whether it be the fanatic or the
judge who is speaking to them. And they must be pardoned for always
treating the \enquote{servants of truth} with special kindness, who possess
neither the will nor the power to judge and have set before them the
task of finding \enquote{pure knowledge without reference to consequences,}
knowledge, in plain terms, that comes to nothing. There are very many
truths which are unimportant; problems that require no struggle to
solve, to say nothing of sacrifice. And in this safe realm of
indifference a man may very successfully become a \enquote{cold demon of
knowledge.} And yet -- if we find whole regiments of learned inquirers
being turned to such demons in some age specially favourable to them,
it is always unfortunately possible that the age is lacking in a
great and strong sense of justice, the noblest spring of the
so-called impulse to truth.

Consider the historical virtuoso of the present time: is he the
justest man of his age? True, he has developed in himself such a
delicacy and sensitiveness that \enquote{nothing human is alien to him.}\footnote{Adapted from a saying of Publius Terentius Afer (c195/185 –- c.159? BC), better known in English as Terence, a Roman playwright -- \enquote{I am a human being, and thus nothing human is alien to me.} }
Times and persons most widely separated come together in the concords
of his lyre. He has become a passive instrument, whose tones find an
echo in similar instruments: until the whole atmosphere of a time is
filled with such echoes, all buzzing in one soft chord. Yet I think
one only hears the overtones of the original historical note: its
rough powerful quality can be no longer guessed from these thin and
shrill vibrations. The original note sang of action, need, and
terror; the overtone lulls us into a soft dilettante sleep. It is as
though the heroic symphony had been arranged for two flutes for the
use of dreaming opium-smokers. We can now judge how these virtuosi
stand towards the claim of the modern man to a higher and purer
conception of justice. This virtue has never a pleasing quality; it
never charms; it is harsh and strident. Generosity stands very low on
the ladder of the virtues in comparison; and generosity is the mark
of a few rare historians! Most of them only get as far as tolerance,
in other words they leave what cannot be explained away, they correct
it and touch it up condescendingly, on the tacit assumption that the
novice will count it as justice if the past be narrated without
harshness or open expressions of hatred. But only superior strength
can really judge; weakness must tolerate, if it do not pretend to be
strength and turn justice to a play-actress. There is still a
dreadful class of historians remaining -- clever, stern and honest, but
narrow-minded: who have the \enquote{good will} to be just with a pathetic
belief in their actual judgments, which are all false; for the same
reason, almost, as the verdicts of the usual juries are false. How
difficult it is to find a real historical talent, if we exclude all
the disguised egoists, and the partisans who pretend to take up an
impartial attitude for the sake of their own unholy game! And we also
exclude the thoughtless folk who write history in the naïve faith
that justice resides in the popular view of their time, and that to
write in the spirit of the time is to be just; a faith that is found
in all religions, and which, in religion, serves very well. The
measurement of the opinions and deeds of the past by the universal
opinions of the present is called \enquote{objectivity} by these simple
people: they find the canon of all truth here: their work is to adapt
the past to the present triviality. And they call all historical
writing \enquote{subjective} that does not regard these popular opinions as
canonical.

Might not an illusion lurk in the highest interpretation of the word
objectivity? We understand by it a certain standpoint in the
historian, who sees the procession of motive and consequence too
clearly for it to have an effect on his own personality. We think of
the æsthetic phenomenon of the detachment from all personal concern
with which the painter sees the picture and forgets himself, in a
stormy landscape, amid thunder and lightning, or on a rough sea: and
we require the same artistic vision and absorption in his object from
the historian. But it is only a superstition to say that the picture
given to such a man by the object really shows the truth of things.
Unless it be that objects are expected in such moments to paint or
photograph themselves by their own activity on a purely passive
medium!

But this would be a myth, and a bad one at that. One forgets that
this moment is actually the powerful and spontaneous moment of
creation in the artist, of \enquote{composition} in its highest form, of
which the result will be an artistically, but not an historically,
true picture. To think objectively, in this sense, of history is the
work of the dramatist: to think one thing with another, and weave the
elements into a single whole; with the presumption that the unity of
plan must be put into the objects if it be not already there. So man
veils and subdues the past, and expresses his impulse to art -- but not
his impulse to truth or justice. Objectivity and justice have nothing
to do with each other. There could be a kind of historical writing
that had no drop of common fact in it and yet could claim to be
called in the highest degree objective. Grillparzer goes so far as to
say that \enquote{history is nothing but the manner in which the spirit of
man apprehends facts that are obscure to him, links things together
whose connection heaven only knows, replaces the unintelligible by
something intelligible, puts his own ideas of causation into the
external world, which can perhaps be explained only from within: and
assumes the existence of chance, where thousands of small causes may
be really at work. Each man has his own individual needs, and so
millions of tendencies are running together, straight or crooked,
parallel or across, forward or backward, helping or hindering each
other. They have all the appearance of chance, and make it
impossible, quite apart from all natural influences, to establish any
universal lines on which past events must have run.} But as a result
of this so-called \enquote{objective} way of looking at things, such a \enquote{must}
ought to be made clear. It is a presumption that takes a curious form
if adopted by the historian as a dogma. Schiller is quite clear about
its truly subjective nature when he says of the historian, \enquote{one event
after the other begins to draw away from blind chance and lawless
freedom, and take its place as the member of an harmonious
whole -- \textit{which is of course only apparent in its presentation}.} But
what is one to think of the innocent statement, wavering between
tautology and nonsense, of a famous historical virtuoso? \enquote{It seems
that all human actions and impulses are subordinate to the process of
the material world, that works unnoticed, powerfully and
irresistibly.} In such a sentence one no longer finds obscure wisdom
in the form of obvious folly; as in the saying of Goethe's gardener,
\enquote{Nature may be forced but not compelled,} or in the notice on the
side-show at a fair, in Swift: \enquote{The largest elephant in the world,
except himself, to be seen here.} For what opposition is there
between human action and the process of the world? It seems to me
that such historians cease to be instructive as soon as they begin to
generalise; their weakness is shown by their obscurity. In other
sciences the generalisations are the most important things, as they
contain the laws. But if such generalisations as these are to stand
as laws, the historian's labour is lost; for the residue of truth,
after the obscure and insoluble part is removed, is nothing but the
commonest knowledge. The smallest range of experience will teach it.
But to worry whole peoples for the purpose, and spend many hard years
of work on it, is like crowding one scientific experiment on another
long after the law can be deduced from the results already obtained:
and this absurd excess of experiment has been the bane of all natural
science since Zollner. If the value of a drama lay merely in its
final scene, the drama itself would be a very long, crooked and
laborious road to the goal: and I hope history will not find its
whole significance in general propositions, and regard them as its
blossom and fruit. On the contrary, its real value lies in inventing
ingenious variations on a probably commonplace theme, in raising the
popular melody to a universal symbol and showing what a world of
depth, power and beauty exists in it.

But this requires above all a great artistic faculty, a creative
vision from a height, the loving study of the data of experience, the
free elaborating of a given type, -- objectivity in fact, though this
time as a positive quality. Objectivity is so often merely a phrase.
Instead of the quiet gaze of the artist that is lit by an inward
flame, we have an affectation of tranquillity; just as a cold
detachment may mask a lack of moral feeling. In some cases a
triviality of thought, the everyday wisdom that is too dull not to
seem calm and disinterested, comes to represent the artistic
condition in which the subjective side has quite sunk out of sight.
Everything is favoured that does not rouse emotion, and the driest
phrase is the correct one. They go so far as to accept a man who is
\textit{not affected at all} by some particular moment in the past as the
right man to describe it. This is the usual relation of the Greeks
and the classical scholars. They have nothing to do with each
other -- and this is called \enquote{objectivity}! The intentional air of
detachment that is assumed for effect, the sober art of the
superficial motive-hunter is most exasperating when the highest and
rarest things are in question; and it is the \textit{vanity} of the
historian that drives him to this attitude of indifference. He goes
to justify the axiom that a man's vanity corresponds to his lack of
wit. No, be honest at any rate! Do not pretend to the artist's
strength, that is the real objectivity; do not try to be just, if you
are not born to that dread vocation. As if it were the task of every
time to be just to everything before it! Ages and generations have
never the right to be the judges of all previous ages and
generations: only to the rarest men in them can that difficult
mission fall. Who compels you to judge? If it is your wish -- you must
prove first that you are capable of justice. As judges, you must
stand higher than that which is to be judged: as it is, you have only
come later. The guests that come last to the table should rightly
take the last places: and will you take the first? Then do some great
and mighty deed: the place may be prepared for you then, even though
you do come last.

\newthought{You can only explain the past by what is highest in the present.}
Only by straining the noblest qualities you have to their highest
power will you find out what is greatest in the past, most worth
knowing and preserving. Like by like! otherwise you will draw the
past to your own level. Do not believe any history that does not
spring from the mind of a rare spirit. You will know the quality of
the spirit, by its being forced to say something universal, or to
repeat something that is known already; the fine historian must have
the power of coining the known into a thing never heard before and
proclaiming the universal so simply and profoundly that the simple is
lost in the profound, and the profound in the simple. No one can be a
great historian and artist, and a shallowpate at the same time. But
one must not despise the workers who sift and cast together the
material because they can never become great historians. They must,
still less, be confounded with them, for they are the necessary
bricklayers and apprentices in the service of the master: just as the
French used to speak, more naïvely than a German would, of the
\enquote{historiens de M. Thiers.} These workmen should gradually become
extremely learned, but never, for that reason, turn to be masters.
Great learning and great shallowness go together very well under one
hat.

Thus, history is to be written by the man of experience and
character. He who has not lived through something greater and nobler
than others, will not be able to explain anything great and noble in
the past. The language of the past is always oracular: you will only
understand it as builders of the future who know the present. We can
only explain the extraordinarily wide influence of Delphi by the fact
that the Delphic priests had an exact knowledge of the past: and,
similarly, only he who is building up the future has a right to judge
the past. If you set a great aim before your eyes, you control at the
same time the itch for analysis that makes the present into a desert
for you, and all rest, all peaceful growth and ripening, impossible.
Hedge yourselves with a great, all-embracing hope, and strive on.
Make of yourselves a mirror where the future may see itself, and
forget the superstition that you are Epigoni. You have enough to
ponder and find out, in pondering the life of the future: but do not
ask history to show you the means and the instrument to it. If you
live yourselves back into the history of great men, you will find in
it the high command to come to maturity and leave that blighting
system of cultivation offered by your time: which sees its own profit
in not allowing you to become ripe, that it may use and dominate you
while you are yet unripe. And if you want biographies, do not look
for those with the legend \enquote{Mr. So-and-so and his times,} but for one
whose title-page might be inscribed \enquote{a fighter against his time.}
Feast your souls on Plutarch, and dare to believe in yourselves when
you believe in his heroes. A hundred such men -- educated against the
fashion of to-day, made familiar with the heroic, and come to
maturity -- are enough to give an eternal quietus to the noisy sham
education of this time.

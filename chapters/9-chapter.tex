\newthought{Is perhaps our time such a \enquote{first-comer}?} Its historical sense is so
strong, and has such universal and boundless expression, that future
times will commend it, if only for this, as a first-comer -- if there
be any future time, in the sense of future culture. But here comes a
grave doubt. Close to the modern man's pride there stands his irony
about himself, his consciousness that he must live in a historical,
or twilit, atmosphere, the fear that he can retain none of his
youthful hopes and powers. Here and there one goes further into
cynicism, and justifies the course of history, nay, the whole
evolution of the world, as simply leading up to the modern man,
according to the cynical canon: -- \enquote{what you see now had to come, man
had to be thus and not otherwise, no one can stand against this
necessity.} He who cannot rest in a state of irony flies for refuge
to the cynicism. The last decade makes him a present of one of its
most beautiful inventions, a full and well-rounded phrase for this
cynicism: he calls his way of living thoughtlessly and after the
fashion of his time, \enquote{the full surrender of his personality to the
world-process.} The personality and the world-process! The
world-process and the personality of the earthworm! If only one did
not eternally hear the word \enquote{world, world, world,} that hyperbole of
all hyperboles; when we should only speak, in a decent manner, of
\enquote{man, man, man}! Heirs of the Greeks and Romans, of Christianity? All
that seems nothing to the cynics. But \enquote{heirs of the world-process};
the final target of the world-process; the meaning and solution of
all riddles of the universe, the ripest fruit on the tree of
knowledge! -- that is what I call a right noble thought: by this token
are the firstlings of every time to be known, although they may have
arrived last. The historical imagination has never flown so far, even
in a dream; for now the history of man is merely the continuation of
that of animals and plants: the universal historian finds traces of
himself even in the utter depths of the sea, in the living slime. He
stands astounded in face of the enormous way that man has run, and
his gaze quivers before the mightier wonder, the modern man who can
see all this way! He stands proudly on the pyramid of the
world-process: and while he lays the final stone of his knowledge, he
seems to cry aloud to listening Nature: \enquote{We are at the top, we are
the top, we are the completion of Nature!}

O thou too proud European of the nineteenth century, art thou not
mad? Thy knowledge does not complete Nature, it only kills thine own
nature! Measure the height of what thou knowest by the depths of thy
power to \textit{do}. Thou climbest the sunbeams of knowledge up towards
heaven -- but also down to Chaos. Thy manner of going is fatal to thee;
the ground slips from under thy feet into the unknown; thy life has
no other stay, but only spider's webs that every new stroke of thy
knowledge tears asunder. -- But not another serious word about this,
for there is a lighter side to it all.

The moralist, the artist, the saint and the statesman may well be
troubled, when they see that all foundations are breaking up in mad
unconscious ruin, and resolving themselves into the ever flowing
stream of becoming; that all creation is being tirelessly spun into
webs of history by the modern man, the great spider in the mesh of
the world-net. We ourselves may be glad for once in a way that we see
it all in the shining magic mirror of a philosophical parodist, in
whose brain the time has come to an ironical consciousness of itself,
to a point even of wickedness, in Goethe's phrase. Hegel once said,
\enquote{when the spirit makes a fresh start, we philosophers are at hand.}
Our time did make a fresh start -- into irony, and lo! Edward von
Hartmann\footnote{Karl Robert Eduard von Hartmann (1842 –- 1906) was a German philosopher and author of \textit{Philosophy of the Unconscious} (1869).} was at hand, with his famous \textit{Philosophy of the
Unconscious} -- or, more plainly, his philosophy of unconscious irony.
We have seldom read a more jovial production, a greater philosophical
joke than Hartmann's book. Any one whom it does not fully enlighten
about \enquote{becoming,} who is not swept and garnished throughout by it, is
ready to become a monument of the past himself. The beginning and end
of the world-process, from the first throb of consciousness to its
final leap into nothingness, with the task of our generation settled
for it; -- all drawn from that clever fount of inspiration, the
Unconscious, and glittering in Apocalyptic light, imitating an honest
seriousness to the life, as if it were a serious philosophy and not a
huge joke, -- such a system shows its creator to be one of the first
philosophical parodists of all time. Let us then sacrifice on his
altar, and offer the inventor of a true universal medicine a lock of
hair, in Schleiermacher's phrase. For what medicine would be more
salutary to combat the excess of historical culture than Hartmann's
parody of the world's history?

If we wished to express in the fewest words what Hartmann really has
to tell us from his mephitic tripod of unconscious irony, it would be
something like this: our time could only remain as it is, if men
should become thoroughly sick of this existence. And I fervently
believe he is right. The frightful petrifaction of the time, the
restless rattle of the ghostly bones, held naïvely up to us by David
Strauss as the most beautiful fact of all -- is justified by Hartmann
not only from the past, \textit{ex causis efficientibus}, but also from the
future, \textit{ex causa finali}. The rogue let light stream over our time
from the last day, and saw that it was very good, -- for him, that is,
who wishes to feel the indigestibility of life at its full strength,
and for whom the last day cannot come quickly enough. True, Hartmann
calls the old age of life that mankind is approaching the \enquote{old age of
man}: but that is the blessed state, according to him, where there is
only a successful mediocrity; where art is the \enquote{evening's amusement
of the Berlin financier,} and \enquote{the time has no more need for
geniuses, either because it would be casting pearls before swine, or
because the time has advanced beyond the stage where the geniuses are
found, to one more important,} to that stage of social evolution, in
fact, in which every worker \enquote{leads a comfortable existence, with
hours of work that leave him sufficient leisure to cultivate his
intellect.} Rogue of rogues, you say well what is the aspiration of
present-day mankind: but you know too what a spectre of disgust will
arise at the end of this old age of mankind, as the result of the
intellectual culture of stolid mediocrity. It is very pitiful to see,
but it will be still more pitiful yet. \enquote{Antichrist is visibly
extending his arms:} yet it \textit{must be so}, for after all we are on the
right road -- of disgust at all existence. \enquote{Forward then, boldly, with
the world-process, as workers in the vineyard of the Lord, for it is
the process alone that can lead to redemption!}

The vineyard of the Lord! The process! To redemption! Who does not
see and hear in this how historical culture, that only knows the word
\enquote{becoming,} parodies itself on purpose and says the most
irresponsible things about itself through its grotesque mask? For
what does the rogue mean by this cry to the workers in the vineyard?
By what \enquote{work} are they to strive boldly forward? Or, to ask another
question: -- what further has the historically educated fanatic of the
world-process to do, -- swimming and drowning as he is in the sea of
becoming, -- that he may at last gather in that vintage of disgust, the
precious grape of the vineyard? He has nothing to do but to live on
as he has lived, love what he has loved, hate what he has hated, and
read the newspapers he has always read. The only sin is for him to
live otherwise than he has lived. We are told how he has lived, with
monumental clearness, by that famous page with its large typed
sentences, on which the whole rabble of our modern cultured folk have
thrown themselves in blind ecstasy, because they believe they read
their own justification there, haloed with an Apocalyptic light. For
the unconscious parodist has demanded of every one of them, \enquote{the full
surrender of his personality to the world-process, for the sake of
his end, the redemption of the world}: or still more clearly, -- \enquote{the
assertion of the will to live is proclaimed to be the first step on
the right road: for it is only in the full surrender to life and its
sorrow, and not in the cowardice of personal renunciation and
retreat, that anything can be done for the world-process.... The
striving for the denial of the individual will is as foolish as it is
useless, more foolish even than suicide.... The thoughtful reader
will understand without further explanation how a practical
philosophy can be erected on these principles, and that such a
philosophy cannot endure any disunion, but only the fullest
reconciliation with life.}

The thoughtful reader will understand! Then one really could
misunderstand Hartmann! And what a splendid joke it is, that he
should be misunderstood! Why should the Germans of to-day be
particularly subtle? A valiant Englishman looks in vain for \enquote{delicacy
of perception} and dares to say that \enquote{in the German mind there does
seem to be something splay, something blunt-edged, unhandy and
infelicitous.} Could the great German parodist contradict this?
According to him, we are approaching \enquote{that ideal condition in which
the human race makes its history with full consciousness}: but we are
obviously far from the perhaps more ideal condition, in which mankind
can read Hartmann's book with full consciousness. If we once reach
it, the word \enquote{world-process} will never pass any man's lips again
without a smile. For he will remember the time when people listened
to the mock gospel of Hartmann, sucked it in, attacked it, reverenced
it, extended it and canonised it with all the honesty of that \enquote{German mind,} with \enquote{the uncanny seriousness of an owl,} as Goethe has it.
But the world must go forward, the ideal condition cannot be won by
dreaming, it must be fought and wrestled for, and the way to
redemption lies only through joyousness, the way to redemption from
that dull, owlish seriousness. The time will come when we shall
wisely keep away from all constructions of the world-process, or even
of the history of man; a time when we shall no more look at masses
but at individuals, who form a sort of bridge over the wan stream of
becoming. They may not perhaps continue a process, but they live out
of time, as contemporaries: and thanks to history that permits such a
company, they live as the Republic of geniuses of which Schopenhauer
speaks. One giant calls to the other across the waste spaces of time,
and the high spirit-talk goes on, undisturbed by the wanton noisy
dwarfs who creep among them. The task of history is to be the
mediator between these, and even to give the motive and power to
produce the great man. The aim of mankind can lie ultimately only in
its highest examples.

Our low comedian has his word on this too, with his wonderful
dialectic, which is just as genuine as its admirers are admirable.
\enquote{The idea of evolution cannot stand with our giving the world-process
an endless duration in the past, for thus every conceivable evolution
must have taken place, which is not the case (O rogue!); and so we
cannot allow the process an endless duration in the future. Both
would raise the conception of evolution to a mere ideal (And again
rogue!), and would make the world-process like the sieve of the
Danaides. The complete victory of the logical over the illogical (O
thou complete rogue!) must coincide with the last day, the end in
time of the world-process.} No, thou clear, scornful spirit, so long
as the illogical rules as it does to-day, -- so long, for example, as
the world-process can be spoken of as thou speakest of it, amid such
deep-throated assent, -- the last day is yet far off. For it is still
too joyful on this earth, many an illusion still blooms here -- like
the illusion of thy contemporaries about thee. We are not yet ripe to
be hurled into thy nothingness: for we believe that we shall have a
still more splendid time, when men once begin to understand thee,
thou misunderstood, unconscious one! But if, in spite of that,
disgust shall come throned in power, as thou hast prophesied to thy
readers; if thy portrayal of the present and the future shall prove
to be right, -- and no one has despised them with such loathing as
thou, -- I am ready then to cry with the majority in the form
prescribed by thee, that next Saturday evening, punctually at twelve
o'clock, thy world shall fall to pieces. And our decree shall
conclude thus -- from to-morrow time shall not exist, and the \textit{Times}
shall no more be published. Perhaps it will be in vain, and our
decree of no avail: at any rate we have still time for a fine
experiment. Take a balance and put Hartmann's \enquote{Unconscious} in one of
the scales, and his \enquote{World-process} in the other. There are some who
believe they weigh equally; for in each scale there is an evil
word -- and a good joke.

When they are once understood, no one will take Hartmann's words on
the world-process as anything but a joke. It is, as a fact, high time
to move forward with the whole battalion of satire and malice against
the excesses of the \enquote{historical sense,} the wanton love of the
world-process at the expense of life and existence, the blind
confusion of all perspective. And it will be to the credit of the
philosopher of the Unconscious that he has been the first to see the
humour of the world-process, and to succeed in making others see it
still more strongly by the extraordinary seriousness of his
presentation. The existence of the \enquote{world} and \enquote{humanity} need not
trouble us for some time, except to provide us with a good joke: for
the presumption of the small earthworm is the most uproariously comic
thing on the face of the earth. Ask thyself to what end thou art
here, as an individual; and if no one can tell thee, try then to
justify the meaning of thy existence \textit{a posteriori}, by putting
before thyself a high and noble end. Perish on that rock! I know no
better aim for life than to be broken on something great and
impossible, \textit{animæ magnæ prodigus}. But if we have the doctrines of
the finality of \enquote{becoming,} of the flux of all ideas, types, and
species, of the lack of all radical difference between man and beast
(a true but fatal idea as I think), -- if we have these thrust on the
people in the usual mad way for another generation, no one need be
surprised if that people drown on its little miserable shoals of
egoism, and petrify in its self-seeking. At first it will fall
asunder and cease to be a people. In its place perhaps individualist
systems, secret societies for the extermination of non-members, and
similar utilitarian creations, will appear on the theatre of the
future. Are we to continue to work for these creations and write
history from the standpoint of the \textit{masses}; to look for laws in it,
to be deduced from the needs of the masses, the laws of motion of the
lowest loam and clay strata of society? The masses seem to be worth
notice in three aspects only: first as the copies of great men,
printed on bad paper from worn-out plates, next as a contrast to the
great men, and lastly as their tools: for the rest, let the devil and
statistics fly away with them! How could statistics prove that there
are laws in history? Laws? Yes, they may prove how common and
abominably uniform the masses are: and should we call the effects of
leaden folly, imitation, love and hunger -- laws? We may admit it: but
we are sure of this too -- that so far as there are laws in history,
the laws are of no value and the history of no value either. And
least valuable of all is that kind of history which takes the great
popular movements as the most important events of the past, and
regards the great men only as their clearest expression, the visible
bubbles on the stream. Thus the masses have to produce the great man,
chaos to bring forth order; and finally all the hymns are naturally
sung to the teeming chaos. Everything is called \enquote{great} that has
moved the masses for some long time, and becomes, as they say, a
\enquote{historical power.} But is not this really an intentional confusion
of quantity and quality? When the brutish mob have found some idea, a
religious idea for example, which satisfies them, when they have
defended it through thick and thin for centuries then, and then only,
will they discover its inventor to have been a great man. The highest
and noblest does not affect the masses at all. The historical
consequences of Christianity, its \enquote{historical power,} toughness and
persistence prove nothing, fortunately, as to its founder's
greatness, They would have been a witness against him. For between
him and the historical success of Christianity lies a dark heavy
weight of passion and error, lust of power and honour, and the
crushing force of the Roman Empire. From this, Christianity had its
earthly taste, and its earthly foundations too, that made its
continuance in this world possible. Greatness should not depend on
success; Demosthenes is great without it. The purest and noblest
adherents of Christianity have always doubted and hindered, rather
than helped, its effect in the world, its so-called \enquote{historical
power}; for they were accustomed to stand outside the \enquote{world,} and
cared little for the \enquote{process of the Christian Idea.} Hence they have
generally remained unknown to history, and their very names are lost.
In Christian terms the devil is the prince of the world, and the lord
of progress and consequence: he is the power behind all \enquote{historical
power,} and so will it remain, however ill it may sound to-day in
ears that are accustomed to canonise such power and consequence. The
world has become skilled at giving new names to things and even
baptizing the devil. It is truly an hour of great danger. Men seem to
be near the discovery that the egoism of individuals, groups or
masses has been at all times the lever of the \enquote{historical movements}:
and yet they are in no way disturbed by the discovery, but proclaim
that \enquote{egoism shall be our god.} With this new faith in their hearts,
they begin quite intentionally to build future history on egoism:
though it must be a clever egoism, one that allows of some
limitation, that it may stand firmer; one that studies history for
the purpose of recognising the foolish kind of egoism. Their study
has taught them that the state has a special mission in all future
egoistic systems: it will be the patron of all the clever egoisms, to
protect them with all the power of its military and police against
the dangerous outbreaks of the other kind. There is the same idea in
introducing history -- natural as well as human history -- among the
labouring classes, whose folly makes them dangerous. For men know
well that a grain of historical culture is able to break down the
rough, blind instincts and desires, or to turn them to the service of
a clever egoism. In fact they are beginning to think, with Edward von
Hartmann, of \enquote{fixing themselves with an eye to the future in their
earthly home, and making themselves comfortable there.} Hartmann
calls this life the \enquote{manhood of humanity} with an ironical reference
to what is now called \enquote{manhood}; -- as if only our sober models of
selfishness were embraced by it; just as he prophesies an age of
graybeards following on this stage, -- obviously another ironical
glance at our ancient time-servers. For he speaks of the ripe
discretion with which \enquote{they view all the stormy passions of their
past life and understand the vanity of the ends they seem to have
striven for.} No, a manhood of crafty and historically cultured
egoism corresponds to an old age that hangs to life with no dignity
but a horrible tenacity, where the
\begin{quote}
                           \hfill \ldots Last scene of all\\
  That ends this strange eventful history,\\
  Is second childishness and mere oblivion,\\
  Sans teeth, sans eyes, sans taste, sans everything.
\footnote{From Shakespeare's \textit{As You Like It}, Act II, scene VII, line 139, a speech by Jaques which begins with \enquote{All the world's a stage,
And all the men and women merely Players;
They have their exits and their entrances,\ldots}}
\end{quote}
Whether the dangers of our life and culture come from these dreary,
toothless old men, or from the so-called \enquote{men} of Hartmann, we have
the right to defend our youth with tooth and claw against both of
them, and never tire of saving the future from these false prophets.
But in this battle we shall discover an unpleasant truth -- that men
intentionally help, and encourage, and use, the worst aberrations of
the historical sense from which the present time suffers.

They use it, however, against youth, in order to transform it into
that ripe \enquote{egoism of manhood} they so long for: they use it to
overcome the natural reluctance of the young by its magical
splendour, which unmans while it enlightens them. Yes, we know only
too well the kind of ascendency history can gain; how it can uproot
the strongest instincts of youth, passion, courage, unselfishness and
love; can cool its feeling for justice, can crush or repress its
desire for a slow ripening by the contrary desire to be soon
productive, ready and useful; and cast a sick doubt over all honesty
and downrightness of feeling. It can even cozen youth of its fairest
privilege, the power of planting a great thought with the fullest
confidence, and letting it grow of itself to a still greater thought.
An excess of history can do all that, as we have seen, by no longer
allowing a man to feel and act \textit{unhistorically}: for history is
continually shifting his horizon and removing the atmosphere
surrounding him. From an infinite horizon he withdraws into himself,
back into the small egoistic circle, where he must become dry and
withered: he may possibly attain to cleverness, but never to wisdom.
He lets himself be talked over, is always calculating and parleying
with facts. He is never enthusiastic, but blinks his eyes, and
understands how to look for his own profit or his party's in the
profit or loss of somebody else. He unlearns all his useless modesty,
and turns little by little into the \enquote{man} or the \enquote{graybeard} of
Hartmann. And that is what they \textit{want} him to be: that is the meaning
of the present cynical demand for the \enquote{full surrender of the
personality to the world-process} -- for the sake of his end, the
redemption of the world, as the rogue E. von Hartmann tells us.
Though redemption can scarcely be the conscious aim of these people:
the world were better redeemed by being redeemed from these \enquote{men} and
\enquote{graybeards.} For then would come the reign of youth.

\newthought{And in this kingdom of youth} I can cry Land! Land! Enough, and more
than enough, of the wild voyage over dark strange seas, of eternal
search and eternal disappointment! The coast is at last in sight.
Whatever it be, we must land there, and the worst haven is better
than tossing again in the hopeless waves of an infinite scepticism.
Let us hold fast by the land: we shall find the good harbours later
and make the voyage easier for those who come after us.

The voyage was dangerous and exciting. How far are we even now from
that quiet state of contemplation with which we first saw our ship
launched! In tracking out the dangers of history, we have found
ourselves especially exposed to them. We carry on us the marks of
that sorrow which an excess of history brings in its train to the men
of the modern time. And this present treatise, as I will not attempt
to deny, shows the modern note of a weak personality in the
intemperateness of its criticism, the unripeness of its humanity, in
the too frequent transitions from irony to cynicism, from arrogance
to scepticism. And yet I trust in the inspiring power that directs my
vessel instead of genius; I trust in \textit{youth}, that has brought me on
the right road in forcing from me a protest against the modern
historical education, and a demand that the man must learn to live,
above all, and only use history in the service of the life that he
has learned to live. He must be young to understand this protest; and
considering the premature grayness of our present youth, he can
scarcely be young enough if he would understand its reason as well.
An example will help me. In Germany, not more than a century ago, a
natural instinct for what is called \enquote{poetry} was awakened in some
young men. Are we to think that the generations who had lived before
that time had not spoken of the art, however really strange and
unnatural it may have been to them? We know the contrary; that they
had thought, written, and quarrelled about it with all their
might -- in \enquote{words, words, words.} Giving life to such words did not
prove the death of the word-makers; in a certain sense they are
living still. For if, as Gibbon says, nothing but time -- though a long
time -- is needed for a world to perish, so nothing but time -- though
still more time -- is needed for a false idea to be destroyed in
Germany, the \enquote{Land of Little-by-little.} In any event, there are
perhaps a hundred men more now than there were a century ago who know
what poetry is: perhaps in another century there will be a hundred
more who have learned in the meantime what culture is, and that the
Germans have had as yet no culture, however proudly they may talk
about it. The general satisfaction of the Germans at their culture
will seem as foolish and incredible to such men as the once lauded
classicism of Gottsched, or the reputation of Ramler as the German
Pindar, seemed to us. They will perhaps think this \enquote{culture} to be
merely a kind of knowledge about culture, and a false and superficial
knowledge at that. False and superficial, because the Germans endured
the contradiction between life and knowledge, and did not see what
was characteristic in the culture of really educated peoples, that it
can only rise and bloom from life. But by the Germans it is worn like
a paper flower, or spread over like the icing on a cake; and so must
remain a useless lie for ever.

The education of youth in Germany starts from this false and
unfruitful idea of culture. Its aim, when faced squarely, is not to
form the liberally educated man, but the professor, the man of
science, who wants to be able to make use of his science as soon as
possible, and stands on one side in order to see life clearly. The
result, even from a ruthlessly practical point of view, is the
historically and æsthetically trained Philistine, the babbler of old
saws and new wisdom on Church, State and Art, the sensorium that
receives a thousand impressions, the insatiable belly that yet knows
not what true hunger and thirst is. An education with such an aim and
result is against nature. But only he who is not quite drowned in it
can feel that; only youth can feel it, because it still has the
instinct of nature, that is the first to be broken by that education.
But he who will break through that education in his turn, must come
to the help of youth when called upon; must let the clear light of
understanding shine on its unconscious striving, and bring it to a
full, vocal consciousness. How is he to attain such a strange end?

Principally by destroying the superstition that this kind of
education is \textit{necessary}. People think nothing but this troublesome
reality of ours is possible. Look through the literature of higher
education in school and college for the last ten years, and you will
be astonished -- and pained -- to find how much alike all the proposals
of reform have been; in spite of all the hesitations and violent
controversies surrounding them. You will see how blindly they have
all adopted the old idea of the \enquote{educated man} (in our sense) being
the necessary and reasonable basis of the system. The monotonous
canon runs thus: the young man must begin with a knowledge of
culture, not even with a knowledge of life, still less with life and
the living of it. This knowledge of culture is forced into the young
mind in the form of historical knowledge; which means that his head
is filled with an enormous mass of ideas, taken second-hand from past
times and peoples, not from immediate contact with life. He desires
to experience something for himself, and feel a close-knit, living
system of experiences growing within himself. But his desire is
drowned and dizzied in the sea of shams, as if it were possible to
sum up in a few years the highest and notablest experiences of
ancient times, and the greatest times too. It is the same mad method
that carries our young artists off to picture-galleries, instead of
the studio of a master, and above all the one studio of the only
master, Nature. As if one could discover by a hasty rush through
history the ideas and technique of past times, and their individual
outlook on life! For life itself is a kind of handicraft that must be
learned thoroughly and industriously, and diligently practised, if we
are not to have mere botchers and babblers as the issue of it all!

Plato thought it necessary for the first generation of his new
society (in the perfect state) to be brought up with the help of a
\enquote{mighty lie.} The children were to be taught to believe that they had
all lain dreaming for a long time under the earth, where they had
been moulded and formed by the master-hand of Nature. It was
impossible to go against the past, and work against the work of gods!
And so it had to be an unbreakable law of nature, that he who is born
to be a philosopher has gold in his body, the fighter has only
silver, and the workman iron and bronze. As it is not possible to
blend these metals, according to Plato, so there could never be any
confusion between the classes: the belief in the \textit{æterna veritas} of
this arrangement was the basis of the new education and the new
state. So the modern German believes also in the \textit{æterna veritas} of
his education, of his kind of culture: and yet this belief will
fail -- as the Platonic state would have failed -- if the mighty German
lie be ever opposed by the truth, that the German has no culture
because he cannot build one on the basis of his education. He wishes
for the flower without the root or the stalk; and so he wishes in
vain. That is the simple truth, a rude and unpleasant truth, but yet
a mighty one.

But our first generation must be brought up in this \enquote{mighty truth,}
and must suffer from it too; for it must educate itself through it,
even against its own nature, to attain a new nature and manner of
life, which shall yet proceed from the old. So it might say to
itself, in the old Spanish phrase, \enquote{Defienda me Dios de my,} God keep
me from myself, from the character, that is, which has been put into
me. It must taste that truth drop by drop, like a bitter, powerful
medicine. And every man in this generation must subdue himself to
pass the judgment on his own nature, which he might pass more easily
on his whole time: -- \enquote{We are without instruction, nay, we are too
corrupt to live, to see and hear truly and simply, to understand what
is near and natural to us. We have not yet laid even the foundations
of culture, for we are not ourselves convinced that we have a sincere
life in us.} We crumble and fall asunder, our whole being is divided,
half mechanically, into an inner and outer side; we are sown with
ideas as with dragon's teeth, and bring forth a new dragon-brood of
them; we suffer from the malady of words, and have no trust in any
feeling that is not stamped with its special word. And being such a
dead fabric of words and ideas, that yet has an uncanny movement in
it, I have still perhaps the right to say \textit{cogito ergo sum}, though
not \textit{vivo ergo cogito}\footnote{
\textit{cogito ergo sum} -- \enquote{I think, therefore I am;} \textit{vivo ergo cogito} -- \enquote{I live, therefore I think.}}. I am permitted the empty \textit{esse}, not the full
green \textit{vivere}\footnote{\textit{esse} -- \enquote{to be;} \textit{vivere} -- \enquote{to live}}. A primary feeling tells me that I am a thinking being
but not a living one, that I am no \enquote{animal,} but at most a \enquote{cogital.}
\enquote{Give me life, and I will soon make you a culture out of it}--will be
the cry of every man in this new generation, and they will all know
each other by this cry. But who will give them this life?

No god and no man will give it--only their own \textit{youth}. Set this
free, and you will set life free as well. For it only lay concealed,
in a prison; it is not yet withered or dead--ask your own selves!

But it is sick, this life that is set free, and must be healed. It
suffers from many diseases, and not only from the memory of its
chains. It suffers from the malady which I have spoken of, the
\textit{malady of history}. Excess of history has attacked the plastic power
of life, that no more understands how to use the past as a means of
strength and nourishment. It is a fearful disease, and yet, if youth
had not a natural gift for clear vision, no one would see that it is
a disease, and that a paradise of health has been lost. But the same
youth, with that same natural instinct of health, has guessed how the
paradise can be regained. It knows the magic herbs and simples for
the malady of history, and the excess of it. And what are they
called?

It is no marvel that they bear the names of poisons:--the antidotes
to history are the \enquote{unhistorical} and the \enquote{super-historical.} With
these names we return to the beginning of our inquiry and draw near
to its final close.

By the word \enquote{unhistorical} I mean the power, the art of \textit{forgetting},
and of drawing a limited horizon round one's self. I call the power
\enquote{super-historical} which turns the eyes from the process of becoming
to that which gives existence an eternal and stable character, to art
and religion. Science--for it is science that makes us speak of
\enquote{poisons}--sees in these powers contrary powers: for it considers
only that view of things to be true and right, and therefore
scientific, which regards something as finished and historical, not
as continuing and eternal. Thus it lives in a deep antagonism towards
the powers that make for eternity--art and religion,--for it hates
the forgetfulness that is the death of knowledge, and tries to remove
all limitation of horizon and cast men into an infinite boundless
sea, whose waves are bright with the clear knowledge--of becoming!

If they could only live therein! Just as towns are shaken by an
avalanche and become desolate, and man builds his house there in fear
and for a season only; so life is broken in sunder and becomes weak
and spiritless, if the avalanche of ideas started by science take
from man the foundation of his rest and security, the belief in what
is stable and eternal. Must life dominate knowledge, or knowledge
life? Which of the two is the higher, and decisive power? There is no
room for doubt: life is the higher, and the dominating power, for the
knowledge that annihilated life would be itself annihilated too.
Knowledge presupposes life, and has the same interest in maintaining
it that every creature has in its own preservation. Science needs
very careful watching: there is a hygiene of life near the volumes of
science, and one of its sentences runs thus:--The unhistorical and
the super-historical are the natural antidotes against the
overpowering of life by history; they are the cures for the
historical disease. We who are sick of the disease may suffer a
little from the antidote. But this is no proof that the treatment we
have chosen is wrong.

And here I see the mission of the youth that forms the first
generation of fighters and dragon-slayers: it will bring a more
beautiful and blessed humanity and culture, but will have itself no
more than a glimpse of the promised land of happiness and wondrous
beauty. This youth will suffer both from the malady and its
antidotes: and yet it believes in strength and health and boasts a
nature closer to the great Nature than its forebears, the cultured
men and graybeards of the present. But its mission is to shake to
their foundations the present conceptions of \enquote{health} and \enquote{culture,}
and erect hatred and scorn in the place of this rococo mass of ideas.
And the clearest sign of its own strength and health is just the fact
that it can use no idea, no party-cry from the present-day mint of
words and ideas to symbolise its own existence: but only claims
conviction from the power in it that acts and fights, breaks up and
destroys; and from an ever heightened feeling of life when the hour
strikes. You may deny this youth any culture--but how would youth
count that a reproach? You may speak of its rawness and
intemperateness--but it is not yet old and wise enough to be
acquiescent. It need not pretend to a ready-made culture at all; but
enjoys all the rights--and the consolations--of youth, especially the
right of brave unthinking honesty and the consolation of an inspiring
hope.

I know that such hopeful beings understand all these truisms from
within, and can translate them into a doctrine for their own use,
through their personal experience. To the others there will appear,
in the meantime, nothing but a row of covered dishes, that may
perhaps seem empty: until they see one day with astonished eyes that
the dishes are full, and that all ideas and impulses and passions are
massed together in these truisms that cannot lie covered for long. I
leave those doubting ones to time, that brings all things to light;
and turn at last to that great company of hope, to tell them the way
and the course of their salvation, their rescue from the disease of
history, and their own history as well, in a parable; whereby they
may again become healthy enough to study history anew, and under the
guidance of life make use of the past in that threefold
way--monumental, antiquarian, or critical. At first they will be more
ignorant than the \enquote{educated men} of the present: for they will have
unlearnt much and have lost any desire even to discover what those
educated men especially wish to know: in fact, their chief mark from
the educated point of view will be just their want of science; their
indifference and inaccessibility to all the good and famous things.
But at the end of the cure, they are men again and have ceased to be
mere shadows of humanity. That is something; there is yet hope, and
do not ye who hope laugh in your hearts?

How can we reach that end? you will ask. The Delphian god cries his
oracle to you at the beginning of your wanderings, \enquote{Know thyself.} It
is a hard saying: for that god \enquote{tells nothing and conceals nothing
but merely points the way,} as Heraclitus said. But whither does he
point?

In certain epochs the Greeks were in a similar danger of being
overwhelmed by what was past and foreign, and perishing on the rock
of \enquote{history.} They never lived proud and untouched. Their \enquote{culture}
was for a long time a chaos of foreign forms and ideas,--Semitic,
Babylonian, Lydian and Egyptian,--and their religion a battle of all
the gods of the East; just as German culture and religion is at
present a death-struggle of all foreign nations and bygone times. And
yet, Hellenic culture was no mere mechanical unity, thanks to that
Delphic oracle. The Greeks gradually learned to organise the chaos,
by taking Apollo's advice and thinking back to themselves, to their
own true necessities, and letting all the sham necessities go. Thus
they again came into possession of themselves, and did not remain
long the Epigoni of the whole East, burdened with their inheritance.
After that hard fight, they increased and enriched the treasure they
had inherited by their obedience to the oracle, and they became the
ancestors and models for all the cultured nations of the future.

This is a parable for each one of us: he must organise the chaos in
himself by \enquote{thinking himself back} to his true needs. He will want
all his honesty, all the sturdiness and sincerity in his character to
help him to revolt against second-hand thought, second-hand learning,
second-hand action. And he will begin then to understand that culture
can be something more than a \enquote{decoration of life} -- a concealment and
disfiguring of it, in other words; for all adornment hides what is
adorned. And thus the Greek idea, as against the Roman, will be
discovered in him, the idea of culture as a new and finer nature,
without distinction of inner and outer, without convention or
disguise, as a unity of thought and will, life and appearance. He
will learn too, from his own experience, that it was by a greater
force of moral character that the Greeks were victorious, and that
everything which makes for sincerity is a further step towards true
culture, however this sincerity may harm the ideals of education that
are reverenced at the time, or even have power to shatter a whole
system of merely decorative culture.


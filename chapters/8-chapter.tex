\newthought{It may seem a paradox,} though it is none, that I should attribute a
kind of \enquote{ironical self-consciousness} to an age that is generally so
honestly, and clamorously, vain of its historical training; and
should see a suspicion hovering near it that there is really nothing
to be proud of, and a fear lest the time for rejoicing at historical
knowledge may soon have gone by. Goethe has shown a similar riddle in
man's nature, in his remarkable study of Newton: he finds a \enquote{troubled
feeling of his own error} at the base -- or rather on the height -- of
his being, just as if he was conscious at times of having a deeper
insight into things, that vanished the moment after. This gave him a
certain ironical view of his own nature. And one finds that the
greater and more developed \enquote{historical men} are conscious of all the
superstition and absurdity in the belief that a people's education
need be so extremely historical as it is; the mightiest nations,
mightiest in action and influence, have lived otherwise, and their
youth has been trained otherwise. The knowledge gives a sceptical
turn to their minds. \enquote{The absurdity and superstition,} these sceptics
say, \enquote{suit men like ourselves, who come as the latest withered shoots
of a gladder and mightier stock, and fulfil Hesiod's prophecy, that
men will one day be born gray-headed, and that Zeus will destroy that
generation as soon as the sign be visible.}\footnote{Hesiod, a Greek poet (700 BC) wrote in \textit{Works and Days}: \enquote{And Zeus will destroy this race of mortal men also when they come to have grey hair on the temples at their birth. The father will not agree with his children, nor the children with their father, nor guest with his host, nor comrade with comrade; nor will brother be dear to brother as aforetime.}} Historical culture is
really a kind of inherited grayness, and those who have borne its
mark from childhood must believe instinctively in \textit{the old age of
mankind}. To old age belongs the old man's business of looking back
and casting up his accounts, of seeking consolation in the memories
of the past, -- in historical culture. But the human race is tough and
persistent, and will not admit that the lapse of a thousand years, or
a hundred thousand, entitles any one to sum up its progress from the
past to the future; that is, it will not be observed as a whole at
all by that infinitesimal atom, the individual man. What is there in
a couple of thousand years -- the period of thirty-four consecutive
human lives of sixty years each -- to make us speak of youth at the
beginning, and \enquote{the old age of mankind} at the end of them? Does not
this paralysing belief in a fast-fading humanity cover the
misunderstanding of a theological idea, inherited from the Middle
Ages, that the end of the world is approaching and we are waiting
anxiously for the judgment? Does not the increasing demand for
historical judgment give us that idea in a new dress? as if our time
were the latest possible time, and commanded to hold that universal
judgment of the past, which the Christian never expected from a man,
but from \enquote{the Son of Man.} The \textit{memento mori},\footnote{\textit{Memento mori} (Latin for \enquote{remember that you have to die}) is a symbol of the inevitability of death -- usually a skull.} spoken to humanity as
well as the individual, was a sting that never ceased to pain, the
crown of mediæval knowledge and consciousness.

The opposite message of a later time, \textit{memento vivere}\footnote{A compliment to \textit{memento mori}, \textit{memento vivere} is \enquote{remember that you must live}}, is spoken
rather timidly, without the full power of the lungs; and there is
something almost dishonest about it. For mankind still keeps to its
\textit{memento mori}, and shows it by the universal need for history;
science may flap its wings as it will, it has never been able to gain
the free air. A deep feeling of hopelessness has remained, and taken
the historical colouring that has now darkened and depressed all
higher education. A religion that, of all the hours of man's life,
thinks the last the most important, that has prophesied the end of
earthly life and condemned all creatures to live in the fifth act of
a tragedy, may call forth the subtlest and noblest powers of man, but
it is an enemy to all new planting, to all bold attempts or free
aspirations. It opposes all flight into the unknown, because it has
no life or hope there itself. It only lets the new bud press forth on
sufferance, to blight it in its own good time: \enquote{it might lead life
astray and give it a false value.} What the Florentines did under the
influence of Savonarola's exhortations\footnote{Girolamo Savonarola (1452 –- 1498) a friar and preacher, urged the people of Florence to destroy  secular art and culture and expelled the ruling Medicis.}, when they made the famous
holocaust of pictures, manuscripts, masks and mirrors, Christianity
would like to do with every culture that allured to further effort
and bore that \textit{memento vivere} on its standard. And if it cannot take
the direct way -- the way of main force -- it gains its end all the same
by allying itself with historical culture, though generally without
its connivance; and speaking through its mouth, turns away every
fresh birth with a shrug of its shoulders, and makes us feel all the
more that we are late-comers and Epigoni, that we are, in a word,
born with gray hair. The deep and serious contemplation of the
unworthiness of all past action, of the world ripe for judgment, has
been whittled down to the sceptical consciousness that it is anyhow a
good thing to know all that has happened, as it is too late to do
anything better. The historical sense makes its servants passive and
retrospective. Only in moments of forgetfulness, when that sense is
dormant, does the man who is sick of the historical fever ever act;
though he only analyses his deed again after it is over (which
prevents it from having any further consequences), and finally puts
it on the dissecting table for the purposes of history. In this sense
we are still living in the Middle Ages, and history is still a
disguised theology; just as the reverence with which the unlearned
layman looks on the learned class is inherited through the clergy.
What men gave formerly to the Church they give now, though in smaller
measure, to science. But the fact of giving at all is the work of the
Church, not of the modern spirit, which among its other good
qualities has something of the miser in it, and is a bad hand at the
excellent virtue of liberality.

These words may not be very acceptable, any more than my derivation
of the excess of history from the mediæval \textit{memento mori} and the
hopelessness that Christianity bears in its heart towards all future
ages of earthly existence. But you should always try to replace my
hesitating explanations by a better one. For the origin of historical
culture, and of its absolutely radical antagonism to the spirit of a
new time and a \enquote{modern consciousness,} must itself be known by a
historical process. History must solve the problem of history,
science must turn its sting against itself. This threefold \enquote{must} is
the imperative of the \enquote{new spirit,} if it is really to contain
something new, powerful, vital and original. Or is it true that we
Germans -- to leave the Romance nations out of account -- must always be
mere \enquote{followers} in all the higher reaches of culture, because that
is all we \textit{can} be? The words of Wilhelm Wackernagel\footnote{Wilhelm Wackernagel (1806 -- 1869) was a German-Swiss philologist and historian of Germanic culture.} are well worth
pondering: \enquote{We Germans are a nation of 'followers,' and with all our
higher science and even our faith, are merely the successors of the
ancient world. Even those who are opposed to it are continually
breathing the immortal spirit of classical culture with that of
Christianity: and if any one could separate these two elements from
the living air surrounding the soul of man, there would not be much
remaining for a spiritual life to exist on.} Even if we would rest
content with our vocation to follow antiquity, even if we decided to
take it in an earnest and strenuous spirit and to show our high
prerogative in our earnestness, -- we should yet be compelled to ask
whether it were our eternal destiny to be pupils of a fading
antiquity. We might be allowed at some time to put our aim higher and
further above us. And after congratulating ourselves on having
brought that secondary spirit of Alexandrian culture in us to such
marvellous productiveness -- through our \enquote{universal history} -- we might
go on to place before us, as our noblest prize, the still higher task
of striving beyond and above this Alexandrian world; and bravely find
our prototypes in the ancient Greek world, where all was great,
natural and human. But it is just \textit{there} that we find the reality of
a true unhistorical culture -- and in spite of that, or perhaps because
of it, an unspeakably rich and vital culture. Were we Germans nothing
but followers, we could not be anything greater or prouder than the
lineal inheritors and followers of such a culture.

This however must be added. The thought of being Epigoni, that is
often a torture, can yet create a spring of hope for the future, to
the individual as well as the people: so far, that is, as we can
regard ourselves as the heirs and followers of the marvellous
classical power, and see therein both our honour and our spur. But
not as the late and bitter fruit of a powerful stock, giving that
stock a further spell of cold life, as antiquaries and grave-diggers.
Such late-comers live truly an ironical existence. Annihilation
follows their halting walk on tiptoe through life. They shudder
before it in the midst of their rejoicing over the past. They are
living memories, and their own memories have no meaning; for there
are none to inherit them. And thus they are wrapped in the melancholy
thought that their life is an injustice, which no future life can set
right again.

Suppose that these antiquaries, these late arrivals, were to change
their painful ironic modesty for a certain shamelessness. Suppose we
heard them saying, aloud, \enquote{The race is at its zenith, for it has
manifested itself consciously for the first time.} We should have a
comedy, in which the dark meaning of a certain very celebrated
philosophy would unroll itself for the benefit of German culture. I
believe there has been no dangerous turning-point in the progress of
German culture in this century that has not been made more dangerous
by the enormous and still living influence of this Hegelian
philosophy. The belief that one is a late-comer in the world is,
anyhow, harmful and degrading: but it must appear frightful and
devastating when it raises our late-comer to godhead, by a neat turn
of the wheel, as the true meaning and object of all past creation,
and his conscious misery is set up as the perfection of the world's
history. Such a point of view has accustomed the Germans to talk of a
\enquote{world-process,} and justify their own time as its necessary result.
And it has put history in the place of the other spiritual powers,
art and religion, as the one sovereign; inasmuch as it is the \enquote{Idea
realising itself,} the \enquote{Dialectic of the spirit of the nations,} and
the \enquote{tribunal of the world.}

History understood in this Hegelian way has been contemptuously
called God's sojourn upon earth, -- though the God was first created by
the history. He, at any rate, became transparent and intelligible
inside Hegelian skulls, and has risen through all the dialectically
possible steps in his being up to the manifestation of the Self: so
that for Hegel the highest and final stage of the world-process came
together in his own Berlin existence. He ought to have said that
everything after him was merely to be regarded as the musical coda of
the great historical rondo, -- or rather, as simply superfluous. He has
not said it; and thus he has implanted in a generation leavened
throughout by him the worship of the \enquote{power of history,} that
practically turns every moment into a sheer gaping at success, into
an idolatry of the actual: for which we have now discovered the
characteristic phrase \enquote{to adapt ourselves to circumstances.} But the
man who has once learnt to crook the knee and bow the head before the
power of history, nods \enquote{yes} at last, like a Chinese doll, to every
power, whether it be a government or a public opinion or a numerical
majority; and his limbs move correctly as the power pulls the string.
If each success have come by a \enquote{rational necessity,} and every event
show the victory of logic or the \enquote{Idea,} then -- down on your knees
quickly, and let every step in the ladder of success have its
reverence! There are no more living mythologies, you say? Religions
are at their last gasp? Look at the religion of the power of history,
and the priests of the mythology of Ideas, with their scarred knees!
Do not all the virtues follow in the train of the new faith? And
shall we not call it unselfishness, when the historical man lets
himself be turned into an \enquote{objective} mirror of all that is? Is it
not magnanimity to renounce all power in heaven and earth in order to
adore the mere fact of power? Is it not justice, always to hold the
balance of forces in your hands and observe which is the stronger and
heavier? And what a school of politeness is such a contemplation of
the past! To take everything objectively, to be angry at nothing, to
love nothing, to understand everything -- makes one gentle and pliable.
Even if a man brought up in this school will show himself openly
offended, one is just as pleased, knowing it is only meant in the
artistic sense of \textit{ira et studium}, though it is really \textit{sine ira et
studio}.

What old-fashioned thoughts I have on such a combination of virtue
and mythology! But they must out, however one may laugh at them. I
would even say that history always teaches -- \enquote{it was once,} and
morality -- \enquote{it ought not to be, or have been.} So history becomes a
compendium of actual immorality. But how wrong would one be to regard
history as the judge of this actual immorality! Morality is offended
by the fact that a Raphael had to die at thirty-six; such a being
ought not to die. If you came to the help of history, as the
apologists of the actual, you would say: \enquote{he had spoken everything
that was in him to speak, a longer life would only have enabled him
to create a similar beauty, and not a new beauty,} and so on. Thus
you become an \textit{advocatus diaboli} by setting up the success, the
fact, as your idol: whereas the fact is always dull, at all times
more like calf than a god. Your apologies for history are helped by
ignorance: for it is only because you do not know what a \textit{natura
naturans} like Raphael is, that you are not on fire when you think it
existed once and can never exist again. Some one has lately tried to
tell us that Goethe had out-lived himself with his eighty-two years:
and yet I would gladly take two of Goethe's \enquote{out-lived} years in
exchange for whole cartloads of fresh modern lifetimes, to have
another set of such conversations as those with Eckermann\footnote{Johann Peter Eckermann (1792 –- 1854), German poet and author, known for his work \textit{Conversations with Goethe}}, and be
preserved from all the \enquote{modern} talk of these esquires of the moment.
How few living men have a right to live, as against those mighty
dead! That the many live and those few live no longer, is simply a
brutal truth, that is, a piece of unalterable folly, a blank wall of
\enquote{it was once so} against the moral judgment \enquote{it ought not to have
been.} Yes, against the moral judgment! For you may speak of what
virtue you will, of justice, courage, magnanimity, of wisdom and
human compassion, -- you will find the virtuous man will always rise
against the blind force of facts, the tyranny of the actual, and
submit himself to laws that are not the fickle laws of history. He
ever swims against the waves of history, either by fighting his
passions, as the nearest brute facts of his existence, or by training
himself to honesty amid the glittering nets spun round him by
falsehood. Were history nothing more than the \enquote{all-embracing system
of passion and error,} man would have to read it as Goethe wished
Werther\footnote{\textit{The Sorrows of Young Werther} (Die Leiden des jungen Werthers), or simply Werther, is a 1774 epistolary novel by Johann Wolfgang Goethe} to be read; -- just as if it called to him, \enquote{Be a man and
follow me not!} But fortunately history also keeps alive for us the
memory of the great \enquote{fighters against history,} that is, against the
blind power of the actual; it puts itself in the pillory just by
glorifying the true historical nature in men who troubled themselves
very little about the \enquote{thus it is,} in order that they might follow a
\enquote{thus it must be} with greater joy and greater pride. Not to drag
their generation to the grave, but to found a new one -- that is the
motive that ever drives them onward; and even if they are born late,
there is a way of living by which they can forget it -- and future
generations will know them only as the first-comers.


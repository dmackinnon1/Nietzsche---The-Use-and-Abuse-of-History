\newthought{The fact that life} does need the service of history must be as
clearly grasped as that an excess of history hurts it; this will be
proved later. History is necessary to the living man in three ways:
in relation to his action and struggle, his conservatism and
reverence, his suffering and his desire for deliverance. These three
relations answer to the three kinds of history -- so far as they can be
distinguished -- the \textit{monumental}, the \textit{antiquarian}, and the
\textit{critical}.

History is necessary above all to the man of action and power who
fights a great fight and needs examples, teachers and comforters; he
cannot find them among his contemporaries. It was necessary in this
sense to Schiller\footnote{Johann Christoph Friedrich von Schiller (1759–1805) is considered to be Germany's most important classical playwright.}; for our time is so evil, Goethe says, that the
poet meets no nature that will profit him, among living men. Polybius\footnote{Polybius  was a Greek historian of the middle Hellenistic period. He is noted for his work The Histories, a universal history documenting the rise of Rome in the Mediterranean in the third and second centuries BC.}
is thinking of the active man when he calls political history the
true preparation for governing a state; it is the great teacher, that
shows us how to bear steadfastly the reverses of fortune, by
reminding us of what others have suffered. Whoever has learned to
recognise this meaning in history must hate to see curious tourists
and laborious beetle-hunters climbing up the great pyramids of
antiquity. He does not wish to meet the idler who is rushing through
the picture-galleries of the past for a new distraction or sensation,
where he himself is looking for example and encouragement. To avoid
being troubled by the weak and hopeless idlers, and those whose
apparent activity is merely neurotic, he looks behind him and stays
his course towards the goal in order to breathe. His goal is
happiness, not perhaps his own, but often the nation's, or humanity's
at large: he avoids quietism, and uses history as a weapon against
it. For the most part he has no hope of reward except fame, which
means the expectation of a niche in the temple of history, where he
in his turn may be the consoler and counsellor of posterity. For his
orders are that what has once been able to extend the conception
\enquote{man} and give it a fairer content, must ever exist for the same
office. The great moments in the individual battle form a chain, a
high road for humanity through the ages, and the highest points of
those vanished moments are yet great and living for men; and this is
the fundamental idea of the belief in humanity, that finds a voice in
the demand for a \enquote{monumental} history.

But the fiercest battle is fought round the demand for greatness to
be eternal. Every other living thing cries no. \enquote{Away with the
monuments,} is the watch-word. Dull custom fills all the chambers of
the world with its meanness, and rises in thick vapour round anything
that is great, barring its way to immortality, blinding and stifling
it. And the way passes through mortal brains! Through the brains of
sick and short-lived beasts that ever rise to the surface to breathe,
and painfully keep off annihilation for a little space. For they wish
but one thing: to live at any cost. Who would ever dream of any
\enquote{monumental history} among them, the hard torch-race that alone gives
life to greatness? And yet there are always men awakening, who are
strengthened and made happy by gazing on past greatness, as though
man's life were a lordly thing, and the fairest fruit of this bitter
tree were the knowledge that there was once a man who walked sternly
and proudly through this world, another who had pity and
loving-kindness, another who lived in contemplation, -- but all leaving
one truth behind them, that his life is the fairest who thinks least
about life. The common man snatches greedily at this little span,
with tragic earnestness, but they, on their way to monumental history
and immortality, knew how to greet it with Olympic laughter, or at
least with a lofty scorn; and they went down to their graves in
irony -- for what had they to bury? Only what they had always treated
as dross, refuse, and vanity, and which now falls into its true home
of oblivion, after being so long the sport of their contempt. One
thing will live, the sign-manual of their inmost being, the rare
flash of light, the deed, the creation; because posterity cannot do
without it. In this spiritualised form fame is something more than
the sweetest morsel for our egoism, in Schopenhauer's phrase: it is
the belief in the oneness and continuity of the great in every age,
and a protest against the change and decay of generations.

What is the use to the modern man of this \enquote{monumental} contemplation
of the past, this preoccupation with the rare and classic? It is the
knowledge that the great thing existed and was therefore possible,
and so may be possible again. He is heartened on his way; for his
doubt in weaker moments, whether his desire be not for the
impossible, is struck aside. Suppose one believe that no more than a
hundred men, brought up in the new spirit, efficient and productive,
were needed to give the deathblow to the present fashion of education
in Germany; he will gather strength from the remembrance that the
culture of the Renaissance was raised on the shoulders of such
another band of a hundred men.

And yet if we really wish to learn something from an example, how
vague and elusive do we find the comparison! If it is to give us
strength, many of the differences must be neglected, the
individuality of the past forced into a general formula and all the
sharp angles broken off for the sake of correspondence. Ultimately,
of course, what was once possible can only become possible a second
time on the Pythagorean theory, that when the heavenly bodies are in
the same position again, the events on earth are reproduced to the
smallest detail; so when the stars have a certain relation, a Stoic
and an Epicurean will form a conspiracy to murder Cæsar, and a
different conjunction will show another Columbus discovering America.
Only if the earth always began its drama again after the fifth act,
and it were certain that the same interaction of motives, the same
\textit{deus ex machina}\footnote{\textit{Deus ex machina} -- \enquote{god from the machine} is a plot device whereby a seemingly unsolvable problem in a story is suddenly or abruptly resolved by an unexpected and unlikely occurrence}, the same catastrophe would occur at particular
intervals, could the man of action venture to look for the whole
archetypic truth in monumental history, to see each fact fully set
out in its uniqueness: it would not probably be before the
astronomers became astrologers again. Till then monumental history
will never be able to have complete truth; it will always bring
together things that are incompatible and generalise them into
compatibility, will always weaken the differences of motive and
occasion. Its object is to depict effects at the expense of the
causes -- \enquote{monumentally,} that is, as examples for imitation: it turns
aside, as far as it may, from reasons, and might be called with far
less exaggeration a collection of \enquote{effects in themselves,} than of
events that will have an effect on all ages. The events of war or
religion cherished in our popular celebrations are such \enquote{effects in
themselves}; it is these that will not let ambition sleep, and lie
like amulets on the bolder hearts -- not the real historical nexus of
cause and effect, which, rightly understood, would only prove that
nothing quite similar could ever be cast again from the dice-boxes of
fate and the future.

As long as the soul of history is found in the great impulse that it
gives to a powerful spirit, as long as the past is principally used
as a model for imitation, it is always in danger of being a little
altered and touched up, and brought nearer to fiction. Sometimes
there is no possible distinction between a \enquote{monumental} past and a
mythical romance, as the same motives for action can be gathered from
the one world as the other. If this monumental method of surveying
the past dominate the others, -- the antiquarian and the critical, -- the
past itself suffers wrong. Whole tracts of it are forgotten and
despised; they flow away like a dark unbroken river, with only a few
gaily coloured islands of fact rising above it. There is something
beyond nature in the rare figures that become visible, like the
golden hips that his disciples attributed to Pythagoras. Monumental
history lives by false analogy; it entices the brave to rashness, and
the enthusiastic to fanaticism by its tempting comparisons. Imagine
this history in the hands -- and the head -- of a gifted egoist or an
inspired scoundrel; kingdoms will be overthrown, princes murdered,
war and revolution let loose, and the number of \enquote{effects in
themselves} -- in other words, effects without sufficient
cause -- increased. So much for the harm done by monumental history to
the powerful men of action, be they good or bad; but what if the weak
and the inactive take it as their servant -- or their master!

Consider the simplest and commonest example, the inartistic or half
artistic natures whom a monumental history provides with sword and
buckler. They will use the weapons against their hereditary enemies,
the great artistic spirits, who alone can learn from that history the
one real lesson, how to live, and embody what they have learnt in
noble action. Their way is obstructed, their free air darkened by the
idolatrous -- and conscientious -- dance round the half understood
monument of a great past. \enquote{See, that is the true and real art,} we
seem to hear: \enquote{of what use are these aspiring little people of
to-day?} The dancing crowd has apparently the monopoly of \enquote{good
taste}: for the creator is always at a disadvantage compared with the
mere looker-on, who never put a hand to the work; just as the
arm-chair politician has ever had more wisdom and foresight than the
actual statesman. But if the custom of democratic suffrage and
numerical majorities be transferred to the realm of art, and the
artist put on his defence before the court of æsthetic dilettanti,
you may take your oath on his condemnation; although, or rather
because, his judges had proclaimed solemnly the canon of \enquote{monumental
art,} the art that has \enquote{had an effect on all ages,} according to the
official definition. In their eyes no need nor inclination nor
historical authority is in favour of the art which is not yet
\enquote{monumental} because it is contemporary. Their instinct tells them
that art can be slain by art: the monumental will never be
reproduced, and the weight of its authority is invoked from the past
to make it sure. They are connoisseurs of art, primarily because they
wish to kill art; they pretend to be physicians, when their real idea
is to dabble in poisons. They develop their tastes to a point of
perversion, that they may be able to show a reason for continually
rejecting all the nourishing artistic fare that is offered them. For
they do not want greatness, to arise: their method is to say, \enquote{See,
the great thing is already here!} In reality they care as little
about the great thing that is already here, as that which is about to
arise: their lives are evidence of that. Monumental history is the
cloak under which their hatred of present power and greatness
masquerades as an extreme admiration of the past: the real meaning of
this way of viewing history is disguised as its opposite; whether
they wish it or no, they are acting as though their motto were, \enquote{let
the dead bury the -- living.}\footnote{A variation on \enquote{Let the dead bury the dead} (Luke 9:60), in which Jesus asks the living to follow him, rather than enact their traditional or family obligations.}

Each of the three kinds of history will only flourish in one ground
and climate: otherwise it grows to a noxious weed. If the man who
will produce something great, have need of the past, he makes himself
its master by means of monumental history: the man who can rest
content with the traditional and venerable, uses the past as an
\enquote{antiquarian historian}: and only he whose heart is oppressed by an
instant need, and who will cast the burden off at any price, feels
the want of \enquote{critical history,} the history that judges and condemns.
There is much harm wrought by wrong and thoughtless planting: the
critic without the need, the antiquary without piety, the knower of
the great deed who cannot be the doer of it, are plants that have
grown to weeds, they are torn from their native soil and therefore
degenerate.

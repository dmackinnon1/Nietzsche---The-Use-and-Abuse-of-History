\enquote{I hate everything that merely instructs me without increasing or
directly quickening my activity.} These words of Goethe\footnote{Johann Wolfgang von Goethe(1749– 1832) is widely regarded as the greatest and most influential writer in the German language.}, like a
sincere \textit{ceterum censeo}\footnote{\textit{furthermore, I propose} -- a formulaic expression used to end a speech by reinforcing one, often unrelated, major view. From Cato the Elder's practice of ending every speech, no matter the topic, with \textit{ceterum censeo Carthaginem esse delendam} (\textit{furthermore, I propose that Carthage is to be destroyed}).}, may well stand at the head of my thoughts
on the worth and the worthlessness of history. I will show in them
why instruction that does not \enquote{quicken,} knowledge that slackens the
rein of activity, why in fact history, in Goethe's phrase, must be
seriously \enquote{hated,} as a costly and superfluous luxury of the
understanding: for we are still in want of the necessaries of life,
and the superfluous is an enemy to the necessary. We do need history,
but quite differently from the jaded idlers in the garden of
knowledge, however grandly they may look down on our rude and
unpicturesque requirements. In other words, we need it for life and
action, not as a convenient way to avoid life and action, or to
excuse a selfish life and a cowardly or base action. We would serve
history only so far as it serves life; but to value its study beyond
a certain point mutilates and degrades life: and this is a fact that
certain marked symptoms of our time make it as necessary as it may be
painful to bring to the test of experience.


I have tried to describe a feeling that has often troubled me: I
revenge myself on it by giving it publicity. This may lead some one
to explain to me that he has also had the feeling, but that I do not
feel it purely and elementally enough, and cannot express it with the
ripe certainty of experience. A few may say so; but most people will
tell me that it is a perverted, unnatural, horrible, and altogether
unlawful feeling to have, and that I show myself unworthy of the
great historical movement which is especially strong among the German
people for the last two generations.

I am at all costs going to venture on a description of my feelings;
which will be decidedly in the interests of propriety, as I shall
give plenty of opportunity for paying compliments to such a
\enquote{movement.} And I gain an advantage for myself that is more valuable
to me than propriety -- the attainment of a correct point of view,
through my critics, with regard to our age.

These thoughts are \enquote{out of season,} because I am trying to represent
something of which the age is rightly proud -- its historical
culture -- as a fault and a defect in our time, believing as I do that
we are all suffering from a malignant historical fever and should at
least recognise the fact. But even if it be a virtue, Goethe may be
right in asserting that we cannot help developing our faults at the
same time as our virtues; and an excess of virtue can obviously bring
a nation to ruin, as well as an excess of vice. In any case I may be
allowed my say. But I will first relieve my mind by the confession
that the experiences which produced those disturbing feelings were
mostly drawn from myself, -- and from other sources only for the sake
of comparison; and that I have only reached such \enquote{unseasonable}
experience, so far as I am the nursling of older ages like the Greek,
and less a child of this age. I must admit so much in virtue of my
profession as a classical scholar: for I do not know what meaning
classical scholarship may have for our time except in its being
\enquote{unseasonable,} -- that is, contrary to our time, and yet with an
influence on it for the benefit, it may be hoped, of a future time.

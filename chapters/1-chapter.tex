
\newthought{Consider the herds that are feeding yonder}: they know not the meaning
of yesterday or to-day, they graze and ruminate, move or rest, from
morning to night, from day to day, taken up with their little loves
and hates, at the mercy of the moment, feeling neither melancholy nor
satiety. Man cannot see them without regret, for even in the pride of
his humanity he looks enviously on the beast's happiness. He wishes
simply to live without satiety or pain, like the beast; yet it is all
in vain, for he will not change places with it. He may ask the
beast -- \enquote{Why do you look at me and not speak to me of your happiness?}
The beast wants to answer -- \enquote{Because I always forget what I wished to
say}: but he forgets this answer too, and is silent; and the man is
left to wonder.

He wonders also about himself, that he cannot learn to forget, but
hangs on the past: however far or fast he run, that chain runs with
him. It is matter for wonder: the moment, that is here and gone, that
was nothing before and nothing after, returns like a spectre to
trouble the quiet of a later moment. A leaf is continually dropping
out of the volume of time and fluttering away -- and suddenly it
flutters back into the man's lap. Then he says, \enquote{I remember...,} and
envies the beast, that forgets at once, and sees every moment really
die, sink into night and mist, extinguished for ever. The beast lives
\textit{unhistorically}; for it \enquote{goes into} the present, like a number,
without leaving any curious remainder. It cannot dissimulate, it
conceals nothing; at every moment it seems what it actually is, and
thus can be nothing that is not honest. But man is always resisting
the great and continually increasing weight of the past; it presses
him down, and bows his shoulders; he travels with a dark invisible
burden that he can plausibly disown, and is only too glad to disown
in converse with his fellows -- in order to excite their envy. And so
it hurts him, like the thought of a lost Paradise, to see a herd
grazing, or, nearer still, a child, that has nothing yet of the past
to disown, and plays in a happy blindness between the walls of the
past and the future. And yet its play must be disturbed, and only too
soon will it be summoned from its little kingdom of oblivion. Then it
learns to understand the words \enquote{once upon a time,} the \enquote{open sesame}
that lets in battle, suffering and weariness on mankind, and reminds
them what their existence really is, an imperfect tense that never
becomes a present. And when death brings at last the desired
forgetfulness, it abolishes life and being together, and sets the
seal on the knowledge that \enquote{being} is merely a continual \enquote{has been,}
a thing that lives by denying and destroying and contradicting
itself.

If happiness and the chase for new happiness keep alive in any sense
the will to live, no philosophy has perhaps more truth than the
cynic's: for the beast's happiness, like that of the perfect cynic,
is the visible proof of the truth of cynicism. The smallest pleasure,
if it be only continuous and make one happy, is incomparably a
greater happiness than the more intense pleasure that comes as an
episode, a wild freak, a mad interval between ennui, desire, and
privation. But in the smallest and greatest happiness there is always
one thing that makes it happiness: the power of forgetting, or, in
more learned phrase, the capacity of feeling \enquote{unhistorically}
throughout its duration. One who cannot leave himself behind on the
threshold of the moment and forget the past, who cannot stand on a
single point, like a goddess of victory, without fear or giddiness,
will never know what happiness is; and, worse still, will never do
anything to make others happy. The extreme case would be the man
without any power to forget, who is condemned to see \enquote{becoming}
everywhere. Such a man believes no more in himself or his own
existence, he sees everything fly past in an eternal succession, and
loses himself in the stream of becoming. At last, like the logical
disciple of Heraclitus\footnote{Heraclitus  was a pre-Socratic philosopher. The central ideas of Heraclitus' philosophy are the unity of opposites and the concept of change. He also saw harmony and justice in strife. He viewed the world as constantly in flux, always \enquote{becoming} but never \enquote{being}.}, he will hardly dare to raise his finger.
Forgetfulness is a property of all action; just as not only light but
darkness is bound up with the life of every organism. One who wished
to feel everything historically, would be like a man forcing himself
to refrain from sleep, or a beast who had to live by chewing a
continual cud. Thus even a happy life is possible without
remembrance, as the beast shows: but life in any true sense is
absolutely impossible without forgetfulness. Or, to put my conclusion
better, there is a degree of sleeplessness, of rumination, of
\enquote{historical sense,} that injures and finally destroys the living
thing, be it a man or a people or a system of culture.

To fix this degree and the limits to the memory of the past, if it is
not to become the gravedigger of the present, we must see clearly how
great is the \enquote{plastic power} of a man or a community or a culture; I
mean the power of specifically growing out of one's self, of making
the past and the strange one body with the near and the present, of
healing wounds, replacing what is lost, repairing broken moulds.
There are men who have this power so slightly that a single sharp
experience, a single pain, often a little injustice, will lacerate
their souls like the scratch of a poisoned knife. There are others,
who are so little injured by the worst misfortunes, and even by their
own spiteful actions, as to feel tolerably comfortable, with a fairly
quiet conscience, in the midst of them, -- or at any rate shortly
afterwards. The deeper the roots of a man's inner nature, the better
will he take the past into himself; and the greatest and most
powerful nature would be known by the absence of limits for the
historical sense to overgrow and work harm. It would assimilate and
digest the past, however foreign, and turn it to sap. Such a nature
can forget what it cannot subdue; there is no break in the horizon,
and nothing to remind it that there are still men, passions, theories
and aims on the other side. This is a universal law; a living thing
can only be healthy, strong and productive within a certain horizon:
if it be incapable of drawing one round itself, or too selfish to
lose its own view in another's, it will come to an untimely end.
Cheerfulness, a good conscience, belief in the future, the joyful
deed, all depend, in the individual as well as the nation, on there
being a line that divides the visible and clear from the vague and
shadowy: we must know the right time to forget as well as the right
time to remember; and instinctively see when it is necessary to feel
historically, and when unhistorically. This is the point that the
reader is asked to consider; that the unhistorical and the historical
are equally necessary to the health of an individual, a community,
and a system of culture.

Every one has noticed that a man's historical knowledge and range of
feeling may be very limited, his horizon as narrow as that of an
Alpine valley, his judgments incorrect and his experience falsely
supposed original, and yet in spite of all the incorrectness and
falsity he may stand forth in unconquerable health and vigour, to the
joy of all who see him; whereas another man with far more judgment
and learning will fail in comparison, because the lines of his
horizon are continually changing and shifting, and he cannot shake
himself free from the delicate network of his truth and righteousness
for a downright act of will or desire. We saw that the beast,
absolutely \enquote{unhistorical,} with the narrowest of horizons, has yet a
certain happiness, and lives at least without hypocrisy or ennui; and
so we may hold the capacity of feeling (to a certain extent)
unhistorically, to be the more important and elemental, as providing
the foundation of every sound and real growth, everything that is
truly great and human. The unhistorical is like the surrounding
atmosphere that can alone create life, and in whose annihilation life
itself disappears. It is true that man can only become man by first
suppressing this unhistorical element in his thoughts, comparisons,
distinctions, and conclusions, letting a clear sudden light break
through these misty clouds by his power of turning the past to the
uses of the present. But an excess of history makes him flag again,
while without the veil of the unhistorical he would never have the
courage to begin. What deeds could man ever have done if he had not
been enveloped in the dust-cloud of the unhistorical? Or, to leave
metaphors and take a concrete example, imagine a man swayed and
driven by a strong passion, whether for a woman or a theory. His
world is quite altered. He is blind to everything behind him, new
sounds are muffled and meaningless; though his perceptions were never
so intimately felt in all their colour, light and music, and he Seems
to grasp them with his five senses together. All his judgments of
value are changed for the worse; there is much he can no longer
value, as he can scarcely feel it: he wonders that he has so long
been the sport of strange words and opinions, that his recollections
have run around in one unwearying circle and are yet too weak and
weary to make a single step away from it. His whole case is most
indefensible; it is narrow, ungrateful to the past, blind to danger,
deaf to warnings, a small living eddy in a dead sea of night and
forgetfulness. And yet this condition, unhistorical and
antihistorical throughout, is the cradle not only of unjust action,
but of every just and justifiable action in the world. No artist will
paint his picture, no general win his victory, no nation gain its
freedom, without having striven and yearned for it under those very
\enquote{unhistorical} conditions. If the man of action, in Goethe's phrase,
is without conscience, he is also without knowledge: he forgets most
things in order to do one, he is unjust to what is behind him, and
only recognises one law, the law of that which is to be. So he loves
his work infinitely more than it deserves to be loved; and the best
works are produced in such an ecstasy of love that they must always
be unworthy of it, however great their worth otherwise.

Should any one be able to dissolve the unhistorical atmosphere in
which every great event happens, and breathe afterwards, he might be
capable of rising to the \enquote{super-historical} standpoint of
consciousness, that Niebuhr\footnote{Barthold Georg Niebuhr (1776–1831) was a founder of modern scholarly historiography.} has described as the possible result of
historical research. \enquote{History,} he says, \enquote{is useful for one purpose,
if studied in detail: that men may know, as the greatest and best
spirits of our generation do not know, the accidental nature of the
forms in which they see and insist on others seeing, -- insist, I say,
because their consciousness of them is exceptionally intense. Any one
who has not grasped this idea in its different applications will fall
under the spell of a more powerful spirit who reads a deeper emotion
into the given form.} Such a standpoint might be called
\enquote{super-historical,} as one who took it could feel no impulse from
history to any further life or work, for he would have recognised the
blindness and injustice in the soul of the doer as a condition of
every deed: he would be cured henceforth of taking history too
seriously, and have learnt to answer the question how and why life
should be lived, -- for all men and all circumstances, Greeks or Turks,
the first century or the nineteenth. Whoever asks his friends whether
they would live the last ten or twenty years over again, will easily
see which of them is born for the \enquote{super-historical standpoint}: they
will all answer no, but will give different reasons for their answer.
Some will say they have the consolation that the next twenty will be
better: they are the men referred to satirically by David Hume: -- 

\begin{quote}
  \enquote{And from the dregs of life hope to receive,\\
  What the first sprightly running could not give.}
\end{quote}

\noindent
We will call them the \enquote{historical men.} Their vision of the past
turns them towards the future, encourages them to persevere with
life, and kindles the hope that justice will yet come and happiness
is behind the mountain they are climbing. They believe that the
meaning of existence will become ever clearer in the course of its
evolution, they only look backward at the process to understand the
present and stimulate their longing for the future. They do not know
how unhistorical their thoughts and actions are in spite of all their
history, and how their preoccupation with it is for the sake of life
rather than mere science.

But that question to which we have heard the first answer, is capable
of another; also a \enquote{no,} but on different grounds. It is the \enquote{no} of
the \enquote{super-historical} man who sees no salvation in evolution, for
whom the world is complete and fulfils its aim in every single
moment. How could the next ten years teach what the past ten were not
able to teach?

Whether the aim of the teaching be happiness or resignation, virtue
or penance, these super-historical men are not agreed; but as against
all merely historical ways of viewing the past, they are unanimous in
the theory that the past and the present are one and the same,
typically alike in all their diversity, and forming together a
picture of eternally present imperishable types of unchangeable value
and significance. Just as the hundreds of different languages
correspond to the same constant and elemental needs of mankind, and
one who understood the needs could learn nothing new from the
languages; so the \enquote{super-historical} philosopher sees all the history
of nations and individuals from within. He has a divine insight into
the original meaning of the hieroglyphs, and comes even to be weary
of the letters that are continually unrolled before him. How should
the endless rush of events not bring satiety, surfeit, loathing? So
the boldest of us is ready perhaps at last to say from his heart with
Giacomo Leopardi\footnote{Count Giacomo Taldegardo Francesco di Sales Saverio Pietro Leopardi (1798 -- 1837) was an Italian philosopher, and is considered the greatest Italian poet of the nineteenth century.}: 
\begin{quote}
\enquote{Nothing lives that were worth thy pains, and the
earth deserves not a sigh. Our being is pain and weariness, and the
world is mud -- nothing else. Be calm.}
\end{quote}

\noindent
But we will leave the super-historical men to their loathings and
their wisdom: we wish rather to-day to be joyful in our unwisdom and
have a pleasant life as active men who go forward, and respect the
course of the world. The value we put on the historical may be merely
a Western prejudice: let us at least go forward within this prejudice
and not stand still. If we could only learn better to study history
as a means to life! We would gladly grant the super-historical people
their superior wisdom, so long as we are sure of having more life
than they: for in that case our unwisdom would have a greater future
before it than their wisdom. To make my opposition between life and
wisdom clear, I will take the usual road of the short summary.

A historical phenomenon, completely understood and reduced to an item
of knowledge, is, in relation to the man who knows it, dead: for he
has found out its madness, its injustice, its blind passion, and
especially the earthly and darkened horizon that was the source of
its power for history. This power has now become, for him who has
recognised it, powerless; not yet, perhaps, for him who is alive.

History regarded as pure knowledge and allowed to sway the intellect
would mean for men the final balancing of the ledger of life.
Historical study is only fruitful for the future if it follow a
powerful life-giving influence, for example, a new system of culture;
only, therefore, if it be guided and dominated by a higher force, and
do not itself guide and dominate.

History, so far as it serves life, serves an unhistorical power, and
thus will never become a pure science like mathematics. The question
how far life needs such a service is one of the most serious
questions affecting the well-being of a man, a people and a culture.
For by excess of history life becomes maimed and degenerate, and is
followed by the degeneration of history as well.

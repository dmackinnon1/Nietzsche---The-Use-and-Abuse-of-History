https://www.gutenberg.org/cache/epub/38226/pg38226.txt
The Project Gutenberg eBook of Thoughts Out of Season, Part II
    
This ebook is for the use of anyone anywhere in the United States and
most other parts of the world at no cost and with almost no restrictions
whatsoever. You may copy it, give it away or re-use it under the terms
of the Project Gutenberg License included with this ebook or online
at www.gutenberg.org. If you are not located in the United States,
you will have to check the laws of the country where you are located
before using this eBook.

Title: Thoughts Out of Season, Part II

Author: Friedrich Wilhelm Nietzsche

Translator: Adrian Collins

Release date: December 5, 2011 [eBook #38226]
                Most recently updated: April 3, 2024

Language: English

Credits: Produced by Marc D'Hooghe, Charles Franks, Michael Roe and the Online Distributed Proofreading Team at http://www.pgdp.net


*** START OF THE PROJECT GUTENBERG EBOOK THOUGHTS OUT OF SEASON, PART II ***

THOUGHTS

OUT OF SEASON

PART II


THE USE AND ABUSE OF HISTORY_

SCHOPENHAUER AS EDUCATOR_

By

FRIEDRICH NIETZSCHE

TRANSLATED BY

ADRIAN COLLINS, M.A.

[Illustration]


The Complete Works of Friedrich Nietzsche

The First Complete and Authorised English Translation

Edited by Dr Oscar Levy

Volume Five

T.N. FOULIS

13 & 15 FREDERICK STREET

EDINBURGH: AND LONDON

1910



TO L. P.
FROM THE TRANSLATOR.
EN RECONNAISSANCE.




CONTENTS.
INTRODUCTION
THE USE AND ABUSE OF HISTORY
SCHOPENHAUER AS EDUCATOR




INTRODUCTION.


The two essays translated in this volume form the second and third
parts of the _Unzeitgemässe Betrachtungen_. The essay on history was
completed in January, that on Schopenhauer in August, 1874. Both were
written in the few months of feverish activity that Nietzsche could
spare from his duties as Professor of Classical Philology in Bâle.

Nietzsche, who served in an ambulance corps in '71, had seen
something of the Franco-German War, and to him it was the "honest
German bravery" that had won the day. But to the rest of his
countrymen it was a victory for German culture as well; though there
were still a few elegancies, a few refinements of manners, that might
veneer the new culture, and in this regard the conquered might be
allowed the traditional privilege of conquering the conquerors.
Nietzsche answered roundly, "the German does not yet know the meaning
of the word culture," and in the essay on history set himself to show
that the so-called culture was a morass into which the German had
been led by a sixth sense he had developed during the nineteenth
century--the "historical sense": he had been brought by his spiritual
teachers to believe that he was the "crown of the world-process" and
that his highest duty lay in surrendering himself to it.

With Nietzsche, the historical sense became a "malady from which men
suffer," the world-process an illusion, evolutionary theories a
subtle excuse for inactivity. History is for the few not the many,
for the man not the youth, for the great not the small--who are
broken and bewildered by it. It is the lesson of remembrance, and few
are strong enough to bear that lesson. History has no meaning except
as the servant of life and action: and most of us can only act if we
forget. This is the burden of the first essay; and turning from
history to the historian he condemns the "noisy little fellows" who
measure the motives of the great men of the past by their own, and
use the past to justify their present.

But who are the men that can use history rightly, and for whom it is
a help and not a hindrance to life? They are the great men of action
and thought, the "lonely giants amid the pigmies." To them alone can
the record of their great forebears be a consolation as well as a
lesson. In the realm of thought, they are of the type of the ideal
philosopher sketched in the second essay. To Nietzsche the only hope
of the race lies in the "production of the genius," of the man who
can bear the burden of the future and not be swamped by the past: he
found the personal expression of such a man, for the time being, in
Schopenhauer.

Schopenhauer here stands, as a personality, for all that makes for
life in philosophy, against the stagnation of the professional
philosopher. The last part of the essay is a fierce polemic against
state-aided philosophy and the official position of the professors,
who formed, and still form, the intellectual aristocracy of Germany,
with a cathedral authority on all their pronouncements.

But "there has never been a eulogy on a philosopher," says Dr. Kögel,
"that has had so little to say about his philosophy." The essay on
Schopenhauer is of value precisely because it has nothing to do with
Schopenhauer. We need not be disturbed by the thought that Nietzsche
afterwards turned from him. He truly recognised that Schopenhauer was
here merely a name for himself, that "not Schopenhauer as educator is
in question, but his opposite, Nietzsche as educator" (_Ecce Homo_).
He could regard Schopenhauer, later, as a siren that called to death;
he put him among the great artists that lead down--who are worse than
the bad artists that lead nowhere. "We must go further in the
pessimistic logic than the denial of the will," he says in the
_Götzendämmerung_; "we must deny Schopenhauer." The pessimism and
denial of the will, the blank despair before suffering, were the
shoals on which Nietzsche's reverence finally broke. They could not
stand before the Dionysian outlook, whose pessimism sprang not from
weakness but strength, and in which the joy of willing and being can
even welcome suffering. In this essay we hear little of the
pessimism, save as the imperfect and "all-too-human" side of
Schopenhauer, that actually brings us nearer to him. Later, he could
part the man and his work, and speak of Schopenhauer's view as the
"Evil eye." But as yet he is a young man who has kept his illusions,
and, like Ogniben, he judges men by what they might be.

Afterwards, he judged himself too in these essays by "what he might
be." "To me," he said in _Ecce Homo_, "they are promises: I know not
what they mean to others."

It is also in the belief they are promises that they are here
translated "for others." The _Thoughts out of Season_ are the first
announcement of the complex theme of the _Zarathustra_. They form the
best possible introduction to Nietzschean thought. Nietzsche is
already the knight-errant of philosophy: but his adventure is just
beginning.

A. C.




THE USE AND ABUSE OF HISTORY.


PREFACE.


"I hate everything that merely instructs me without increasing or
directly quickening my activity." These words of Goethe, like a
sincere _ceterum censeo_, may well stand at the head of my thoughts
on the worth and the worthlessness of history. I will show in them
why instruction that does not "quicken," knowledge that slackens the
rein of activity, why in fact history, in Goethe's phrase, must be
seriously "hated," as a costly and superfluous luxury of the
understanding: for we are still in want of the necessaries of life,
and the superfluous is an enemy to the necessary. We do need history,
but quite differently from the jaded idlers in the garden of
knowledge, however grandly they may look down on our rude and
unpicturesque requirements. In other words, we need it for life and
action, not as a convenient way to avoid life and action, or to
excuse a selfish life and a cowardly or base action. We would serve
history only so far as it serves life; but to value its study beyond
a certain point mutilates and degrades life: and this is a fact that
certain marked symptoms of our time make it as necessary as it may be
painful to bring to the test of experience.

I have tried to describe a feeling that has often troubled me: I
revenge myself on it by giving it publicity. This may lead some one
to explain to me that he has also had the feeling, but that I do not
feel it purely and elementally enough, and cannot express it with the
ripe certainty of experience. A few may say so; but most people will
tell me that it is a perverted, unnatural, horrible, and altogether
unlawful feeling to have, and that I show myself unworthy of the
great historical movement which is especially strong among the German
people for the last two generations.

I am at all costs going to venture on a description of my feelings;
which will be decidedly in the interests of propriety, as I shall
give plenty of opportunity for paying compliments to such a
"movement." And I gain an advantage for myself that is more valuable
to me than propriety--the attainment of a correct point of view,
through my critics, with regard to our age.

These thoughts are "out of season," because I am trying to represent
something of which the age is rightly proud--its historical
culture--as a fault and a defect in our time, believing as I do that
we are all suffering from a malignant historical fever and should at
least recognise the fact. But even if it be a virtue, Goethe may be
right in asserting that we cannot help developing our faults at the
same time as our virtues; and an excess of virtue can obviously bring
a nation to ruin, as well as an excess of vice. In any case I may be
allowed my say. But I will first relieve my mind by the confession
that the experiences which produced those disturbing feelings were
mostly drawn from myself,--and from other sources only for the sake
of comparison; and that I have only reached such "unseasonable"
experience, so far as I am the nursling of older ages like the Greek,
and less a child of this age. I must admit so much in virtue of my
profession as a classical scholar: for I do not know what meaning
classical scholarship may have for our time except in its being
"unseasonable,"--that is, contrary to our time, and yet with an
influence on it for the benefit, it may be hoped, of a future time.


I.

Consider the herds that are feeding yonder: they know not the meaning
of yesterday or to-day, they graze and ruminate, move or rest, from
morning to night, from day to day, taken up with their little loves
and hates, at the mercy of the moment, feeling neither melancholy nor
satiety. Man cannot see them without regret, for even in the pride of
his humanity he looks enviously on the beast's happiness. He wishes
simply to live without satiety or pain, like the beast; yet it is all
in vain, for he will not change places with it. He may ask the
beast--"Why do you look at me and not speak to me of your happiness?"
The beast wants to answer--"Because I always forget what I wished to
say": but he forgets this answer too, and is silent; and the man is
left to wonder.

He wonders also about himself, that he cannot learn to forget, but
hangs on the past: however far or fast he run, that chain runs with
him. It is matter for wonder: the moment, that is here and gone, that
was nothing before and nothing after, returns like a spectre to
trouble the quiet of a later moment. A leaf is continually dropping
out of the volume of time and fluttering away--and suddenly it
flutters back into the man's lap. Then he says, "I remember...," and
envies the beast, that forgets at once, and sees every moment really
die, sink into night and mist, extinguished for ever. The beast lives
_unhistorically_; for it "goes into" the present, like a number,
without leaving any curious remainder. It cannot dissimulate, it
conceals nothing; at every moment it seems what it actually is, and
thus can be nothing that is not honest. But man is always resisting
the great and continually increasing weight of the past; it presses
him down, and bows his shoulders; he travels with a dark invisible
burden that he can plausibly disown, and is only too glad to disown
in converse with his fellows--in order to excite their envy. And so
it hurts him, like the thought of a lost Paradise, to see a herd
grazing, or, nearer still, a child, that has nothing yet of the past
to disown, and plays in a happy blindness between the walls of the
past and the future. And yet its play must be disturbed, and only too
soon will it be summoned from its little kingdom of oblivion. Then it
learns to understand the words "once upon a time," the "open sesame"
that lets in battle, suffering and weariness on mankind, and reminds
them what their existence really is, an imperfect tense that never
becomes a present. And when death brings at last the desired
forgetfulness, it abolishes life and being together, and sets the
seal on the knowledge that "being" is merely a continual "has been,"
a thing that lives by denying and destroying and contradicting
itself.

If happiness and the chase for new happiness keep alive in any sense
the will to live, no philosophy has perhaps more truth than the
cynic's: for the beast's happiness, like that of the perfect cynic,
is the visible proof of the truth of cynicism. The smallest pleasure,
if it be only continuous and make one happy, is incomparably a
greater happiness than the more intense pleasure that comes as an
episode, a wild freak, a mad interval between ennui, desire, and
privation. But in the smallest and greatest happiness there is always
one thing that makes it happiness: the power of forgetting, or, in
more learned phrase, the capacity of feeling "unhistorically"
throughout its duration. One who cannot leave himself behind on the
threshold of the moment and forget the past, who cannot stand on a
single point, like a goddess of victory, without fear or giddiness,
will never know what happiness is; and, worse still, will never do
anything to make others happy. The extreme case would be the man
without any power to forget, who is condemned to see "becoming"
everywhere. Such a man believes no more in himself or his own
existence, he sees everything fly past in an eternal succession, and
loses himself in the stream of becoming. At last, like the logical
disciple of Heraclitus, he will hardly dare to raise his finger.
Forgetfulness is a property of all action; just as not only light but
darkness is bound up with the life of every organism. One who wished
to feel everything historically, would be like a man forcing himself
to refrain from sleep, or a beast who had to live by chewing a
continual cud. Thus even a happy life is possible without
remembrance, as the beast shows: but life in any true sense is
absolutely impossible without forgetfulness. Or, to put my conclusion
better, there is a degree of sleeplessness, of rumination, of
"historical sense," that injures and finally destroys the living
thing, be it a man or a people or a system of culture.

To fix this degree and the limits to the memory of the past, if it is
not to become the gravedigger of the present, we must see clearly how
great is the "plastic power" of a man or a community or a culture; I
mean the power of specifically growing out of one's self, of making
the past and the strange one body with the near and the present, of
healing wounds, replacing what is lost, repairing broken moulds.
There are men who have this power so slightly that a single sharp
experience, a single pain, often a little injustice, will lacerate
their souls like the scratch of a poisoned knife. There are others,
who are so little injured by the worst misfortunes, and even by their
own spiteful actions, as to feel tolerably comfortable, with a fairly
quiet conscience, in the midst of them,--or at any rate shortly
afterwards. The deeper the roots of a man's inner nature, the better
will he take the past into himself; and the greatest and most
powerful nature would be known by the absence of limits for the
historical sense to overgrow and work harm. It would assimilate and
digest the past, however foreign, and turn it to sap. Such a nature
can forget what it cannot subdue; there is no break in the horizon,
and nothing to remind it that there are still men, passions, theories
and aims on the other side. This is a universal law; a living thing
can only be healthy, strong and productive within a certain horizon:
if it be incapable of drawing one round itself, or too selfish to
lose its own view in another's, it will come to an untimely end.
Cheerfulness, a good conscience, belief in the future, the joyful
deed, all depend, in the individual as well as the nation, on there
being a line that divides the visible and clear from the vague and
shadowy: we must know the right time to forget as well as the right
time to remember; and instinctively see when it is necessary to feel
historically, and when unhistorically. This is the point that the
reader is asked to consider; that the unhistorical and the historical
are equally necessary to the health of an individual, a community,
and a system of culture.

Every one has noticed that a man's historical knowledge and range of
feeling may be very limited, his horizon as narrow as that of an
Alpine valley, his judgments incorrect and his experience falsely
supposed original, and yet in spite of all the incorrectness and
falsity he may stand forth in unconquerable health and vigour, to the
joy of all who see him; whereas another man with far more judgment
and learning will fail in comparison, because the lines of his
horizon are continually changing and shifting, and he cannot shake
himself free from the delicate network of his truth and righteousness
for a downright act of will or desire. We saw that the beast,
absolutely "unhistorical," with the narrowest of horizons, has yet a
certain happiness, and lives at least without hypocrisy or ennui; and
so we may hold the capacity of feeling (to a certain extent)
unhistorically, to be the more important and elemental, as providing
the foundation of every sound and real growth, everything that is
truly great and human. The unhistorical is like the surrounding
atmosphere that can alone create life, and in whose annihilation life
itself disappears. It is true that man can only become man by first
suppressing this unhistorical element in his thoughts, comparisons,
distinctions, and conclusions, letting a clear sudden light break
through these misty clouds by his power of turning the past to the
uses of the present. But an excess of history makes him flag again,
while without the veil of the unhistorical he would never have the
courage to begin. What deeds could man ever have done if he had not
been enveloped in the dust-cloud of the unhistorical? Or, to leave
metaphors and take a concrete example, imagine a man swayed and
driven by a strong passion, whether for a woman or a theory. His
world is quite altered. He is blind to everything behind him, new
sounds are muffled and meaningless; though his perceptions were never
so intimately felt in all their colour, light and music, and he Seems
to grasp them with his five senses together. All his judgments of
value are changed for the worse; there is much he can no longer
value, as he can scarcely feel it: he wonders that he has so long
been the sport of strange words and opinions, that his recollections
have run around in one unwearying circle and are yet too weak and
weary to make a single step away from it. His whole case is most
indefensible; it is narrow, ungrateful to the past, blind to danger,
deaf to warnings, a small living eddy in a dead sea of night and
forgetfulness. And yet this condition, unhistorical and
antihistorical throughout, is the cradle not only of unjust action,
but of every just and justifiable action in the world. No artist will
paint his picture, no general win his victory, no nation gain its
freedom, without having striven and yearned for it under those very
"unhistorical" conditions. If the man of action, in Goethe's phrase,
is without conscience, he is also without knowledge: he forgets most
things in order to do one, he is unjust to what is behind him, and
only recognises one law, the law of that which is to be. So he loves
his work infinitely more than it deserves to be loved; and the best
works are produced in such an ecstasy of love that they must always
be unworthy of it, however great their worth otherwise.

Should any one be able to dissolve the unhistorical atmosphere in
which every great event happens, and breathe afterwards, he might be
capable of rising to the "super-historical" standpoint of
consciousness, that Niebuhr has described as the possible result of
historical research. "History," he says, "is useful for one purpose,
if studied in detail: that men may know, as the greatest and best
spirits of our generation do not know, the accidental nature of the
forms in which they see and insist on others seeing,--insist, I say,
because their consciousness of them is exceptionally intense. Any one
who has not grasped this idea in its different applications will fall
under the spell of a more powerful spirit who reads a deeper emotion
into the given form." Such a standpoint might be called
"super-historical," as one who took it could feel no impulse from
history to any further life or work, for he would have recognised the
blindness and injustice in the soul of the doer as a condition of
every deed: he would be cured henceforth of taking history too
seriously, and have learnt to answer the question how and why life
should be lived,--for all men and all circumstances, Greeks or Turks,
the first century or the nineteenth. Whoever asks his friends whether
they would live the last ten or twenty years over again, will easily
see which of them is born for the "super-historical standpoint": they
will all answer no, but will give different reasons for their answer.
Some will say they have the consolation that the next twenty will be
better: they are the men referred to satirically by David Hume:--

  "And from the dregs of life hope to receive,
  What the first sprightly running could not give."

We will call them the "historical men." Their vision of the past
turns them towards the future, encourages them to persevere with
life, and kindles the hope that justice will yet come and happiness
is behind the mountain they are climbing. They believe that the
meaning of existence will become ever clearer in the course of its
evolution, they only look backward at the process to understand the
present and stimulate their longing for the future. They do not know
how unhistorical their thoughts and actions are in spite of all their
history, and how their preoccupation with it is for the sake of life
rather than mere science.

But that question to which we have heard the first answer, is capable
of another; also a "no," but on different grounds. It is the "no" of
the "super-historical" man who sees no salvation in evolution, for
whom the world is complete and fulfils its aim in every single
moment. How could the next ten years teach what the past ten were not
able to teach?

Whether the aim of the teaching be happiness or resignation, virtue
or penance, these super-historical men are not agreed; but as against
all merely historical ways of viewing the past, they are unanimous in
the theory that the past and the present are one and the same,
typically alike in all their diversity, and forming together a
picture of eternally present imperishable types of unchangeable value
and significance. Just as the hundreds of different languages
correspond to the same constant and elemental needs of mankind, and
one who understood the needs could learn nothing new from the
languages; so the "super-historical" philosopher sees all the history
of nations and individuals from within. He has a divine insight into
the original meaning of the hieroglyphs, and comes even to be weary
of the letters that are continually unrolled before him. How should
the endless rush of events not bring satiety, surfeit, loathing? So
the boldest of us is ready perhaps at last to say from his heart with
Giacomo Leopardi: "Nothing lives that were worth thy pains, and the
earth deserves not a sigh. Our being is pain and weariness, and the
world is mud--nothing else. Be calm."

But we will leave the super-historical men to their loathings and
their wisdom: we wish rather to-day to be joyful in our unwisdom and
have a pleasant life as active men who go forward, and respect the
course of the world. The value we put on the historical may be merely
a Western prejudice: let us at least go forward within this prejudice
and not stand still. If we could only learn better to study history
as a means to life! We would gladly grant the super-historical people
their superior wisdom, so long as we are sure of having more life
than they: for in that case our unwisdom would have a greater future
before it than their wisdom. To make my opposition between life and
wisdom clear, I will take the usual road of the short summary.

A historical phenomenon, completely understood and reduced to an item
of knowledge, is, in relation to the man who knows it, dead: for he
has found out its madness, its injustice, its blind passion, and
especially the earthly and darkened horizon that was the source of
its power for history. This power has now become, for him who has
recognised it, powerless; not yet, perhaps, for him who is alive.

History regarded as pure knowledge and allowed to sway the intellect
would mean for men the final balancing of the ledger of life.
Historical study is only fruitful for the future if it follow a
powerful life-giving influence, for example, a new system of culture;
only, therefore, if it be guided and dominated by a higher force, and
do not itself guide and dominate.

History, so far as it serves life, serves an unhistorical power, and
thus will never become a pure science like mathematics. The question
how far life needs such a service is one of the most serious
questions affecting the well-being of a man, a people and a culture.
For by excess of history life becomes maimed and degenerate, and is
followed by the degeneration of history as well.


II.

The fact that life does need the service of history must be as
clearly grasped as that an excess of history hurts it; this will be
proved later. History is necessary to the living man in three ways:
in relation to his action and struggle, his conservatism and
reverence, his suffering and his desire for deliverance. These three
relations answer to the three kinds of history--so far as they can be
distinguished--the _monumental_, the _antiquarian_, and the
_critical_.

History is necessary above all to the man of action and power who
fights a great fight and needs examples, teachers and comforters; he
cannot find them among his contemporaries. It was necessary in this
sense to Schiller; for our time is so evil, Goethe says, that the
poet meets no nature that will profit him, among living men. Polybius
is thinking of the active man when he calls political history the
true preparation for governing a state; it is the great teacher, that
shows us how to bear steadfastly the reverses of fortune, by
reminding us of what others have suffered. Whoever has learned to
recognise this meaning in history must hate to see curious tourists
and laborious beetle-hunters climbing up the great pyramids of
antiquity. He does not wish to meet the idler who is rushing through
the picture-galleries of the past for a new distraction or sensation,
where he himself is looking for example and encouragement. To avoid
being troubled by the weak and hopeless idlers, and those whose
apparent activity is merely neurotic, he looks behind him and stays
his course towards the goal in order to breathe. His goal is
happiness, not perhaps his own, but often the nation's, or humanity's
at large: he avoids quietism, and uses history as a weapon against
it. For the most part he has no hope of reward except fame, which
means the expectation of a niche in the temple of history, where he
in his turn may be the consoler and counsellor of posterity. For his
orders are that what has once been able to extend the conception
"man" and give it a fairer content, must ever exist for the same
office. The great moments in the individual battle form a chain, a
high road for humanity through the ages, and the highest points of
those vanished moments are yet great and living for men; and this is
the fundamental idea of the belief in humanity, that finds a voice in
the demand for a "monumental" history.

But the fiercest battle is fought round the demand for greatness to
be eternal. Every other living thing cries no. "Away with the
monuments," is the watch-word. Dull custom fills all the chambers of
the world with its meanness, and rises in thick vapour round anything
that is great, barring its way to immortality, blinding and stifling
it. And the way passes through mortal brains! Through the brains of
sick and short-lived beasts that ever rise to the surface to breathe,
and painfully keep off annihilation for a little space. For they wish
but one thing: to live at any cost. Who would ever dream of any
"monumental history" among them, the hard torch-race that alone gives
life to greatness? And yet there are always men awakening, who are
strengthened and made happy by gazing on past greatness, as though
man's life were a lordly thing, and the fairest fruit of this bitter
tree were the knowledge that there was once a man who walked sternly
and proudly through this world, another who had pity and
loving-kindness, another who lived in contemplation,--but all leaving
one truth behind them, that his life is the fairest who thinks least
about life. The common man snatches greedily at this little span,
with tragic earnestness, but they, on their way to monumental history
and immortality, knew how to greet it with Olympic laughter, or at
least with a lofty scorn; and they went down to their graves in
irony--for what had they to bury? Only what they had always treated
as dross, refuse, and vanity, and which now falls into its true home
of oblivion, after being so long the sport of their contempt. One
thing will live, the sign-manual of their inmost being, the rare
flash of light, the deed, the creation; because posterity cannot do
without it. In this spiritualised form fame is something more than
the sweetest morsel for our egoism, in Schopenhauer's phrase: it is
the belief in the oneness and continuity of the great in every age,
and a protest against the change and decay of generations.

What is the use to the modern man of this "monumental" contemplation
of the past, this preoccupation with the rare and classic? It is the
knowledge that the great thing existed and was therefore possible,
and so may be possible again. He is heartened on his way; for his
doubt in weaker moments, whether his desire be not for the
impossible, is struck aside. Suppose one believe that no more than a
hundred men, brought up in the new spirit, efficient and productive,
were needed to give the deathblow to the present fashion of education
in Germany; he will gather strength from the remembrance that the
culture of the Renaissance was raised on the shoulders of such
another band of a hundred men.

And yet if we really wish to learn something from an example, how
vague and elusive do we find the comparison! If it is to give us
strength, many of the differences must be neglected, the
individuality of the past forced into a general formula and all the
sharp angles broken off for the sake of correspondence. Ultimately,
of course, what was once possible can only become possible a second
time on the Pythagorean theory, that when the heavenly bodies are in
the same position again, the events on earth are reproduced to the
smallest detail; so when the stars have a certain relation, a Stoic
and an Epicurean will form a conspiracy to murder Cæsar, and a
different conjunction will show another Columbus discovering America.
Only if the earth always began its drama again after the fifth act,
and it were certain that the same interaction of motives, the same
_deus ex machina_, the same catastrophe would occur at particular
intervals, could the man of action venture to look for the whole
archetypic truth in monumental history, to see each fact fully set
out in its uniqueness: it would not probably be before the
astronomers became astrologers again. Till then monumental history
will never be able to have complete truth; it will always bring
together things that are incompatible and generalise them into
compatibility, will always weaken the differences of motive and
occasion. Its object is to depict effects at the expense of the
causes--"monumentally," that is, as examples for imitation: it turns
aside, as far as it may, from reasons, and might be called with far
less exaggeration a collection of "effects in themselves," than of
events that will have an effect on all ages. The events of war or
religion cherished in our popular celebrations are such "effects in
themselves"; it is these that will not let ambition sleep, and lie
like amulets on the bolder hearts--not the real historical nexus of
cause and effect, which, rightly understood, would only prove that
nothing quite similar could ever be cast again from the dice-boxes of
fate and the future.

As long as the soul of history is found in the great impulse that it
gives to a powerful spirit, as long as the past is principally used
as a model for imitation, it is always in danger of being a little
altered and touched up, and brought nearer to fiction. Sometimes
there is no possible distinction between a "monumental" past and a
mythical romance, as the same motives for action can be gathered from
the one world as the other. If this monumental method of surveying
the past dominate the others,--the antiquarian and the critical,--the
past itself suffers wrong. Whole tracts of it are forgotten and
despised; they flow away like a dark unbroken river, with only a few
gaily coloured islands of fact rising above it. There is something
beyond nature in the rare figures that become visible, like the
golden hips that his disciples attributed to Pythagoras. Monumental
history lives by false analogy; it entices the brave to rashness, and
the enthusiastic to fanaticism by its tempting comparisons. Imagine
this history in the hands--and the head--of a gifted egoist or an
inspired scoundrel; kingdoms will be overthrown, princes murdered,
war and revolution let loose, and the number of "effects in
themselves"--in other words, effects without sufficient
cause--increased. So much for the harm done by monumental history to
the powerful men of action, be they good or bad; but what if the weak
and the inactive take it as their servant--or their master!

Consider the simplest and commonest example, the inartistic or half
artistic natures whom a monumental history provides with sword and
buckler. They will use the weapons against their hereditary enemies,
the great artistic spirits, who alone can learn from that history the
one real lesson, how to live, and embody what they have learnt in
noble action. Their way is obstructed, their free air darkened by the
idolatrous--and conscientious--dance round the half understood
monument of a great past. "See, that is the true and real art," we
seem to hear: "of what use are these aspiring little people of
to-day?" The dancing crowd has apparently the monopoly of "good
taste": for the creator is always at a disadvantage compared with the
mere looker-on, who never put a hand to the work; just as the
arm-chair politician has ever had more wisdom and foresight than the
actual statesman. But if the custom of democratic suffrage and
numerical majorities be transferred to the realm of art, and the
artist put on his defence before the court of æsthetic dilettanti,
you may take your oath on his condemnation; although, or rather
because, his judges had proclaimed solemnly the canon of "monumental
art," the art that has "had an effect on all ages," according to the
official definition. In their eyes no need nor inclination nor
historical authority is in favour of the art which is not yet
"monumental" because it is contemporary. Their instinct tells them
that art can be slain by art: the monumental will never be
reproduced, and the weight of its authority is invoked from the past
to make it sure. They are connoisseurs of art, primarily because they
wish to kill art; they pretend to be physicians, when their real idea
is to dabble in poisons. They develop their tastes to a point of
perversion, that they may be able to show a reason for continually
rejecting all the nourishing artistic fare that is offered them. For
they do not want greatness, to arise: their method is to say, "See,
the great thing is already here!" In reality they care as little
about the great thing that is already here, as that which is about to
arise: their lives are evidence of that. Monumental history is the
cloak under which their hatred of present power and greatness
masquerades as an extreme admiration of the past: the real meaning of
this way of viewing history is disguised as its opposite; whether
they wish it or no, they are acting as though their motto were, "let
the dead bury the--living."

Each of the three kinds of history will only flourish in one ground
and climate: otherwise it grows to a noxious weed. If the man who
will produce something great, have need of the past, he makes himself
its master by means of monumental history: the man who can rest
content with the traditional and venerable, uses the past as an
"antiquarian historian": and only he whose heart is oppressed by an
instant need, and who will cast the burden off at any price, feels
the want of "critical history," the history that judges and condemns.
There is much harm wrought by wrong and thoughtless planting: the
critic without the need, the antiquary without piety, the knower of
the great deed who cannot be the doer of it, are plants that have
grown to weeds, they are torn from their native soil and therefore
degenerate.


III.

Secondly, history is necessary to the man of conservative and
reverent nature, who looks back to the origins of his existence with
love and trust; through it, he gives thanks for life. He is careful
to preserve what survives from ancient days, and will reproduce the
conditions of his own upbringing for those who come after him; thus
he does life a service. The possession of his ancestors' furniture
changes its meaning in his soul: for his soul is rather possessed by
it. All that is small and limited, mouldy and obsolete, gains a worth
and inviolability of its own from the conservative and reverent soul
of the antiquary migrating into it, and building a secret nest there.
The history of his town becomes the history of himself; he looks on
the walls, the turreted gate, the town council, the fair, as an
illustrated diary of his youth, and sees himself in it all--his
strength, industry, desire, reason, faults and follies. "Here one
could live," he says, "as one can live here now--and will go on
living; for we are tough folk, and will not be uprooted in the
night." And so, with his "we," he surveys the marvellous individual
life of the past and identifies himself with the spirit of the house,
the family and the city. He greets the soul of his people from afar
as his own, across the dim and troubled centuries: his gifts and his
virtues lie in such power of feeling and divination, his scent of a
half-vanished trail, his instinctive correctness in reading the
scribbled past, and understanding at once its palimpsests--nay, its
polypsests. Goethe stood with such thoughts before the monument of
Erwin von Steinbach: the storm of his feeling rent the historical
cloud-veil that hung between them, and he saw the German work for the
first time "coming from the stern, rough, German soul." This was the
road that the Italians of the Renaissance travelled, the spirit that
reawakened the ancient Italic genius in their poets to "a wondrous
echo of the immemorial lyre," as Jacob Burckhardt says. But the
greatest value of this antiquarian spirit of reverence lies in the
simple emotions of pleasure and content that it lends to the drab,
rough, even painful circumstances of a nation's or individual's life:
Niebuhr confesses that he could live happily on a moor among free
peasants with a history, and would never feel the want of art. How
could history serve life better than by anchoring the less gifted
races and peoples to the homes and customs of their ancestors, and
keeping them from ranging far afield in search of better, to find
only struggle and competition? The influence that ties men down to
the same companions and circumstances, to the daily round of toil, to
their bare mountain-side,--seems to be selfish and unreasonable: but
it is a healthy unreason and of profit to the community; as every one
knows who has clearly realised the terrible consequences of mere
desire for migration and adventure,--perhaps in whole peoples,--or
who watches the destiny of a nation that has lost confidence in its
earlier days, and is given up to a restless cosmopolitanism and an
unceasing desire for novelty. The feeling of the tree that clings to
its roots, the happiness of knowing one's growth to be one not merely
arbitrary and fortuitous, but the inheritance, the fruit and blossom
of a past, that does not merely justify but crown the present--this
is what we nowadays prefer to call the real historical sense.

These are not the conditions most favourable to reducing the past to
pure science: and we see here too, as we saw in the case of
monumental history, that the past itself suffers when history serves
life and is directed by its end. To vary the metaphor, the tree feels
its roots better than it can see them: the greatness of the feeling
is measured by the greatness and strength of the visible branches.
The tree may be wrong here; how far more wrong will it be in regard
to the whole forest, which it only knows and feels so far as it is
hindered or helped by it, and not otherwise! The antiquarian sense of
a man, a city or a nation has always a very limited field. Many
things are not noticed at all; the others are seen in isolation, as
through a microscope. There is no measure: equal importance is given
to everything, and therefore too much to anything. For the things of
the past are never viewed in their true perspective or receive their
just value; but value and perspective change with the individual or
the nation that is looking back on its past.

There is always the danger here, that everything ancient will be
regarded as equally venerable, and everything without this respect
for antiquity, like a new spirit, rejected as an enemy. The Greeks
themselves admitted the archaic style of plastic art by the side of
the freer and greater style; and later, did not merely tolerate the
pointed nose and the cold mouth, but made them even a canon of taste.
If the judgment of a people harden in this way, and history's service
to the past life be to undermine a further and higher life; if the
historical sense no longer preserve life, but mummify it: then the
tree dies, unnaturally, from the top downwards, and at last the roots
themselves wither. Antiquarian history degenerates from the moment
that it no longer gives a soul and inspiration to the fresh life of
the present. The spring of piety is dried up, but the learned habit
persists without it and revolves complaisantly round its own centre.
The horrid spectacle is seen of the mad collector raking over all the
dust-heaps of the past. He breathes a mouldy air; the antiquarian
habit may degrade a considerable talent, a real spiritual need in
him, to a mere insatiable curiosity for everything old: he often
sinks so low as to be satisfied with any food, and greedily devour
all the scraps that fall from the bibliographical table.

Even if this degeneration do not take place, and the foundation be
not withered on which antiquarian history can alone take root with
profit to life: yet there are dangers enough, if it become too
powerful and invade the territories of the other methods. It only
understands how to preserve life, not to create it; and thus always
undervalues the present growth, having, unlike monumental history, no
certain instinct for it. Thus it hinders the mighty impulse to a new
deed and paralyses the doer, who must always, as doer, be grazing
some piety or other. The fact that has grown old carries with it a
demand for its own immortality. For when one considers the
life-history of such an ancient fact, the amount of reverence paid to
it for generations--whether it be a custom, a religious creed, or a
political principle,--it seems presumptuous, even impious, to replace
it by a new fact, and the ancient congregation of pieties by a new
piety.

Here we see clearly how necessary a third way of looking at the past
is to man, beside the other two. This is the "critical" way; which is
also in the service of life. Man must have the strength to break up
the past; and apply it too, in order to live. He must bring the past
to the bar of judgment, interrogate it remorselessly, and finally
condemn it. Every past is worth condemning: this is the rule in
mortal affairs, which always contain a large measure of human power
and human weakness. It is not justice that sits in judgment here; nor
mercy that proclaims the verdict; but only life, the dim, driving
force that insatiably desires--itself. Its sentence is always
unmerciful, always unjust, as it never flows from a pure fountain of
knowledge: though it would generally turn out the same, if Justice
herself delivered it. "For everything that is born is _worthy_ of
being destroyed: better were it then that nothing should be born." It
requires great strength to be able to live and forget how far life
and injustice are one. Luther himself once said that the world only
arose by an oversight of God; if he had ever dreamed of heavy
ordnance, he would never have created it. The same life that needs
forgetfulness, needs sometimes its destruction; for should the
injustice of something ever become obvious--a monopoly, a caste, a
dynasty for example--the thing deserves to fall. Its past is
critically examined, the knife put to its roots, and all the
"pieties" are grimly trodden under foot. The process is always
dangerous, even for life; and the men or the times that serve life in
this way, by judging and annihilating the past, are always dangerous
to themselves and others. For as we are merely the resultant of
previous generations, we are also the resultant of their errors,
passions, and crimes: it is impossible to shake off this chain.
Though we condemn the errors and think we have escaped them, we
cannot escape the fact that we spring from them. At best, it comes to
a conflict between our innate, inherited nature and our knowledge,
between a stern, new discipline and an ancient tradition; and we
plant a new way of life, a new instinct, a second nature, that
withers the first. It is an attempt to gain a past _a posteriori_
from which we might spring, as against that from which we do spring;
always a dangerous attempt, as it is difficult to find a limit to the
denial of the past, and the second natures are generally weaker than
the first. We stop too often at knowing the good without doing it,
because we also know the better but cannot do it. Here and there the
victory is won, which gives a strange consolation to the fighters, to
those who use critical history for the sake of life. The consolation
is the knowledge that this "first nature" was once a second, and that
every conquering "second nature" becomes a first.


IV.

This is how history can serve life. Every man and nation needs a
certain knowledge of the past, whether it be through monumental,
antiquarian, or critical history, according to his objects, powers,
and necessities. The need is not that of the mere thinkers who only
look on at life, or the few who desire knowledge and can only be
satisfied with knowledge; but it has always a reference to the end of
life, and is under its absolute rule and direction. This is the
natural relation of an age, a culture and a people to history; hunger
is its source, necessity its norm, the inner plastic power assigns
its limits. The knowledge of the past is only desired for the service
of the future and the present, not to weaken the present or undermine
a living future. All this is as simple as truth itself, and quite
convincing to any one who is not in the toils of "historical
deduction."

And now to take a quick glance at our time! We fly back in
astonishment. The clearness, naturalness, and purity of the
connection between life and history has vanished; and in what a maze
of exaggeration and contradiction do we now see the problem! Is the
guilt ours who see it, or have life and history really altered their
conjunction and an inauspicious star risen between them? Others may
prove we have seen falsely; I am merely saying what we believe we
see. There is such a star, a bright and lordly star, and the
conjunction is really altered--by science, and the demand for history
to be a science. Life is no more dominant, and knowledge of the past
no longer its thrall: boundary marks are overthrown everything bursts
its limits. The perspective of events is blurred, and the blur
extends through their whole immeasurable course. No generation has
seen such a panoramic comedy as is shown by the "science of universal
evolution," history; that shows it with the dangerous audacity of its
motto--"Fiat veritas, pereat vita."

Let me give a picture of the spiritual events in the soul of the
modern man. Historical knowledge streams on him from sources that are
inexhaustible, strange incoherencies come together, memory opens all
its gates and yet is never open wide enough, nature busies herself to
receive all the foreign guests, to honour them and put them in their
places. But they are at war with each other: violent measures seem
necessary, in order to escape destruction one's self. It becomes
second nature to grow gradually accustomed to this irregular and
stormy home-life, though this second nature is unquestionably weaker,
more restless, more radically unsound than the first. The modern man
carries inside him an enormous heap of indigestible knowledge-stones
that occasionally rattle together in his body, as the fairy-tale has
it. And the rattle reveals the most striking characteristic of these
modern men, the opposition of something inside them to which nothing
external corresponds; and the reverse. The ancient nations knew
nothing of this. Knowledge, taken in excess without hunger, even
contrary to desire, has no more the effect of transforming the
external life; and remains hidden in a chaotic inner world that the
modern man has a curious pride in calling his "real personality." He
has the substance, he says, and only wants the form; but this is
quite an unreal opposition in a living thing. Our modern culture is
for that reason not a living one, because it cannot be understood
without that opposition. In other words, it is not a real culture but
a kind of knowledge about culture, a complex of various thoughts and
feelings about it, from which no decision as to its direction can
come. Its real motive force that issues in visible action is often no
more than a mere convention, a wretched imitation, or even a
shameless caricature. The man probably feels like the snake that has
swallowed a rabbit whole and lies still in the sun, avoiding all
movement not absolutely necessary. The "inner life" is now the only
thing that matters to education, and all who see it hope that the
education may not fail by being too indigestible. Imagine a Greek
meeting it; he would observe that for modern men "education" and
"historical education" seem to mean the same thing, with the
difference that the one phrase is longer. And if he spoke of his own
theory, that a man can be very well educated without any history at
all, people would shake their heads and think they had not heard
aright. The Greeks, the famous people of a past still near to us, had
the "unhistorical sense" strongly developed in the period of the
greatest power. If a typical child of this age were transported to
that world by some enchantment, he would probably find the Greeks
very "uneducated." And that discovery would betray the closely
guarded secret of modern culture to the laughter of the world. For we
moderns have nothing of our own. We only become worth notice by
filling ourselves to overflowing with foreign customs, arts,
philosophies, religions and sciences: we are wandering encyclopædias,
as an ancient Greek who had strayed into our time would probably call
us. But the only value of an encyclopædia lies in the inside, in the
contents, not in what is written outside, in the binding or the
wrapper. And so the whole of modern culture is essentially internal;
the bookbinder prints something like this on the cover: "Manual of
internal culture for external barbarians." The opposition of inner
and outer makes the outer side still more barbarous, as it would
naturally be, when the outward growth of a rude people merely
developed its primitive inner needs. For what means has nature of
repressing too great a luxuriance from without? Only one,--to be
affected by it as little as possible, to set it aside and stamp it
out at the first opportunity. And so we have the custom of no longer
taking real things seriously, we get the feeble personality on which
the real and the permanent make so little impression. Men become at
last more careless and accommodating in external matters, and the
considerable cleft between substance and form is widened; until they
have no longer any feeling for barbarism, if only their memories be
kept continually titillated, and there flow a constant stream of new
things to be known, that can be neatly packed up in the cupboards of
their memory. The culture of a people as against this barbarism, can
be, I think, described with justice as the "unity of artistic style
in every outward expression of the people's life." This must not be
misunderstood, as though it were merely a question of the opposition
between barbarism and "fine style." The people that can be called
cultured, must be in a real sense a living unity, and not be
miserably cleft asunder into form and substance. If one wish to
promote a people's culture, let him try to promote this higher unity
first, and work for the destruction of the modern educative system
for the sake of a true education. Let him dare to consider how the
health of a people that has been destroyed by history may be
restored, and how it may recover its instincts with its honour.

I am only speaking, directly, about the Germans of the present day,
who have had to suffer more than other people from the feebleness of
personality and the opposition of substance and form. "Form"
generally implies for us some convention, disguise or hypocrisy, and
if not hated, is at any rate not loved. We have an extraordinary fear
of both the word convention and the thing. This fear drove the German
from the French school; for he wished to become more natural, and
therefore more German. But he seems to have come to a false
conclusion with his "therefore." First he ran away from his school of
convention, and went by any road he liked: he has come ultimately to
imitate voluntarily in a slovenly fashion, what he imitated painfully
and often successfully before. So now the lazy fellow lives under
French conventions that are actually incorrect: his manner of walking
shows it, his conversation and dress, his general way of life. In the
belief that he was returning to Nature, he merely followed caprice
and comfort, with the smallest possible amount of self-control. Go
through any German town; you will see conventions that are nothing
but the negative aspect of the national characteristics of foreign
states. Everything is colourless, worn out, shoddy and ill-copied.
Every one acts at his own sweet will--which is not a strong or
serious will--on laws dictated by the universal rush and the general
desire for comfort. A dress that made no head ache in its inventing
and wasted no time in the making, borrowed from foreign models and
imperfectly copied, is regarded as an important contribution to
German fashion. The sense of form is ironically disclaimed by the
people--for they have the "sense of substance": they are famous for
their cult of "inwardness."

But there is also a famous danger in their "inwardness": the internal
substance cannot be seen from the outside, and so may one day take
the opportunity of vanishing, and no one notice its absence, any more
than its presence before. One may think the German people to be very
far from this danger: yet the foreigner will have some warrant for
his reproach that our inward life is too weak and ill-organised to
provide a form and external expression for itself. It may in rare
cases show itself finely receptive, earnest and powerful, richer
perhaps than the inward life of other peoples; but, taken as a whole,
it remains weak, as all its fine threads are not tied together in one
strong knot. The visible action is not the self-manifestation of the
inward life, but only a weak and crude attempt of a single thread to
make a show of representing the whole. And thus the German is not to
be judged on any one action, for the individual may be as completely
obscure after it as before. He must obviously be measured by his
thoughts and feelings, which are now expressed in his books; if only
the books did not, more than ever, raise the doubt whether the famous
inward life is still really sitting in its inaccessible shrine. It
might one day vanish and leave behind it only the external
life,--with its vulgar pride and vain servility,--to mark the German.
Fearful thought!--as fearful as if the inward life still sat there,
painted and rouged and disguised, become a play-actress or something
worse; as his theatrical experience seems to have taught the quiet
observer Grillparzer, standing aside as he did from the main press.
"We feel by theory," he says. "We hardly know any more how our
contemporaries give expression to their feelings: we make them use
gestures that are impossible nowadays. Shakespeare has spoilt us
moderns."

This is a single example, its general application perhaps too hastily
assumed. But how terrible it would be were that generalisation
justified before our eyes! There would be then a note of despair in
the phrase, "We Germans feel by theory, we are all spoilt by
history;"--a phrase that would cut at the roots of any hope for a
future national culture. For every hope of that kind grows from the
belief in the genuineness and immediacy of German feeling, from the
belief in an untarnished inward life. Where is our hope or belief,
when its spring is muddied, and the inward quality has learned
gestures and dances and the use of cosmetics, has learned to express
itself "with due reflection in abstract terms," and gradually lose
itself? And how should a great productive spirit exist among a nation
that is not sure of its inward unity and is divided into educated men
whose inner life has been drawn from the true path of education, and
uneducated men whose inner life cannot be approached at all? How
should it exist, I say, when the people has lost its own unity of
feeling, and knows that the feeling of the part calling itself the
educated part and claiming the right of controlling the artistic
spirit of the nation, is false and hypocritical? Here and there the
judgment and taste of individuals may be higher and finer than the
rest, but that is no compensation: it tortures a man to have to speak
only to one section and be no longer in sympathy with his people. He
would rather bury his treasure now, in disgust at the vulgar
patronage of a class, though his heart be filled with tenderness for
all. The instinct of the people can no longer meet him half-way; it
is useless for them to stretch their arms out to him in yearning.
What remains but to turn his quickened hatred against the ban, strike
at the barrier raised by the so-called culture, and condemn as judge
what blasted and degraded him as a living man and a source of life?
He takes a profound insight into fate in exchange for the godlike
desire of creation and help, and ends his days as a lonely
philosopher, with the wisdom of disillusion. It is the painfullest
comedy: he who sees it will feel a sacred obligation on him, and say
to himself,--"Help must come: the higher unity in the nature and soul
of a people must be brought back, the cleft between inner and outer
must again disappear under the hammer of necessity." But to what
means can he look? What remains to him now but his knowledge? He
hopes to plant the feeling of a need, by speaking from the breadth of
that knowledge, giving it freely with both hands. From the strong
need the strong action may one day arise. And to leave no doubt of
the instance I am taking of the need and the knowledge, my testimony
shall stand, that it is German unity in its highest sense which is
the goal of our endeavour, far more than political union: it is the
unity of the German spirit and life after the annihilation of the
antagonism between form and substance, inward life and convention.


V.

An excess of history seems to be an enemy to the life of a time, and
dangerous in five ways. Firstly, the contrast of inner and outer is
emphasised and personality weakened. Secondly, the time comes to
imagine that it possesses the rarest of virtues, justice, to a higher
degree than any other time. Thirdly, the instincts of a nation are
thwarted, the maturity of the individual arrested no less than that
of the whole. Fourthly, we get the belief in the old age of mankind,
the belief, at all times harmful, that we are late survivals, mere
Epigoni. Lastly, an age reaches a dangerous condition of irony with
regard to itself, and the still more dangerous state of cynicism,
when a cunning egoistic theory of action is matured that maims and at
last destroys the vital strength.

To return to the first point: the modern man suffers from a weakened
personality. The Roman of the Empire ceased to be a Roman through the
contemplation of the world that lay at his feet; he lost himself in
the crowd of foreigners that streamed into Rome, and degenerated amid
the cosmopolitan carnival of arts, worships and moralities. It is the
same with the modern man, who is continually having a world-panorama
unrolled before his eyes by his historical artists. He is turned into
a restless, dilettante spectator, and arrives at a condition when
even great wars and revolutions cannot affect him beyond the moment.
The war is hardly at an end, and it is already converted into
thousands of copies of printed matter, and will be soon served up as
the latest means of tickling the jaded palates of the historical
gourmets. It seems impossible for a strong full chord to be
prolonged, however powerfully the strings are swept: it dies away
again the next moment in the soft and strengthless echo of history.
In ethical language, one never succeeds in staying on a height; your
deeds are sudden crashes, and not a long roll of thunder. One may
bring the greatest and most marvellous thing to perfection; it must
yet go down to Orcus unhonoured and unsung. For art flies away when
you are roofing your deeds with the historical awning. The man who
wishes to understand everything in a moment, when he ought to grasp
the unintelligible as the sublime by a long struggle, can be called
intelligent only in the sense of Schiller's epigram on the "reason of
reasonable men." There is something the child sees that he does not
see; something the child hears that he does not hear; and this
something is the most important thing of all. Because he does not
understand it, his understanding is more childish than the child's
and more simple than simplicity itself; in spite of the many clever
wrinkles on his parchment face, and the masterly play of his fingers
in unravelling the knots. He has lost or destroyed his instinct; he
can no longer trust the "divine animal" and let the reins hang loose,
when his understanding fails him and his way lies through the desert.
His individuality is shaken, and left without any sure belief in
itself; it sinks into its own inner being, which only means here the
disordered chaos of what it has learned, which will never express
itself externally, being mere dogma that cannot turn to life. Looking
further, we see how the banishment of instinct by history has turned
men to shades and abstractions: no one ventures to show a
personality, but masks himself as a man of culture, a savant, poet or
politician.

If one take hold of these masks, believing he has to do with a
serious thing and not a mere puppet-show--for they all have an
appearance of seriousness--he will find nothing but rags and coloured
streamers in his hands. He must deceive himself no more, but cry
aloud, "Off with your jackets, or be what you seem!" A man of the
royal stock of seriousness must no longer be Don Quixote, for he has
better things to do than to tilt at such pretended realities. But he
must always keep a sharp look about him, call his "Halt! who goes
there?" to all the shrouded figures, and tear the masks from their
faces. And see the result! One might have thought that history
encouraged men above all to be honest, even if it were only to be
honest fools: this used to be its effect, but is so no longer.
Historical education and the uniform frock-coat of the citizen are
both dominant at the same time. While there has never been such a
full-throated chatter about "free personality," personalities can be
seen no more (to say nothing of free ones); but merely men in
uniform, with their coats anxiously pulled over their ears.
Individuality has withdrawn itself to its recesses; it is seen no
more from the outside, which makes one doubt if it be possible to
have causes without effects. Or will a race of eunuchs prove to be
necessary to guard the historical harem of the world? We can
understand the reason for their aloofness very well. Does it not seem
as if their task were to watch over history to see that nothing comes
out except other histories, but no deed that might be historical; to
prevent personalities becoming "free," that is, sincere towards
themselves and others, both in word and deed? Only through this
sincerity will the inner need and misery of the modern man be brought
to the light, and art and religion come as true helpers in the place
of that sad hypocrisy of convention and masquerade, to plant a common
culture which will answer to real necessities, and not teach, as the
present "liberal education" teaches, to tell lies about these needs,
and thus become a walking lie one's self.

In such an age, that suffers from its "liberal education," how
unnatural, artificial and unworthy will be the conditions under which
the sincerest of all sciences, the holy naked goddess Philosophy,
must exist! She remains, in such a world of compulsion and outward
conformity, the subject of the deep monologue of the lonely wanderer
or the chance prey of any hunter, the dark secret of the chamber or
the daily talk of the old men and children at the university. No one
dare fulfil the law of philosophy in himself; no one lives
philosophically, with that single-hearted virile faith that forced
one of the olden time to bear himself as a Stoic, wherever he was and
whatever he did, if he had once sworn allegiance to the Stoa. All
modern philosophising is political or official, bound down to be a
mere phantasmagoria of learning by our modern governments, churches,
universities, moralities and cowardices: it lives by sighing "if
only...." and by knowing that "it happened once upon a time...."
Philosophy has no place in historical education, if it will be more
than the knowledge that lives indoors, and can have no expression in
action. Were the modern man once courageous and determined, and not
merely such an indoor being even in his hatreds, he would banish
philosophy. At present he is satisfied with modestly covering her
nakedness. Yes, men think, write, print, speak and teach
philosophically: so much is permitted them. It is only otherwise in
action, in "life." Only one thing is permitted there, and everything
else quite impossible: such are the orders of historical education.
"Are these human beings," one might ask, "or only machines for
thinking, writing and speaking?"

Goethe says of Shakespeare: "No one has more despised correctness of
costume than he: he knows too well the inner costume that all men
wear alike. You hear that he describes Romans wonderfully; I do not
think so: they are flesh-and-blood Englishmen; but at any rate they
are men from top to toe, and the Roman toga sits well on them." Would
it be possible, I wonder, to represent our present literary and
national heroes, officials and politicians as Romans? I am sure it
would not, as they are no men, but incarnate compendia, abstractions
made concrete. If they have a character of their own, it is so deeply
sunk that it can never rise to the light of day: if they are men,
they are only men to a physiologist. To all others they are something
else, not men, not "beasts or gods," but historical pictures of the
march of civilisation, and nothing but pictures and civilisation,
form without any ascertainable substance, bad form unfortunately, and
uniform at that. And in this way my thesis is to be understood and
considered: "only strong personalities can endure history, the weak
are extinguished by it." History unsettles the feelings when they are
not powerful enough to measure the past by themselves. The man who
dare no longer trust himself, but asks history against his will for
advice "how he ought to feel now," is insensibly turned by his
timidity into a play-actor, and plays a part, or generally many
parts,--very badly therefore and superficially. Gradually all
connection ceases between the man and his historical subjects. We see
noisy little fellows measuring themselves with the Romans as though
they were like them: they burrow in the remains of the Greek poets,
as if these were _corpora_ for their dissection--and as _vilia_ as
their own well-educated _corpora_ might be. Suppose a man is working
at Democritus. The question is always on my tongue, why precisely
Democritus? Why not Heraclitus, or Philo, or Bacon, or Descartes? And
then, why a philosopher? Why not a poet or orator? And why especially
a Greek? Why not an Englishman or a Turk? Is not the past large
enough to let you find some place where you may disport yourself
without becoming ridiculous? But, as I said, they are a race of
eunuchs: and to the eunuch one woman is the same as another, merely a
woman, "woman in herself," the Ever-unapproachable. And it is
indifferent what they study, if history itself always remain
beautifully "objective" to them, as men, in fact, who could never
make history themselves. And since the Eternal Feminine could never
"draw you upward," you draw it down to you, and being neuter
yourselves, regard history as neuter also. But in order that no one
may take any comparison of history and the Eternal Feminine too
seriously, I will say at once that I hold it, on the contrary, to be
the Eternal Masculine: I only add that for those who are
"historically trained" throughout, it must be quite indifferent which
it is; for they are themselves neither man nor woman, nor even
hermaphrodite, but mere neuters, or, in more philosophic language,
the Eternal Objective.

If the personality be once emptied of its subjectivity, and come to
what men call an "objective" condition, nothing can have any more
effect on it. Something good and true may be done, in action, poetry
or music: but the hollow culture of the day will look beyond the work
and ask the history of the author. If the author have already created
something, our historian will set out clearly the past and the
probable future course of his development, he will put him with
others and compare them, and separate by analysis the choice of his
material and his treatment; he will wisely sum the author up and give
him general advice for his future path. The most astonishing works
may be created; the swarm of historical neuters will always be in
their place, ready to consider the authors through their long
telescopes. The echo is heard at once: but always in the form of
"criticism," though the critic never dreamed of the work's
possibility a moment before. It never comes to have an influence, but
only a criticism: and the criticism itself has no influence, but only
breeds another criticism. And so we come to consider the fact of many
critics as a mark of influence, that of few or none as a mark of
failure. Actually everything remains in the old condition, even in
the presence of such "influence": men talk a little while of a new
thing, and then of some other new thing, and in the meantime they do
what they have always done. The historical training of our critics
prevents their having an influence in the true sense, an influence on
life and action. They put their blotting paper on the blackest
writing, and their thick brushes over the gracefullest designs; these
they call "corrections";--and that is all. Their critical pens never
cease to fly, for they have lost power over them; they are driven by
their pens instead of driving them. The weakness of modern
personality comes out well in the measureless overflow of criticism,
in the want of self-mastery, and in what the Romans called
_impotentia_.


VI.

But leaving these weaklings, let us turn rather to a point of
strength for which the modern man is famous. Let us ask the painful
question whether he has the right in virtue of his historical
"objectivity" to call himself strong and just in a higher degree than
the man of another age. Is it true that this objectivity has its
source in a heightened sense of the need for justice? Or, being
really an effect of quite other causes, does it only have the
appearance of coming from justice, and really lead to an unhealthy
prejudice in favour of the modern man? Socrates thought it near
madness to imagine one possessed a virtue without really possessing
it. Such imagination has certainly more danger in it than the
contrary madness of a positive vice. For of this there is still a
cure; but the other makes a man or a time daily worse, and therefore
more unjust.

No one has a higher claim to our reverence than the man with the
feeling and the strength for justice. For the highest and rarest
virtues unite and are lost in it, as an unfathomable sea absorbs the
streams that flow from every side. The hand of the just man, who is
called to sit in judgment, trembles no more when it holds the scales:
he piles the weights inexorably against his own side, his eyes are
not dimmed as the balance rises and falls, and his voice is neither
hard nor broken when he pronounces the sentence. Were he a cold demon
of knowledge, he would cast round him the icy atmosphere of an awful,
superhuman majesty, that we should fear, not reverence. But he is a
man, and has tried to rise from a careless doubt to a strong
certainty, from gentle tolerance to the imperative "thou must"; from
the rare virtue of magnanimity to the rarest, of justice. He has come
to be like that demon without being more than a poor mortal at the
outset; above all, he has to atone to himself for his humanity and
tragically shatter his own nature on the rock of an impossible
virtue.--All this places him on a lonely height as the most reverend
example of the human race. For truth is his aim, not in the form of
cold intellectual knowledge, but the truth of the judge who punishes
according to law; not as the selfish possession of an individual, but
the sacred authority that removes the boundary stones from all
selfish possessions; truth, in a word, as the tribunal of the world,
and not as the chance prey of a single hunter. The search for truth
is often thoughtlessly praised: but it only has anything great in it
if the seeker have the sincere unconditional will for justice. Its
roots are in justice alone: but a whole crowd of different motives
may combine in the search for it, that have nothing to do with truth
at all; curiosity, for example, or dread of ennui, envy, vanity, or
amusement. Thus the world seems to be full of men who "serve truth":
and yet the virtue of justice is seldom present, more seldom known,
and almost always mortally hated. On the other hand a throng of sham
virtues has entered in at all times with pomp and honour.

Few in truth serve truth, as only few have the pure will for justice;
and very few even of these have the strength to be just. The will
alone is not enough: the impulse to justice without the power of
judgment has been the cause of the greatest suffering to men. And
thus the common good could require nothing better than for the seed
of this power to be strewn as widely as possible, that the fanatic
may be distinguished from the true judge, and the blind desire from
the conscious power. But there are no means of planting a power of
judgment: and so when one speaks to men of truth and justice, they
will be ever troubled by the doubt whether it be the fanatic or the
judge who is speaking to them. And they must be pardoned for always
treating the "servants of truth" with special kindness, who possess
neither the will nor the power to judge and have set before them the
task of finding "pure knowledge without reference to consequences,"
knowledge, in plain terms, that comes to nothing. There are very many
truths which are unimportant; problems that require no struggle to
solve, to say nothing of sacrifice. And in this safe realm of
indifference a man may very successfully become a "cold demon of
knowledge." And yet--if we find whole regiments of learned inquirers
being turned to such demons in some age specially favourable to them,
it is always unfortunately possible that the age is lacking in a
great and strong sense of justice, the noblest spring of the
so-called impulse to truth.

Consider the historical virtuoso of the present time: is he the
justest man of his age? True, he has developed in himself such a
delicacy and sensitiveness that "nothing human is alien to him."
Times and persons most widely separated come together in the concords
of his lyre. He has become a passive instrument, whose tones find an
echo in similar instruments: until the whole atmosphere of a time is
filled with such echoes, all buzzing in one soft chord. Yet I think
one only hears the overtones of the original historical note: its
rough powerful quality can be no longer guessed from these thin and
shrill vibrations. The original note sang of action, need, and
terror; the overtone lulls us into a soft dilettante sleep. It is as
though the heroic symphony had been arranged for two flutes for the
use of dreaming opium-smokers. We can now judge how these virtuosi
stand towards the claim of the modern man to a higher and purer
conception of justice. This virtue has never a pleasing quality; it
never charms; it is harsh and strident. Generosity stands very low on
the ladder of the virtues in comparison; and generosity is the mark
of a few rare historians! Most of them only get as far as tolerance,
in other words they leave what cannot be explained away, they correct
it and touch it up condescendingly, on the tacit assumption that the
novice will count it as justice if the past be narrated without
harshness or open expressions of hatred. But only superior strength
can really judge; weakness must tolerate, if it do not pretend to be
strength and turn justice to a play-actress. There is still a
dreadful class of historians remaining--clever, stern and honest, but
narrow-minded: who have the "good will" to be just with a pathetic
belief in their actual judgments, which are all false; for the same
reason, almost, as the verdicts of the usual juries are false. How
difficult it is to find a real historical talent, if we exclude all
the disguised egoists, and the partisans who pretend to take up an
impartial attitude for the sake of their own unholy game! And we also
exclude the thoughtless folk who write history in the naïve faith
that justice resides in the popular view of their time, and that to
write in the spirit of the time is to be just; a faith that is found
in all religions, and which, in religion, serves very well. The
measurement of the opinions and deeds of the past by the universal
opinions of the present is called "objectivity" by these simple
people: they find the canon of all truth here: their work is to adapt
the past to the present triviality. And they call all historical
writing "subjective" that does not regard these popular opinions as
canonical.

Might not an illusion lurk in the highest interpretation of the word
objectivity? We understand by it a certain standpoint in the
historian, who sees the procession of motive and consequence too
clearly for it to have an effect on his own personality. We think of
the æsthetic phenomenon of the detachment from all personal concern
with which the painter sees the picture and forgets himself, in a
stormy landscape, amid thunder and lightning, or on a rough sea: and
we require the same artistic vision and absorption in his object from
the historian. But it is only a superstition to say that the picture
given to such a man by the object really shows the truth of things.
Unless it be that objects are expected in such moments to paint or
photograph themselves by their own activity on a purely passive
medium!

But this would be a myth, and a bad one at that. One forgets that
this moment is actually the powerful and spontaneous moment of
creation in the artist, of "composition" in its highest form, of
which the result will be an artistically, but not an historically,
true picture. To think objectively, in this sense, of history is the
work of the dramatist: to think one thing with another, and weave the
elements into a single whole; with the presumption that the unity of
plan must be put into the objects if it be not already there. So man
veils and subdues the past, and expresses his impulse to art--but not
his impulse to truth or justice. Objectivity and justice have nothing
to do with each other. There could be a kind of historical writing
that had no drop of common fact in it and yet could claim to be
called in the highest degree objective. Grillparzer goes so far as to
say that "history is nothing but the manner in which the spirit of
man apprehends facts that are obscure to him, links things together
whose connection heaven only knows, replaces the unintelligible by
something intelligible, puts his own ideas of causation into the
external world, which can perhaps be explained only from within: and
assumes the existence of chance, where thousands of small causes may
be really at work. Each man has his own individual needs, and so
millions of tendencies are running together, straight or crooked,
parallel or across, forward or backward, helping or hindering each
other. They have all the appearance of chance, and make it
impossible, quite apart from all natural influences, to establish any
universal lines on which past events must have run." But as a result
of this so-called "objective" way of looking at things, such a "must"
ought to be made clear. It is a presumption that takes a curious form
if adopted by the historian as a dogma. Schiller is quite clear about
its truly subjective nature when he says of the historian, "one event
after the other begins to draw away from blind chance and lawless
freedom, and take its place as the member of an harmonious
whole--_which is of course only apparent in its presentation_." But
what is one to think of the innocent statement, wavering between
tautology and nonsense, of a famous historical virtuoso? "It seems
that all human actions and impulses are subordinate to the process of
the material world, that works unnoticed, powerfully and
irresistibly." In such a sentence one no longer finds obscure wisdom
in the form of obvious folly; as in the saying of Goethe's gardener,
"Nature may be forced but not compelled," or in the notice on the
side-show at a fair, in Swift: "The largest elephant in the world,
except himself, to be seen here." For what opposition is there
between human action and the process of the world? It seems to me
that such historians cease to be instructive as soon as they begin to
generalise; their weakness is shown by their obscurity. In other
sciences the generalisations are the most important things, as they
contain the laws. But if such generalisations as these are to stand
as laws, the historian's labour is lost; for the residue of truth,
after the obscure and insoluble part is removed, is nothing but the
commonest knowledge. The smallest range of experience will teach it.
But to worry whole peoples for the purpose, and spend many hard years
of work on it, is like crowding one scientific experiment on another
long after the law can be deduced from the results already obtained:
and this absurd excess of experiment has been the bane of all natural
science since Zollner. If the value of a drama lay merely in its
final scene, the drama itself would be a very long, crooked and
laborious road to the goal: and I hope history will not find its
whole significance in general propositions, and regard them as its
blossom and fruit. On the contrary, its real value lies in inventing
ingenious variations on a probably commonplace theme, in raising the
popular melody to a universal symbol and showing what a world of
depth, power and beauty exists in it.

But this requires above all a great artistic faculty, a creative
vision from a height, the loving study of the data of experience, the
free elaborating of a given type,--objectivity in fact, though this
time as a positive quality. Objectivity is so often merely a phrase.
Instead of the quiet gaze of the artist that is lit by an inward
flame, we have an affectation of tranquillity; just as a cold
detachment may mask a lack of moral feeling. In some cases a
triviality of thought, the everyday wisdom that is too dull not to
seem calm and disinterested, comes to represent the artistic
condition in which the subjective side has quite sunk out of sight.
Everything is favoured that does not rouse emotion, and the driest
phrase is the correct one. They go so far as to accept a man who is
_not affected at all_ by some particular moment in the past as the
right man to describe it. This is the usual relation of the Greeks
and the classical scholars. They have nothing to do with each
other--and this is called "objectivity"! The intentional air of
detachment that is assumed for effect, the sober art of the
superficial motive-hunter is most exasperating when the highest and
rarest things are in question; and it is the _vanity_ of the
historian that drives him to this attitude of indifference. He goes
to justify the axiom that a man's vanity corresponds to his lack of
wit. No, be honest at any rate! Do not pretend to the artist's
strength, that is the real objectivity; do not try to be just, if you
are not born to that dread vocation. As if it were the task of every
time to be just to everything before it! Ages and generations have
never the right to be the judges of all previous ages and
generations: only to the rarest men in them can that difficult
mission fall. Who compels you to judge? If it is your wish--you must
prove first that you are capable of justice. As judges, you must
stand higher than that which is to be judged: as it is, you have only
come later. The guests that come last to the table should rightly
take the last places: and will you take the first? Then do some great
and mighty deed: the place may be prepared for you then, even though
you do come last.

_You can only explain the past by what is highest in the present._
Only by straining the noblest qualities you have to their highest
power will you find out what is greatest in the past, most worth
knowing and preserving. Like by like! otherwise you will draw the
past to your own level. Do not believe any history that does not
spring from the mind of a rare spirit. You will know the quality of
the spirit, by its being forced to say something universal, or to
repeat something that is known already; the fine historian must have
the power of coining the known into a thing never heard before and
proclaiming the universal so simply and profoundly that the simple is
lost in the profound, and the profound in the simple. No one can be a
great historian and artist, and a shallowpate at the same time. But
one must not despise the workers who sift and cast together the
material because they can never become great historians. They must,
still less, be confounded with them, for they are the necessary
bricklayers and apprentices in the service of the master: just as the
French used to speak, more naïvely than a German would, of the
"historiens de M. Thiers." These workmen should gradually become
extremely learned, but never, for that reason, turn to be masters.
Great learning and great shallowness go together very well under one
hat.

Thus, history is to be written by the man of experience and
character. He who has not lived through something greater and nobler
than others, will not be able to explain anything great and noble in
the past. The language of the past is always oracular: you will only
understand it as builders of the future who know the present. We can
only explain the extraordinarily wide influence of Delphi by the fact
that the Delphic priests had an exact knowledge of the past: and,
similarly, only he who is building up the future has a right to judge
the past. If you set a great aim before your eyes, you control at the
same time the itch for analysis that makes the present into a desert
for you, and all rest, all peaceful growth and ripening, impossible.
Hedge yourselves with a great, all-embracing hope, and strive on.
Make of yourselves a mirror where the future may see itself, and
forget the superstition that you are Epigoni. You have enough to
ponder and find out, in pondering the life of the future: but do not
ask history to show you the means and the instrument to it. If you
live yourselves back into the history of great men, you will find in
it the high command to come to maturity and leave that blighting
system of cultivation offered by your time: which sees its own profit
in not allowing you to become ripe, that it may use and dominate you
while you are yet unripe. And if you want biographies, do not look
for those with the legend "Mr. So-and-so and his times," but for one
whose title-page might be inscribed "a fighter against his time."
Feast your souls on Plutarch, and dare to believe in yourselves when
you believe in his heroes. A hundred such men--educated against the
fashion of to-day, made familiar with the heroic, and come to
maturity--are enough to give an eternal quietus to the noisy sham
education of this time.


VII.

The unrestrained historical sense, pushed to its logical extreme,
uproots the future, because it destroys illusions and robs existing
things of the only atmosphere in which they can live. Historical
justice, even if practised conscientiously, with a pure heart, is
therefore a dreadful virtue, because it always undermines and ruins
the living thing: its judgment always means annihilation. If there be
no constructive impulse behind the historical one, if the clearance
of rubbish be not merely to leave the ground free for the hopeful
living future to build its house, if justice alone be supreme, the
creative instinct is sapped and discouraged. A religion, for example,
that has to be turned into a matter of historical knowledge by the
power of pure justice, and to be scientifically studied throughout,
is destroyed at the end of it all. For the historical audit brings so
much to light which is false and absurd, violent and inhuman, that
the condition of pious illusion falls to pieces. And a thing can only
live through a pious illusion. For man is creative only through love
and in the shadow of love's illusions, only through the unconditional
belief in perfection and righteousness. Everything that forces a man
to be no longer unconditioned in his love, cuts at the root of his
strength: he must wither, and be dishonoured. Art has the opposite
effect to history: and only perhaps if history suffer transformation
into a pure work of art, can it preserve instincts or arouse them.
Such history would be quite against the analytical and inartistic
tendencies of our time, and even be considered false. But the history
that merely destroys without any impulse to construct, will in the
long-run make its instruments tired of life; for such men destroy
illusions, and "he who destroys illusions in himself and others is
punished by the ultimate tyrant, Nature." For a time a man can take
up history like any other study, and it will be perfectly harmless.
Recent theology seems to have entered quite innocently into
partnership with history, and scarcely sees even now that it has
unwittingly bound itself to the Voltairean _écrasez_! No one need
expect from that any new and powerful constructive impulse: they
might as well have let the so-called Protestant Union serve as the
cradle of a new religion, and the jurist Holtzendorf, the editor of
the far more dubiously named Protestant Bible, be its John the
Baptist. This state of innocence may be continued for some time by
the Hegelian philosophy,--still seething in some of the older
heads,--by which men can distinguish the "idea of Christianity" from
its various imperfect "manifestations"; and persuade themselves that
it is the "self-movement of the Idea" that is ever particularising
itself in purer and purer forms, and at last becomes the purest, most
transparent, in fact scarcely visible form in the brain of the
present _theologus liberalis vulgaris_. But to listen to this pure
Christianity speaking its mind about the earlier impure Christianity,
the uninitiated hearer would often get the impression that the talk
was not of Christianity at all but of ...--what are we to think? if
we find Christianity described by the "greatest theologians of the
century" as the religion that claims to "find itself in all real
religions and some other barely possible religions," and if the "true
church" is to be a thing "which may become a liquid mass with no
fixed outline, with no fixed place for its different parts, but
everything to be peacefully welded together"--what, I ask again, are
we to think?

Christianity has been denaturalised by historical treatment--which in
its most complete form means "just" treatment--until it has been
resolved into pure knowledge and destroyed in the process. This can
be studied in everything that has life. For it ceases to have life if
it be perfectly dissected, and lives in pain and anguish as soon as
the historical dissection begins. There are some who believe in the
saving power of German music to revolutionise the German nature. They
angrily exclaim against the special injustice done to our culture,
when such men as Mozart and Beethoven are beginning to be spattered
with the learned mud of the biographers and forced to answer a
thousand searching questions on the rack of historical criticism. Is
it not premature death, or at least mutilation, for anything whose
living influence is not yet exhausted, when men turn their curious
eyes to the little minutiæ of life and art, and look for problems of
knowledge where one ought to learn to live, and forget problems? Set
a couple of these modern biographers to consider the origins of
Christianity or the Lutheran reformation: their sober, practical
investigations would be quite sufficient to make all spiritual
"action at a distance" impossible: just as the smallest animal can
prevent the growth of the mightiest oak by simply eating up the
acorn. All living things need an atmosphere, a mysterious mist,
around them. If that veil be taken away and a religion, an art, or a
genius condemned to revolve like a star without an atmosphere, we
must not be surprised if it becomes hard and unfruitful, and soon
withers. It is so with all great things "that never prosper without
some illusion," as Hans Sachs says in the Meistersinger.

Every people, every man even, who would become ripe, needs such a
veil of illusion, such a protecting cloud. But now men hate to become
ripe, for they honour history above life. They cry in triumph that
"science is now beginning to rule life." Possibly it might; but a
life thus ruled is not of much value. It is not such true _life_, and
promises much less for the future than the life that used to be
guided not by science, but by instincts and powerful illusions. But
this is not to be the age of ripe, alert and harmonious
personalities, but of work that may be of most use to the
commonwealth. Men are to be fashioned to the needs of the time, that
they may soon take their place in the machine. They must work in the
factory of the "common good" before they are ripe, or rather to
prevent them becoming ripe; for this would be a luxury that would
draw away a deal of power from the "labour market." Some birds are
blinded that they may sing better; I do not think men sing to-day
better than their grandfathers, though I am sure they are blinded
early. But light, too clear, too sudden and dazzling, is the infamous
means used to blind them. The young man is kicked through all the
centuries: boys who know nothing of war, diplomacy, or commerce are
considered fit to be introduced to political history. We moderns also
run through art galleries and hear concerts in the same way as the
young man runs through history. We can feel that one thing sounds
differently from another, and pronounce on the different "effects."
And the power of gradually losing all feelings of strangeness or
astonishment, and finally being pleased at anything, is called the
historical sense, or historical culture. The crowd of influences
streaming on the young soul is so great, the clods of barbarism and
violence flung at him so strange and overwhelming, that an assumed
stupidity is his only refuge. Where there is a subtler and stronger
self-consciousness we find another emotion too--disgust. The young
man has become homeless: he doubts all ideas, all moralities. He
knows "it was different in every age, and what you are does not
matter." In a heavy apathy he lets opinion on opinion pass by him,
and understands the meaning of Hölderlin's words when he read the
work of Diogenes Laertius on the lives and doctrines of the Greek
philosophers: "I have seen here too what has often occurred to me,
that the change and waste in men's thoughts and systems is far more
tragic than the fates that overtake what men are accustomed to call
the only realities." No, such study of history bewilders and
overwhelms. It is not necessary for youth, as the ancients show, but
even in the highest degree dangerous, as the moderns show. Consider
the historical student, the heir of ennui, that appears even in his
boyhood. He has the "methods" for original work, the "correct ideas"
and the airs of the master at his fingers' ends. A little isolated
period of the past is marked out for sacrifice. He cleverly applies
his method, and produces something, or rather, in prouder phrase,
"creates" something. He becomes a "servant of truth" and a ruler in
the great domain of history. If he was what they call ripe as a boy,
he is now over-ripe. You only need shake him and wisdom will rattle
down into your lap; but the wisdom is rotten, and every apple has its
worm. Believe me, if men work in the factory of science and have to
make themselves useful before they are really ripe, science is ruined
as much as the slaves who have been employed too soon. I am sorry to
use the common jargon about slave-owners and taskmasters in respect
of such conditions, that might be thought free from any economic
taint: but the words "factory, labour-market, auction-sale, practical
use," and all the auxiliaries of egoism, come involuntarily to the
lips in describing the younger generation of savants. Successful
mediocrity tends to become still more mediocre, science still more
"useful." Our modern savants are only wise on one subject, in all the
rest they are, to say the least, different from those of the old
stamp. In spite of that they demand honour and profit for themselves,
as if the state and public opinion were bound to take the new coinage
for the same value as the old. The carters have made a trade-compact
among themselves, and settled that genius is superfluous, for every
carrier is being re-stamped as one. And probably a later age will see
that their edifices are only carted together and not built. To those
who have ever on their lips the modern cry of battle and
sacrifice--"Division of labour! fall into line!" we may say roundly:
"If you try to further the progress of science as quickly as
possible, you will end by destroying it as quickly as possible; just
as the hen is worn out which you force to lay too many eggs." The
progress of science has been amazingly rapid in the last decade; but
consider the savants, those exhausted hens. They are certainly not
"harmonious" natures: they can merely cackle more than before,
because they lay eggs oftener: but the eggs are always smaller,
though their books are bigger. The natural result of it all is the
favourite "popularising" of science (or rather its feminising and
infantising), the villainous habit of cutting the cloth of science to
fit the figure of the "general public." Goethe saw the abuse in this,
and demanded that science should only influence the outer world by
way of _a nobler ideal of action_. The older generation of savants
had good reason for thinking this abuse an oppressive burden: the
modern savants have an equally good reason for welcoming it, because,
leaving their little corner of knowledge out of account, they are
part of the "general public" themselves, and its needs are theirs.
They only require to take themselves less seriously to be able to
open their little kingdom successfully to popular curiosity. This
easy-going behaviour is called "the modest condescension of the
savant to the people"; whereas in reality he has only "descended" to
himself, so far as he is not a savant but a plebeian. Rise to the
conception of a people, you learned men; you can never have one noble
or high enough. If you thought much of the people, you would have
compassion towards them, and shrink from offering your historical
aquafortis as a refreshing drink. But you really think very little of
them, for you dare not take any reasonable pains for their future;
and you act like practical pessimists, men who feel the coming
catastrophe and become indifferent and careless of their own and
others' existence. "If only the earth last for us: and if it do not
last, it is no matter." Thus they come to live an _ironical_
existence.


VIII.

It may seem a paradox, though it is none, that I should attribute a
kind of "ironical self-consciousness" to an age that is generally so
honestly, and clamorously, vain of its historical training; and
should see a suspicion hovering near it that there is really nothing
to be proud of, and a fear lest the time for rejoicing at historical
knowledge may soon have gone by. Goethe has shown a similar riddle in
man's nature, in his remarkable study of Newton: he finds a "troubled
feeling of his own error" at the base--or rather on the height--of
his being, just as if he was conscious at times of having a deeper
insight into things, that vanished the moment after. This gave him a
certain ironical view of his own nature. And one finds that the
greater and more developed "historical men" are conscious of all the
superstition and absurdity in the belief that a people's education
need be so extremely historical as it is; the mightiest nations,
mightiest in action and influence, have lived otherwise, and their
youth has been trained otherwise. The knowledge gives a sceptical
turn to their minds. "The absurdity and superstition," these sceptics
say, "suit men like ourselves, who come as the latest withered shoots
of a gladder and mightier stock, and fulfil Hesiod's prophecy, that
men will one day be born gray-headed, and that Zeus will destroy that
generation as soon as the sign be visible." Historical culture is
really a kind of inherited grayness, and those who have borne its
mark from childhood must believe instinctively in _the old age of
mankind_. To old age belongs the old man's business of looking back
and casting up his accounts, of seeking consolation in the memories
of the past,--in historical culture. But the human race is tough and
persistent, and will not admit that the lapse of a thousand years, or
a hundred thousand, entitles any one to sum up its progress from the
past to the future; that is, it will not be observed as a whole at
all by that infinitesimal atom, the individual man. What is there in
a couple of thousand years--the period of thirty-four consecutive
human lives of sixty years each--to make us speak of youth at the
beginning, and "the old age of mankind" at the end of them? Does not
this paralysing belief in a fast-fading humanity cover the
misunderstanding of a theological idea, inherited from the Middle
Ages, that the end of the world is approaching and we are waiting
anxiously for the judgment? Does not the increasing demand for
historical judgment give us that idea in a new dress? as if our time
were the latest possible time, and commanded to hold that universal
judgment of the past, which the Christian never expected from a man,
but from "the Son of Man." The _memento mori_, spoken to humanity as
well as the individual, was a sting that never ceased to pain, the
crown of mediæval knowledge and consciousness.

The opposite message of a later time, _memento vivere_, is spoken
rather timidly, without the full power of the lungs; and there is
something almost dishonest about it. For mankind still keeps to its
_memento mori_, and shows it by the universal need for history;
science may flap its wings as it will, it has never been able to gain
the free air. A deep feeling of hopelessness has remained, and taken
the historical colouring that has now darkened and depressed all
higher education. A religion that, of all the hours of man's life,
thinks the last the most important, that has prophesied the end of
earthly life and condemned all creatures to live in the fifth act of
a tragedy, may call forth the subtlest and noblest powers of man, but
it is an enemy to all new planting, to all bold attempts or free
aspirations. It opposes all flight into the unknown, because it has
no life or hope there itself. It only lets the new bud press forth on
sufferance, to blight it in its own good time: "it might lead life
astray and give it a false value." What the Florentines did under the
influence of Savonarola's exhortations, when they made the famous
holocaust of pictures, manuscripts, masks and mirrors, Christianity
would like to do with every culture that allured to further effort
and bore that _memento vivere_ on its standard. And if it cannot take
the direct way--the way of main force--it gains its end all the same
by allying itself with historical culture, though generally without
its connivance; and speaking through its mouth, turns away every
fresh birth with a shrug of its shoulders, and makes us feel all the
more that we are late-comers and Epigoni, that we are, in a word,
born with gray hair. The deep and serious contemplation of the
unworthiness of all past action, of the world ripe for judgment, has
been whittled down to the sceptical consciousness that it is anyhow a
good thing to know all that has happened, as it is too late to do
anything better. The historical sense makes its servants passive and
retrospective. Only in moments of forgetfulness, when that sense is
dormant, does the man who is sick of the historical fever ever act;
though he only analyses his deed again after it is over (which
prevents it from having any further consequences), and finally puts
it on the dissecting table for the purposes of history. In this sense
we are still living in the Middle Ages, and history is still a
disguised theology; just as the reverence with which the unlearned
layman looks on the learned class is inherited through the clergy.
What men gave formerly to the Church they give now, though in smaller
measure, to science. But the fact of giving at all is the work of the
Church, not of the modern spirit, which among its other good
qualities has something of the miser in it, and is a bad hand at the
excellent virtue of liberality.

These words may not be very acceptable, any more than my derivation
of the excess of history from the mediæval _memento mori_ and the
hopelessness that Christianity bears in its heart towards all future
ages of earthly existence. But you should always try to replace my
hesitating explanations by a better one. For the origin of historical
culture, and of its absolutely radical antagonism to the spirit of a
new time and a "modern consciousness," must itself be known by a
historical process. History must solve the problem of history,
science must turn its sting against itself. This threefold "must" is
the imperative of the "new spirit," if it is really to contain
something new, powerful, vital and original. Or is it true that we
Germans--to leave the Romance nations out of account--must always be
mere "followers" in all the higher reaches of culture, because that
is all we _can_ be? The words of Wilhelm Wackernagel are well worth
pondering: "We Germans are a nation of 'followers,' and with all our
higher science and even our faith, are merely the successors of the
ancient world. Even those who are opposed to it are continually
breathing the immortal spirit of classical culture with that of
Christianity: and if any one could separate these two elements from
the living air surrounding the soul of man, there would not be much
remaining for a spiritual life to exist on." Even if we would rest
content with our vocation to follow antiquity, even if we decided to
take it in an earnest and strenuous spirit and to show our high
prerogative in our earnestness,--we should yet be compelled to ask
whether it were our eternal destiny to be pupils of a fading
antiquity. We might be allowed at some time to put our aim higher and
further above us. And after congratulating ourselves on having
brought that secondary spirit of Alexandrian culture in us to such
marvellous productiveness--through our "universal history"--we might
go on to place before us, as our noblest prize, the still higher task
of striving beyond and above this Alexandrian world; and bravely find
our prototypes in the ancient Greek world, where all was great,
natural and human. But it is just _there_ that we find the reality of
a true unhistorical culture--and in spite of that, or perhaps because
of it, an unspeakably rich and vital culture. Were we Germans nothing
but followers, we could not be anything greater or prouder than the
lineal inheritors and followers of such a culture.

This however must be added. The thought of being Epigoni, that is
often a torture, can yet create a spring of hope for the future, to
the individual as well as the people: so far, that is, as we can
regard ourselves as the heirs and followers of the marvellous
classical power, and see therein both our honour and our spur. But
not as the late and bitter fruit of a powerful stock, giving that
stock a further spell of cold life, as antiquaries and grave-diggers.
Such late-comers live truly an ironical existence. Annihilation
follows their halting walk on tiptoe through life. They shudder
before it in the midst of their rejoicing over the past. They are
living memories, and their own memories have no meaning; for there
are none to inherit them. And thus they are wrapped in the melancholy
thought that their life is an injustice, which no future life can set
right again.

Suppose that these antiquaries, these late arrivals, were to change
their painful ironic modesty for a certain shamelessness. Suppose we
heard them saying, aloud, "The race is at its zenith, for it has
manifested itself consciously for the first time." We should have a
comedy, in which the dark meaning of a certain very celebrated
philosophy would unroll itself for the benefit of German culture. I
believe there has been no dangerous turning-point in the progress of
German culture in this century that has not been made more dangerous
by the enormous and still living influence of this Hegelian
philosophy. The belief that one is a late-comer in the world is,
anyhow, harmful and degrading: but it must appear frightful and
devastating when it raises our late-comer to godhead, by a neat turn
of the wheel, as the true meaning and object of all past creation,
and his conscious misery is set up as the perfection of the world's
history. Such a point of view has accustomed the Germans to talk of a
"world-process," and justify their own time as its necessary result.
And it has put history in the place of the other spiritual powers,
art and religion, as the one sovereign; inasmuch as it is the "Idea
realising itself," the "Dialectic of the spirit of the nations," and
the "tribunal of the world."

History understood in this Hegelian way has been contemptuously
called God's sojourn upon earth,--though the God was first created by
the history. He, at any rate, became transparent and intelligible
inside Hegelian skulls, and has risen through all the dialectically
possible steps in his being up to the manifestation of the Self: so
that for Hegel the highest and final stage of the world-process came
together in his own Berlin existence. He ought to have said that
everything after him was merely to be regarded as the musical coda of
the great historical rondo,--or rather, as simply superfluous. He has
not said it; and thus he has implanted in a generation leavened
throughout by him the worship of the "power of history," that
practically turns every moment into a sheer gaping at success, into
an idolatry of the actual: for which we have now discovered the
characteristic phrase "to adapt ourselves to circumstances." But the
man who has once learnt to crook the knee and bow the head before the
power of history, nods "yes" at last, like a Chinese doll, to every
power, whether it be a government or a public opinion or a numerical
majority; and his limbs move correctly as the power pulls the string.
If each success have come by a "rational necessity," and every event
show the victory of logic or the "Idea," then--down on your knees
quickly, and let every step in the ladder of success have its
reverence! There are no more living mythologies, you say? Religions
are at their last gasp? Look at the religion of the power of history,
and the priests of the mythology of Ideas, with their scarred knees!
Do not all the virtues follow in the train of the new faith? And
shall we not call it unselfishness, when the historical man lets
himself be turned into an "objective" mirror of all that is? Is it
not magnanimity to renounce all power in heaven and earth in order to
adore the mere fact of power? Is it not justice, always to hold the
balance of forces in your hands and observe which is the stronger and
heavier? And what a school of politeness is such a contemplation of
the past! To take everything objectively, to be angry at nothing, to
love nothing, to understand everything--makes one gentle and pliable.
Even if a man brought up in this school will show himself openly
offended, one is just as pleased, knowing it is only meant in the
artistic sense of _ira et studium_, though it is really _sine ira et
studio_.

What old-fashioned thoughts I have on such a combination of virtue
and mythology! But they must out, however one may laugh at them. I
would even say that history always teaches--"it was once," and
morality--"it ought not to be, or have been." So history becomes a
compendium of actual immorality. But how wrong would one be to regard
history as the judge of this actual immorality! Morality is offended
by the fact that a Raphael had to die at thirty-six; such a being
ought not to die. If you came to the help of history, as the
apologists of the actual, you would say: "he had spoken everything
that was in him to speak, a longer life would only have enabled him
to create a similar beauty, and not a new beauty," and so on. Thus
you become an _advocatus diaboli_ by setting up the success, the
fact, as your idol: whereas the fact is always dull, at all times
more like calf than a god. Your apologies for history are helped by
ignorance: for it is only because you do not know what a _natura
naturans_ like Raphael is, that you are not on fire when you think it
existed once and can never exist again. Some one has lately tried to
tell us that Goethe had out-lived himself with his eighty-two years:
and yet I would gladly take two of Goethe's "out-lived" years in
exchange for whole cartloads of fresh modern lifetimes, to have
another set of such conversations as those with Eckermann, and be
preserved from all the "modern" talk of these esquires of the moment.
How few living men have a right to live, as against those mighty
dead! That the many live and those few live no longer, is simply a
brutal truth, that is, a piece of unalterable folly, a blank wall of
"it was once so" against the moral judgment "it ought not to have
been." Yes, against the moral judgment! For you may speak of what
virtue you will, of justice, courage, magnanimity, of wisdom and
human compassion,--you will find the virtuous man will always rise
against the blind force of facts, the tyranny of the actual, and
submit himself to laws that are not the fickle laws of history. He
ever swims against the waves of history, either by fighting his
passions, as the nearest brute facts of his existence, or by training
himself to honesty amid the glittering nets spun round him by
falsehood. Were history nothing more than the "all-embracing system
of passion and error," man would have to read it as Goethe wished
Werther to be read;--just as if it called to him, "Be a man and
follow me not!" But fortunately history also keeps alive for us the
memory of the great "fighters against history," that is, against the
blind power of the actual; it puts itself in the pillory just by
glorifying the true historical nature in men who troubled themselves
very little about the "thus it is," in order that they might follow a
"thus it must be" with greater joy and greater pride. Not to drag
their generation to the grave, but to found a new one--that is the
motive that ever drives them onward; and even if they are born late,
there is a way of living by which they can forget it--and future
generations will know them only as the first-comers.


IX.

Is perhaps our time such a "first-comer"? Its historical sense is so
strong, and has such universal and boundless expression, that future
times will commend it, if only for this, as a first-comer--if there
be any future time, in the sense of future culture. But here comes a
grave doubt. Close to the modern man's pride there stands his irony
about himself, his consciousness that he must live in a historical,
or twilit, atmosphere, the fear that he can retain none of his
youthful hopes and powers. Here and there one goes further into
cynicism, and justifies the course of history, nay, the whole
evolution of the world, as simply leading up to the modern man,
according to the cynical canon:--"what you see now had to come, man
had to be thus and not otherwise, no one can stand against this
necessity." He who cannot rest in a state of irony flies for refuge
to the cynicism. The last decade makes him a present of one of its
most beautiful inventions, a full and well-rounded phrase for this
cynicism: he calls his way of living thoughtlessly and after the
fashion of his time, "the full surrender of his personality to the
world-process." The personality and the world-process! The
world-process and the personality of the earthworm! If only one did
not eternally hear the word "world, world, world," that hyperbole of
all hyperboles; when we should only speak, in a decent manner, of
"man, man, man"! Heirs of the Greeks and Romans, of Christianity? All
that seems nothing to the cynics. But "heirs of the world-process";
the final target of the world-process; the meaning and solution of
all riddles of the universe, the ripest fruit on the tree of
knowledge!--that is what I call a right noble thought: by this token
are the firstlings of every time to be known, although they may have
arrived last. The historical imagination has never flown so far, even
in a dream; for now the history of man is merely the continuation of
that of animals and plants: the universal historian finds traces of
himself even in the utter depths of the sea, in the living slime. He
stands astounded in face of the enormous way that man has run, and
his gaze quivers before the mightier wonder, the modern man who can
see all this way! He stands proudly on the pyramid of the
world-process: and while he lays the final stone of his knowledge, he
seems to cry aloud to listening Nature: "We are at the top, we are
the top, we are the completion of Nature!"

O thou too proud European of the nineteenth century, art thou not
mad? Thy knowledge does not complete Nature, it only kills thine own
nature! Measure the height of what thou knowest by the depths of thy
power to _do_. Thou climbest the sunbeams of knowledge up towards
heaven--but also down to Chaos. Thy manner of going is fatal to thee;
the ground slips from under thy feet into the unknown; thy life has
no other stay, but only spider's webs that every new stroke of thy
knowledge tears asunder.--But not another serious word about this,
for there is a lighter side to it all.

The moralist, the artist, the saint and the statesman may well be
troubled, when they see that all foundations are breaking up in mad
unconscious ruin, and resolving themselves into the ever flowing
stream of becoming; that all creation is being tirelessly spun into
webs of history by the modern man, the great spider in the mesh of
the world-net. We ourselves may be glad for once in a way that we see
it all in the shining magic mirror of a philosophical parodist, in
whose brain the time has come to an ironical consciousness of itself,
to a point even of wickedness, in Goethe's phrase. Hegel once said,
"when the spirit makes a fresh start, we philosophers are at hand."
Our time did make a fresh start--into irony, and lo! Edward von
Hartmann was at hand, with his famous Philosophy of the
Unconscious--or, more plainly, his philosophy of unconscious irony.
We have seldom read a more jovial production, a greater philosophical
joke than Hartmann's book. Any one whom it does not fully enlighten
about "becoming," who is not swept and garnished throughout by it, is
ready to become a monument of the past himself. The beginning and end
of the world-process, from the first throb of consciousness to its
final leap into nothingness, with the task of our generation settled
for it;--all drawn from that clever fount of inspiration, the
Unconscious, and glittering in Apocalyptic light, imitating an honest
seriousness to the life, as if it were a serious philosophy and not a
huge joke,--such a system shows its creator to be one of the first
philosophical parodists of all time. Let us then sacrifice on his
altar, and offer the inventor of a true universal medicine a lock of
hair, in Schleiermacher's phrase. For what medicine would be more
salutary to combat the excess of historical culture than Hartmann's
parody of the world's history?

If we wished to express in the fewest words what Hartmann really has
to tell us from his mephitic tripod of unconscious irony, it would be
something like this: our time could only remain as it is, if men
should become thoroughly sick of this existence. And I fervently
believe he is right. The frightful petrifaction of the time, the
restless rattle of the ghostly bones, held naïvely up to us by David
Strauss as the most beautiful fact of all--is justified by Hartmann
not only from the past, _ex causis efficientibus_, but also from the
future, _ex causa finali_. The rogue let light stream over our time
from the last day, and saw that it was very good,--for him, that is,
who wishes to feel the indigestibility of life at its full strength,
and for whom the last day cannot come quickly enough. True, Hartmann
calls the old age of life that mankind is approaching the "old age of
man": but that is the blessed state, according to him, where there is
only a successful mediocrity; where art is the "evening's amusement
of the Berlin financier," and "the time has no more need for
geniuses, either because it would be casting pearls before swine, or
because the time has advanced beyond the stage where the geniuses are
found, to one more important," to that stage of social evolution, in
fact, in which every worker "leads a comfortable existence, with
hours of work that leave him sufficient leisure to cultivate his
intellect." Rogue of rogues, you say well what is the aspiration of
present-day mankind: but you know too what a spectre of disgust will
arise at the end of this old age of mankind, as the result of the
intellectual culture of stolid mediocrity. It is very pitiful to see,
but it will be still more pitiful yet. "Antichrist is visibly
extending his arms:" yet it _must be so_, for after all we are on the
right road--of disgust at all existence. "Forward then, boldly, with
the world-process, as workers in the vineyard of the Lord, for it is
the process alone that can lead to redemption!"

The vineyard of the Lord! The process! To redemption! Who does not
see and hear in this how historical culture, that only knows the word
"becoming," parodies itself on purpose and says the most
irresponsible things about itself through its grotesque mask? For
what does the rogue mean by this cry to the workers in the vineyard?
By what "work" are they to strive boldly forward? Or, to ask another
question:--what further has the historically educated fanatic of the
world-process to do,--swimming and drowning as he is in the sea of
becoming,--that he may at last gather in that vintage of disgust, the
precious grape of the vineyard? He has nothing to do but to live on
as he has lived, love what he has loved, hate what he has hated, and
read the newspapers he has always read. The only sin is for him to
live otherwise than he has lived. We are told how he has lived, with
monumental clearness, by that famous page with its large typed
sentences, on which the whole rabble of our modern cultured folk have
thrown themselves in blind ecstasy, because they believe they read
their own justification there, haloed with an Apocalyptic light. For
the unconscious parodist has demanded of every one of them, "the full
surrender of his personality to the world-process, for the sake of
his end, the redemption of the world": or still more clearly,--"the
assertion of the will to live is proclaimed to be the first step on
the right road: for it is only in the full surrender to life and its
sorrow, and not in the cowardice of personal renunciation and
retreat, that anything can be done for the world-process.... The
striving for the denial of the individual will is as foolish as it is
useless, more foolish even than suicide.... The thoughtful reader
will understand without further explanation how a practical
philosophy can be erected on these principles, and that such a
philosophy cannot endure any disunion, but only the fullest
reconciliation with life."

The thoughtful reader will understand! Then one really could
misunderstand Hartmann! And what a splendid joke it is, that he
should be misunderstood! Why should the Germans of to-day be
particularly subtle? A valiant Englishman looks in vain for "delicacy
of perception" and dares to say that "in the German mind there does
seem to be something splay, something blunt-edged, unhandy and
infelicitous." Could the great German parodist contradict this?
According to him, we are approaching "that ideal condition in which
the human race makes its history with full consciousness": but we are
obviously far from the perhaps more ideal condition, in which mankind
can read Hartmann's book with full consciousness. If we once reach
it, the word "world-process" will never pass any man's lips again
without a smile. For he will remember the time when people listened
to the mock gospel of Hartmann, sucked it in, attacked it, reverenced
it, extended it and canonised it with all the honesty of that "German
mind," with "the uncanny seriousness of an owl," as Goethe has it.
But the world must go forward, the ideal condition cannot be won by
dreaming, it must be fought and wrestled for, and the way to
redemption lies only through joyousness, the way to redemption from
that dull, owlish seriousness. The time will come when we shall
wisely keep away from all constructions of the world-process, or even
of the history of man; a time when we shall no more look at masses
but at individuals, who form a sort of bridge over the wan stream of
becoming. They may not perhaps continue a process, but they live out
of time, as contemporaries: and thanks to history that permits such a
company, they live as the Republic of geniuses of which Schopenhauer
speaks. One giant calls to the other across the waste spaces of time,
and the high spirit-talk goes on, undisturbed by the wanton noisy
dwarfs who creep among them. The task of history is to be the
mediator between these, and even to give the motive and power to
produce the great man. The aim of mankind can lie ultimately only in
its highest examples.

Our low comedian has his word on this too, with his wonderful
dialectic, which is just as genuine as its admirers are admirable.
"The idea of evolution cannot stand with our giving the world-process
an endless duration in the past, for thus every conceivable evolution
must have taken place, which is not the case (O rogue!); and so we
cannot allow the process an endless duration in the future. Both
would raise the conception of evolution to a mere ideal (And again
rogue!), and would make the world-process like the sieve of the
Danaides. The complete victory of the logical over the illogical (O
thou complete rogue!) must coincide with the last day, the end in
time of the world-process." No, thou clear, scornful spirit, so long
as the illogical rules as it does to-day,--so long, for example, as
the world-process can be spoken of as thou speakest of it, amid such
deep-throated assent,--the last day is yet far off. For it is still
too joyful on this earth, many an illusion still blooms here--like
the illusion of thy contemporaries about thee. We are not yet ripe to
be hurled into thy nothingness: for we believe that we shall have a
still more splendid time, when men once begin to understand thee,
thou misunderstood, unconscious one! But if, in spite of that,
disgust shall come throned in power, as thou hast prophesied to thy
readers; if thy portrayal of the present and the future shall prove
to be right,--and no one has despised them with such loathing as
thou,--I am ready then to cry with the majority in the form
prescribed by thee, that next Saturday evening, punctually at twelve
o'clock, thy world shall fall to pieces. And our decree shall
conclude thus--from to-morrow time shall not exist, and the _Times_
shall no more be published. Perhaps it will be in vain, and our
decree of no avail: at any rate we have still time for a fine
experiment. Take a balance and put Hartmann's "Unconscious" in one of
the scales, and his "World-process" in the other. There are some who
believe they weigh equally; for in each scale there is an evil
word--and a good joke.

When they are once understood, no one will take Hartmann's words on
the world-process as anything but a joke. It is, as a fact, high time
to move forward with the whole battalion of satire and malice against
the excesses of the "historical sense," the wanton love of the
world-process at the expense of life and existence, the blind
confusion of all perspective. And it will be to the credit of the
philosopher of the Unconscious that he has been the first to see the
humour of the world-process, and to succeed in making others see it
still more strongly by the extraordinary seriousness of his
presentation. The existence of the "world" and "humanity" need not
trouble us for some time, except to provide us with a good joke: for
the presumption of the small earthworm is the most uproariously comic
thing on the face of the earth. Ask thyself to what end thou art
here, as an individual; and if no one can tell thee, try then to
justify the meaning of thy existence _a posteriori_, by putting
before thyself a high and noble end. Perish on that rock! I know no
better aim for life than to be broken on something great and
impossible, _animæ magnæ prodigus_. But if we have the doctrines of
the finality of "becoming," of the flux of all ideas, types, and
species, of the lack of all radical difference between man and beast
(a true but fatal idea as I think),--if we have these thrust on the
people in the usual mad way for another generation, no one need be
surprised if that people drown on its little miserable shoals of
egoism, and petrify in its self-seeking. At first it will fall
asunder and cease to be a people. In its place perhaps individualist
systems, secret societies for the extermination of non-members, and
similar utilitarian creations, will appear on the theatre of the
future. Are we to continue to work for these creations and write
history from the standpoint of the _masses_; to look for laws in it,
to be deduced from the needs of the masses, the laws of motion of the
lowest loam and clay strata of society? The masses seem to be worth
notice in three aspects only: first as the copies of great men,
printed on bad paper from worn-out plates, next as a contrast to the
great men, and lastly as their tools: for the rest, let the devil and
statistics fly away with them! How could statistics prove that there
are laws in history? Laws? Yes, they may prove how common and
abominably uniform the masses are: and should we call the effects of
leaden folly, imitation, love and hunger--laws? We may admit it: but
we are sure of this too--that so far as there are laws in history,
the laws are of no value and the history of no value either. And
least valuable of all is that kind of history which takes the great
popular movements as the most important events of the past, and
regards the great men only as their clearest expression, the visible
bubbles on the stream. Thus the masses have to produce the great man,
chaos to bring forth order; and finally all the hymns are naturally
sung to the teeming chaos. Everything is called "great" that has
moved the masses for some long time, and becomes, as they say, a
"historical power." But is not this really an intentional confusion
of quantity and quality? When the brutish mob have found some idea, a
religious idea for example, which satisfies them, when they have
defended it through thick and thin for centuries then, and then only,
will they discover its inventor to have been a great man. The highest
and noblest does not affect the masses at all. The historical
consequences of Christianity, its "historical power," toughness and
persistence prove nothing, fortunately, as to its founder's
greatness, They would have been a witness against him. For between
him and the historical success of Christianity lies a dark heavy
weight of passion and error, lust of power and honour, and the
crushing force of the Roman Empire. From this, Christianity had its
earthly taste, and its earthly foundations too, that made its
continuance in this world possible. Greatness should not depend on
success; Demosthenes is great without it. The purest and noblest
adherents of Christianity have always doubted and hindered, rather
than helped, its effect in the world, its so-called "historical
power"; for they were accustomed to stand outside the "world," and
cared little for the "process of the Christian Idea." Hence they have
generally remained unknown to history, and their very names are lost.
In Christian terms the devil is the prince of the world, and the lord
of progress and consequence: he is the power behind all "historical
power," and so will it remain, however ill it may sound to-day in
ears that are accustomed to canonise such power and consequence. The
world has become skilled at giving new names to things and even
baptizing the devil. It is truly an hour of great danger. Men seem to
be near the discovery that the egoism of individuals, groups or
masses has been at all times the lever of the "historical movements":
and yet they are in no way disturbed by the discovery, but proclaim
that "egoism shall be our god." With this new faith in their hearts,
they begin quite intentionally to build future history on egoism:
though it must be a clever egoism, one that allows of some
limitation, that it may stand firmer; one that studies history for
the purpose of recognising the foolish kind of egoism. Their study
has taught them that the state has a special mission in all future
egoistic systems: it will be the patron of all the clever egoisms, to
protect them with all the power of its military and police against
the dangerous outbreaks of the other kind. There is the same idea in
introducing history--natural as well as human history--among the
labouring classes, whose folly makes them dangerous. For men know
well that a grain of historical culture is able to break down the
rough, blind instincts and desires, or to turn them to the service of
a clever egoism. In fact they are beginning to think, with Edward von
Hartmann, of "fixing themselves with an eye to the future in their
earthly home, and making themselves comfortable there." Hartmann
calls this life the "manhood of humanity" with an ironical reference
to what is now called "manhood";--as if only our sober models of
selfishness were embraced by it; just as he prophesies an age of
graybeards following on this stage,--obviously another ironical
glance at our ancient time-servers. For he speaks of the ripe
discretion with which "they view all the stormy passions of their
past life and understand the vanity of the ends they seem to have
striven for." No, a manhood of crafty and historically cultured
egoism corresponds to an old age that hangs to life with no dignity
but a horrible tenacity, where the

                           "last scene of all
  That ends this strange eventful history,
  Is second childishness and mere oblivion,
  Sans teeth, sans eyes, sans taste, sans everything."

Whether the dangers of our life and culture come from these dreary,
toothless old men, or from the so-called "men" of Hartmann, we have
the right to defend our youth with tooth and claw against both of
them, and never tire of saving the future from these false prophets.
But in this battle we shall discover an unpleasant truth--that men
intentionally help, and encourage, and use, the worst aberrations of
the historical sense from which the present time suffers.

They use it, however, against youth, in order to transform it into
that ripe "egoism of manhood" they so long for: they use it to
overcome the natural reluctance of the young by its magical
splendour, which unmans while it enlightens them. Yes, we know only
too well the kind of ascendency history can gain; how it can uproot
the strongest instincts of youth, passion, courage, unselfishness and
love; can cool its feeling for justice, can crush or repress its
desire for a slow ripening by the contrary desire to be soon
productive, ready and useful; and cast a sick doubt over all honesty
and downrightness of feeling. It can even cozen youth of its fairest
privilege, the power of planting a great thought with the fullest
confidence, and letting it grow of itself to a still greater thought.
An excess of history can do all that, as we have seen, by no longer
allowing a man to feel and act _unhistorically_: for history is
continually shifting his horizon and removing the atmosphere
surrounding him. From an infinite horizon he withdraws into himself,
back into the small egoistic circle, where he must become dry and
withered: he may possibly attain to cleverness, but never to wisdom.
He lets himself be talked over, is always calculating and parleying
with facts. He is never enthusiastic, but blinks his eyes, and
understands how to look for his own profit or his party's in the
profit or loss of somebody else. He unlearns all his useless modesty,
and turns little by little into the "man" or the "graybeard" of
Hartmann. And that is what they _want_ him to be: that is the meaning
of the present cynical demand for the "full surrender of the
personality to the world-process"--for the sake of his end, the
redemption of the world, as the rogue E. von Hartmann tells us.
Though redemption can scarcely be the conscious aim of these people:
the world were better redeemed by being redeemed from these "men" and
"graybeards." For then would come the reign of youth.


X.

And in this kingdom of youth I can cry Land! Land! Enough, and more
than enough, of the wild voyage over dark strange seas, of eternal
search and eternal disappointment! The coast is at last in sight.
Whatever it be, we must land there, and the worst haven is better
than tossing again in the hopeless waves of an infinite scepticism.
Let us hold fast by the land: we shall find the good harbours later
and make the voyage easier for those who come after us.

The voyage was dangerous and exciting. How far are we even now from
that quiet state of contemplation with which we first saw our ship
launched! In tracking out the dangers of history, we have found
ourselves especially exposed to them. We carry on us the marks of
that sorrow which an excess of history brings in its train to the men
of the modern time. And this present treatise, as I will not attempt
to deny, shows the modern note of a weak personality in the
intemperateness of its criticism, the unripeness of its humanity, in
the too frequent transitions from irony to cynicism, from arrogance
to scepticism. And yet I trust in the inspiring power that directs my
vessel instead of genius; I trust in _youth_, that has brought me on
the right road in forcing from me a protest against the modern
historical education, and a demand that the man must learn to live,
above all, and only use history in the service of the life that he
has learned to live. He must be young to understand this protest; and
considering the premature grayness of our present youth, he can
scarcely be young enough if he would understand its reason as well.
An example will help me. In Germany, not more than a century ago, a
natural instinct for what is called "poetry" was awakened in some
young men. Are we to think that the generations who had lived before
that time had not spoken of the art, however really strange and
unnatural it may have been to them? We know the contrary; that they
had thought, written, and quarrelled about it with all their
might--in "words, words, words." Giving life to such words did not
prove the death of the word-makers; in a certain sense they are
living still. For if, as Gibbon says, nothing but time--though a long
time--is needed for a world to perish, so nothing but time--though
still more time--is needed for a false idea to be destroyed in
Germany, the "Land of Little-by-little." In any event, there are
perhaps a hundred men more now than there were a century ago who know
what poetry is: perhaps in another century there will be a hundred
more who have learned in the meantime what culture is, and that the
Germans have had as yet no culture, however proudly they may talk
about it. The general satisfaction of the Germans at their culture
will seem as foolish and incredible to such men as the once lauded
classicism of Gottsched, or the reputation of Ramler as the German
Pindar, seemed to us. They will perhaps think this "culture" to be
merely a kind of knowledge about culture, and a false and superficial
knowledge at that. False and superficial, because the Germans endured
the contradiction between life and knowledge, and did not see what
was characteristic in the culture of really educated peoples, that it
can only rise and bloom from life. But by the Germans it is worn like
a paper flower, or spread over like the icing on a cake; and so must
remain a useless lie for ever.

The education of youth in Germany starts from this false and
unfruitful idea of culture. Its aim, when faced squarely, is not to
form the liberally educated man, but the professor, the man of
science, who wants to be able to make use of his science as soon as
possible, and stands on one side in order to see life clearly. The
result, even from a ruthlessly practical point of view, is the
historically and æsthetically trained Philistine, the babbler of old
saws and new wisdom on Church, State and Art, the sensorium that
receives a thousand impressions, the insatiable belly that yet knows
not what true hunger and thirst is. An education with such an aim and
result is against nature. But only he who is not quite drowned in it
can feel that; only youth can feel it, because it still has the
instinct of nature, that is the first to be broken by that education.
But he who will break through that education in his turn, must come
to the help of youth when called upon; must let the clear light of
understanding shine on its unconscious striving, and bring it to a
full, vocal consciousness. How is he to attain such a strange end?

Principally by destroying the superstition that this kind of
education is _necessary_. People think nothing but this troublesome
reality of ours is possible. Look through the literature of higher
education in school and college for the last ten years, and you will
be astonished--and pained--to find how much alike all the proposals
of reform have been; in spite of all the hesitations and violent
controversies surrounding them. You will see how blindly they have
all adopted the old idea of the "educated man" (in our sense) being
the necessary and reasonable basis of the system. The monotonous
canon runs thus: the young man must begin with a knowledge of
culture, not even with a knowledge of life, still less with life and
the living of it. This knowledge of culture is forced into the young
mind in the form of historical knowledge; which means that his head
is filled with an enormous mass of ideas, taken second-hand from past
times and peoples, not from immediate contact with life. He desires
to experience something for himself, and feel a close-knit, living
system of experiences growing within himself. But his desire is
drowned and dizzied in the sea of shams, as if it were possible to
sum up in a few years the highest and notablest experiences of
ancient times, and the greatest times too. It is the same mad method
that carries our young artists off to picture-galleries, instead of
the studio of a master, and above all the one studio of the only
master, Nature. As if one could discover by a hasty rush through
history the ideas and technique of past times, and their individual
outlook on life! For life itself is a kind of handicraft that must be
learned thoroughly and industriously, and diligently practised, if we
are not to have mere botchers and babblers as the issue of it all!

Plato thought it necessary for the first generation of his new
society (in the perfect state) to be brought up with the help of a
"mighty lie." The children were to be taught to believe that they had
all lain dreaming for a long time under the earth, where they had
been moulded and formed by the master-hand of Nature. It was
impossible to go against the past, and work against the work of gods!
And so it had to be an unbreakable law of nature, that he who is born
to be a philosopher has gold in his body, the fighter has only
silver, and the workman iron and bronze. As it is not possible to
blend these metals, according to Plato, so there could never be any
confusion between the classes: the belief in the _æterna veritas_ of
this arrangement was the basis of the new education and the new
state. So the modern German believes also in the _æterna veritas_ of
his education, of his kind of culture: and yet this belief will
fail--as the Platonic state would have failed--if the mighty German
lie be ever opposed by the truth, that the German has no culture
because he cannot build one on the basis of his education. He wishes
for the flower without the root or the stalk; and so he wishes in
vain. That is the simple truth, a rude and unpleasant truth, but yet
a mighty one.

But our first generation must be brought up in this "mighty truth,"
and must suffer from it too; for it must educate itself through it,
even against its own nature, to attain a new nature and manner of
life, which shall yet proceed from the old. So it might say to
itself, in the old Spanish phrase, "Defienda me Dios de my," God keep
me from myself, from the character, that is, which has been put into
me. It must taste that truth drop by drop, like a bitter, powerful
medicine. And every man in this generation must subdue himself to
pass the judgment on his own nature, which he might pass more easily
on his whole time:--"We are without instruction, nay, we are too
corrupt to live, to see and hear truly and simply, to understand what
is near and natural to us. We have not yet laid even the foundations
of culture, for we are not ourselves convinced that we have a sincere
life in us." We crumble and fall asunder, our whole being is divided,
half mechanically, into an inner and outer side; we are sown with
ideas as with dragon's teeth, and bring forth a new dragon-brood of
them; we suffer from the malady of words, and have no trust in any
feeling that is not stamped with its special word. And being such a
dead fabric of words and ideas, that yet has an uncanny movement in
it, I have still perhaps the right to say _cogito ergo sum_, though
not _vivo ergo cogito_. I am permitted the empty _esse_, not the full
green _vivere_. A primary feeling tells me that I am a thinking being
but not a living one, that I am no "animal," but at most a "cogital."
"Give me life, and I will soon make you a culture out of it"--will be
the cry of every man in this new generation, and they will all know
each other by this cry. But who will give them this life?

No god and no man will give it--only their own _youth_. Set this
free, and you will set life free as well. For it only lay concealed,
in a prison; it is not yet withered or dead--ask your own selves!

But it is sick, this life that is set free, and must be healed. It
suffers from many diseases, and not only from the memory of its
chains. It suffers from the malady which I have spoken of, the
_malady of history_. Excess of history has attacked the plastic power
of life, that no more understands how to use the past as a means of
strength and nourishment. It is a fearful disease, and yet, if youth
had not a natural gift for clear vision, no one would see that it is
a disease, and that a paradise of health has been lost. But the same
youth, with that same natural instinct of health, has guessed how the
paradise can be regained. It knows the magic herbs and simples for
the malady of history, and the excess of it. And what are they
called?

It is no marvel that they bear the names of poisons:--the antidotes
to history are the "unhistorical" and the "super-historical." With
these names we return to the beginning of our inquiry and draw near
to its final close.

By the word "unhistorical" I mean the power, the art of _forgetting_,
and of drawing a limited horizon round one's self. I call the power
"super-historical" which turns the eyes from the process of becoming
to that which gives existence an eternal and stable character, to art
and religion. Science--for it is science that makes us speak of
"poisons"--sees in these powers contrary powers: for it considers
only that view of things to be true and right, and therefore
scientific, which regards something as finished and historical, not
as continuing and eternal. Thus it lives in a deep antagonism towards
the powers that make for eternity--art and religion,--for it hates
the forgetfulness that is the death of knowledge, and tries to remove
all limitation of horizon and cast men into an infinite boundless
sea, whose waves are bright with the clear knowledge--of becoming!

If they could only live therein! Just as towns are shaken by an
avalanche and become desolate, and man builds his house there in fear
and for a season only; so life is broken in sunder and becomes weak
and spiritless, if the avalanche of ideas started by science take
from man the foundation of his rest and security, the belief in what
is stable and eternal. Must life dominate knowledge, or knowledge
life? Which of the two is the higher, and decisive power? There is no
room for doubt: life is the higher, and the dominating power, for the
knowledge that annihilated life would be itself annihilated too.
Knowledge presupposes life, and has the same interest in maintaining
it that every creature has in its own preservation. Science needs
very careful watching: there is a hygiene of life near the volumes of
science, and one of its sentences runs thus:--The unhistorical and
the super-historical are the natural antidotes against the
overpowering of life by history; they are the cures for the
historical disease. We who are sick of the disease may suffer a
little from the antidote. But this is no proof that the treatment we
have chosen is wrong.

And here I see the mission of the youth that forms the first
generation of fighters and dragon-slayers: it will bring a more
beautiful and blessed humanity and culture, but will have itself no
more than a glimpse of the promised land of happiness and wondrous
beauty. This youth will suffer both from the malady and its
antidotes: and yet it believes in strength and health and boasts a
nature closer to the great Nature than its forebears, the cultured
men and graybeards of the present. But its mission is to shake to
their foundations the present conceptions of "health" and "culture,"
and erect hatred and scorn in the place of this rococo mass of ideas.
And the clearest sign of its own strength and health is just the fact
that it can use no idea, no party-cry from the present-day mint of
words and ideas to symbolise its own existence: but only claims
conviction from the power in it that acts and fights, breaks up and
destroys; and from an ever heightened feeling of life when the hour
strikes. You may deny this youth any culture--but how would youth
count that a reproach? You may speak of its rawness and
intemperateness--but it is not yet old and wise enough to be
acquiescent. It need not pretend to a ready-made culture at all; but
enjoys all the rights--and the consolations--of youth, especially the
right of brave unthinking honesty and the consolation of an inspiring
hope.

I know that such hopeful beings understand all these truisms from
within, and can translate them into a doctrine for their own use,
through their personal experience. To the others there will appear,
in the meantime, nothing but a row of covered dishes, that may
perhaps seem empty: until they see one day with astonished eyes that
the dishes are full, and that all ideas and impulses and passions are
massed together in these truisms that cannot lie covered for long. I
leave those doubting ones to time, that brings all things to light;
and turn at last to that great company of hope, to tell them the way
and the course of their salvation, their rescue from the disease of
history, and their own history as well, in a parable; whereby they
may again become healthy enough to study history anew, and under the
guidance of life make use of the past in that threefold
way--monumental, antiquarian, or critical. At first they will be more
ignorant than the "educated men" of the present: for they will have
unlearnt much and have lost any desire even to discover what those
educated men especially wish to know: in fact, their chief mark from
the educated point of view will be just their want of science; their
indifference and inaccessibility to all the good and famous things.
But at the end of the cure, they are men again and have ceased to be
mere shadows of humanity. That is something; there is yet hope, and
do not ye who hope laugh in your hearts?

How can we reach that end? you will ask. The Delphian god cries his
oracle to you at the beginning of your wanderings, "Know thyself." It
is a hard saying: for that god "tells nothing and conceals nothing
but merely points the way," as Heraclitus said. But whither does he
point?

In certain epochs the Greeks were in a similar danger of being
overwhelmed by what was past and foreign, and perishing on the rock
of "history." They never lived proud and untouched. Their "culture"
was for a long time a chaos of foreign forms and ideas,--Semitic,
Babylonian, Lydian and Egyptian,--and their religion a battle of all
the gods of the East; just as German culture and religion is at
present a death-struggle of all foreign nations and bygone times. And
yet, Hellenic culture was no mere mechanical unity, thanks to that
Delphic oracle. The Greeks gradually learned to organise the chaos,
by taking Apollo's advice and thinking back to themselves, to their
own true necessities, and letting all the sham necessities go. Thus
they again came into possession of themselves, and did not remain
long the Epigoni of the whole East, burdened with their inheritance.
After that hard fight, they increased and enriched the treasure they
had inherited by their obedience to the oracle, and they became the
ancestors and models for all the cultured nations of the future.

This is a parable for each one of us: he must organise the chaos in
himself by "thinking himself back" to his true needs. He will want
all his honesty, all the sturdiness and sincerity in his character to
help him to revolt against second-hand thought, second-hand learning,
second-hand action. And he will begin then to understand that culture
can be something more than a "decoration of life"--a concealment and
disfiguring of it, in other words; for all adornment hides what is
adorned. And thus the Greek idea, as against the Roman, will be
discovered in him, the idea of culture as a new and finer nature,
without distinction of inner and outer, without convention or
disguise, as a unity of thought and will, life and appearance. He
will learn too, from his own experience, that it was by a greater
force of moral character that the Greeks were victorious, and that
everything which makes for sincerity is a further step towards true
culture, however this sincerity may harm the ideals of education that
are reverenced at the time, or even have power to shatter a whole
system of merely decorative culture.




SCHOPENHAUER AS EDUCATOR.


I.

When the traveller, who had seen many countries and nations and
continents, was asked what common attribute he had found everywhere
existing among men, he answered, "They have a tendency to sloth."
Many may think that the fuller truth would have been, "They are all
timid." They hide themselves behind "manners" and "opinions." At
bottom every man knows well enough that he is a unique being, only
once on this earth; and by no extraordinary chance will such a
marvellously picturesque piece of diversity in unity as he is, ever
be put together a second time. He knows this, but hides it like an
evil conscience;--and why? From fear of his neighbour, who looks for
the latest conventionalities in him, and is wrapped up in them
himself. But what is it that forces the man to fear his neighbour, to
think and act with his herd, and not seek his own joy? Shyness
perhaps, in a few rare cases, but in the majority it is idleness, the
"taking things easily," in a word the "tendency to sloth," of which
the traveller spoke. He was right; men are more slothful than timid,
and their greatest fear is of the burdens that an uncompromising
honesty and nakedness of speech and action would lay on them. It is
only the artists who hate this lazy wandering in borrowed manners and
ill-fitting opinions, and discover the secret of the evil conscience,
the truth that each human being is a unique marvel. They show us, how
in every little movement of his muscles the man is an individual
self, and further--as an analytical deduction from his individuality--
a beautiful and interesting object, a new and incredible phenomenon
(as is every work of nature), that can never become tedious. If the
great thinker despise mankind, it is for their laziness; they seem
mere indifferent bits of pottery, not worth any commerce or
improvement. The man who will not belong to the general mass, has
only to stop "taking himself easily"; to follow his conscience, which
cries out to him, "Be thyself! all that thou doest and thinkest and
desirest, is not--thyself!"

Every youthful soul hears this cry day and night, and quivers to hear
it: for she divines the sum of happiness that has been from eternity
destined for her, if she think of her true deliverance; and towards
this happiness she can in no wise be helped, so long as she lies in
the chains of Opinion and of Fear. And how comfortless and unmeaning
may life become without this deliverance! There is no more desolate
or Ishmaelitish creature in nature than the man who has broken away
from his true genius, and does nothing but peer aimlessly about him.
There is no reason to attack such a man at all, for he is a mere husk
without a kernel, a painted cloth, tattered and sagging, a scarecrow
ghost, that can rouse no fear, and certainly no pity. And though one
be right in saying of a sluggard that he is "killing time," yet in
respect of an age that rests its salvation on public opinion,--that
is, on private laziness,--one must be quite determined that such a
time shall be "killed," once and for all: I mean that it shall be
blotted from life's true History of Liberty. Later generations will
be greatly disgusted, when they come to treat the movements of a
period in which no living men ruled, but shadow-men on the screen of
public opinion; and to some far posterity our age may well be the
darkest chapter of history, the most unknown because the least human.
I have walked through the new streets of our cities, and thought how
of all the dreadful houses that these gentlemen with their public
opinion have built for themselves, not a stone will remain in a
hundred years, and that the opinions of these busy masons may well
have fallen with them. But how full of hope should they all be who
feel that they are no citizens of this age! If they were, they would
have to help on the work of "killing their time," and of perishing
with it,--when they wish rather to quicken the time to life, and in
that life themselves to _live_.

But even if the future leave us nothing to hope for, the wonderful
fact of our existing at this present moment of time gives us the
greatest encouragement to live after our own rule and measure; so
inexplicable is it, that we should be living just to-day, though
there have been an infinity of time wherein we might have arisen;
that we own nothing but a span's length of it, this "to-day," and
must show in it wherefore and whereunto we have arisen. We have to
answer for our existence to ourselves; and will therefore be our own
true pilots, and not admit that our being resembles a blind fortuity.
One must take a rather impudent and reckless way with the riddle;
especially as the key is apt to be lost, however things turn out. Why
cling to your bit of earth, or your little business, or listen to
what your neighbour says? It is so provincial to bind oneself to
views which are no longer binding a couple of hundred miles away.
East and West are signs that somebody chalks up in front of us to
fool such cowards as we are. "I will make the attempt to gain
freedom," says the youthful soul; and will be hindered, just because
two nations happen to hate each other and go to war, or because there
is a sea between two parts of the earth, or a religion is taught in
the vicinity, which did not exist two thousand years ago. "And this
is not--thyself," the soul says. "No one can build thee the bridge,
over which thou must cross the river of life, save thyself alone.
There are paths and bridges and demi-gods without number, that will
gladly carry thee over, but only at the price of thine own self: thy
self wouldst thou have to give in pawn, and then lose it. There is in
the world one road whereon none may go, except thou: ask not whither
it lead, but go forward. Who was it that spake that true word--'A man
has never risen higher than when he knoweth not whither his road may
yet lead him'?"

But how can we "find ourselves" again, and how can man "know
himself"? He is a thing obscure and veiled: if the hare have seven
skins, man can cast from him seventy times seven, and yet will not be
able to say "Here art thou in very truth; this is outer shell no
more." Also this digging into one's self, this straight, violent
descent into the pit of one's being, is a troublesome and dangerous
business to start. A man may easily take such hurt, that no physician
can heal him. And again, what were the use, since everything bears
witness to our essence,--our friendships and enmities, our looks and
greetings, our memories and forgetfulnesses, our books and our
writing! This is the most effective way:--to let the youthful soul
look back on life with the question, "What hast thou up to now truly
loved, what has drawn thy soul upward, mastered it and blessed it
too?" Set up these things that thou hast honoured before thee, and,
maybe, they will show thee, in their being and their order, a law
which is the fundamental law of thine own self. Compare these
objects, consider how one completes and broadens and transcends and
explains another, how they form a ladder on which thou hast all the
time been climbing to thy self: for thy true being lies not deeply
hidden in thee, but an infinite height above thee, or at least above
that which thou dost commonly take to be thyself. The true educators
and moulders reveal to thee the real groundwork and import of thy
being, something that in itself cannot be moulded or educated, but is
anyhow difficult of approach, bound and crippled: thy educators can
be nothing but thy deliverers. And that is the secret of all culture:
it does not give artificial limbs, wax noses, or spectacles for the
eyes--a thing that could buy such gifts is but the base coin of
education. But it is rather a liberation, a removal of all the
weeds and rubbish and vermin that attack the delicate shoots, the
streaming forth of light and warmth, the tender dropping of the night
rain; it is the following and the adoring of Nature when she is
pitifully-minded as a mother;--her completion, when it bends before
her fierce and ruthless blasts and turns them to good, and draws
a veil over all expression of her tragic unreason--for she is a
step-mother too, sometimes.

There are other means of "finding ourselves," of coming to ourselves
out of the confusion wherein we all wander as in a dreary cloud; but
I know none better than to think on our educators. So I will to-day
take as my theme the hard teacher Arthur Schopenhauer, and speak of
others later.


II.

In order to describe properly what an event my first look into
Schopenhauer's writings was for me, I must dwell for a minute on an
idea, that recurred more constantly in my youth, and touched me more
nearly, than any other. I wandered then as I pleased in a world of
wishes, and thought that destiny would relieve me of the dreadful and
wearisome duty of educating myself: some philosopher would come at
the right moment to do it for me,--some true philosopher, who could
be obeyed without further question, as he would be trusted more than
one's self. Then I said within me: "What would be the principles, on
which he might teach thee?" And I pondered in my mind what he would
say to the two maxims of education that hold the field in our time.
The first demands that the teacher should find out at once the strong
point in his pupil, and then direct all his skill and will, all the
moisture and all the sunshine, to bring the fruit of that single
virtue to maturity. The second requires him to raise to a higher
power all the qualities that already exist, cherish them and bring
them into a harmonious relation. But, we may ask, should one who has
a decided talent for working in gold be made for that reason to learn
music? And can we admit that Benvenuto Cellini's father was right in
continually forcing him back to the "dear little horn"--the "cursed
piping," as his son called it? We cannot think so in the case of such
a strong and clearly marked talent as his, and it may well be that
this maxim of harmonious development applies only to weaker natures,
in which there is a whole swarm of desires and inclinations, though
they may not amount to very much, singly or together. On the other
hand, where do we find such a blending of harmonious voices--nay, the
soul of harmony itself--as we see in natures like Cellini's, where
everything--knowledge, desire, love and hate--tends towards a single
point, the root of all, and a harmonious system, the resultant of the
various forces, is built up through the irresistible domination of
this vital centre? And so perhaps the two maxims are not contrary at
all; the one merely saying that man must have a centre, the other, a
circumference as well. The philosophic teacher of my dream would not
only discover the central force, but would know how to prevent its
being destructive of the other powers: his task, I thought, would be
the welding of the whole man into a solar system with life and
movement, and the discovery of its paraphysical laws.

In the meantime I could not find my philosopher, however I tried; I
saw how badly we moderns compare with the Greeks and Romans, even in
the serious study of educational problems. You can go through all
Germany, and especially all the universities, with this need in your
heart, and will not find what you seek; many humbler wishes than that
are still unfulfilled there. For example, if a German seriously wish
to make himself an orator, or to enter a "school for authors," he
will find neither master nor school: no one yet seems to have thought
that speaking and writing are arts which cannot be learnt without the
most careful method and untiring application. But, to their shame,
nothing shows more clearly the insolent self-satisfaction of our
people than the lack of demand for educators; it comes partly from
meanness, partly from want of thought. Anything will do as a
so-called "family tutor," even among our most eminent and cultured
people; and what a menagerie of crazy heads and mouldy devices mostly
go to make up the belauded Gymnasium! And consider what we are
satisfied with in our finishing schools,--our universities. Look at
our professors and their institutions! And compare the difficulty of
the task of educating a man to be a man! Above all, the wonderful way
in which the German savants fall to their dish of knowledge, shows
that they are thinking more of Science than mankind; and they are
trained to lead a forlorn hope in her service, in order to encourage
ever new generations to the same sacrifice. If their traffic with
knowledge be not limited and controlled by any more general
principles of education, but allowed to run on indefinitely,--"the
more the better,"--it is as harmful to learning as the economic
theory of _laisser faire_ to common morality. No one recognises now
that the education of the professors is an exceedingly difficult
problem, if their humanity is not to be sacrificed or shrivelled
up:--this difficulty can be actually seen in countless examples of
natures warped and twisted by their reckless and premature devotion
to science. There is a still more important testimony to the complete
absence of higher education, pointing to a greater and more universal
danger. It is clear at once why an orator or writer cannot now be
educated,--because there are no teachers; and why a savant must be a
distorted and perverted thing,--because he will have been trained by
the inhuman abstraction, science. This being so, let a man ask
himself: "Where are now the types of moral excellence and fame for
all our generation--learned and unlearned, high and low--the visible
abstract of constructive ethics for this age? Where has vanished all
the reflection on moral questions that has occupied every great
developed society at all epochs?" There is no fame for that now, and
there are none to reflect: we are really drawing on the inherited
moral capital which our predecessors accumulated for us, and which we
do not know how to increase, but only to squander. Such things are
either not mentioned in our society, or, if at all, with a naïve want
of personal experience that makes one disgusted. It comes to this,
that our schools and professors simply turn aside from any moral
instruction or content themselves with formulæ; virtue is a word and
nothing more, on both sides, an old-fashioned word that they laugh
at--and it is worse when they do not laugh, for then they are
hypocrites.

An explanation of this faint-heartedness and ebbing of all moral
strength would be difficult and complex: but whoever is considering
the influence of Christianity in its hour of victory on the morality
of the mediæval world, must not forget that it reacts also in its
defeat, which is apparently its position to-day. By its lofty ideal,
Christianity has outbidden the ancient Systems of Ethics and their
invariable naturalism, with which men came to feel a dull disgust:
and afterwards when they did reach the knowledge of what was better
and higher, they found they had no longer the power, for all their
desire, to return to its embodiment in the antique virtues. And so
the life of the modern man is passed in see-sawing between
Christianity and Paganism, between a furtive or hypocritical approach
to Christian morality, and an equally shy and spiritless dallying
with the antique: and he does not thrive under it. His inherited fear
of naturalism, and its more recent attraction for him, his desire to
come to rest somewhere, while in the impotence of his intellect he
swings backwards and forwards between the "good" and the "better"
course--all this argues an instability in the modern mind that
condemns it to be without joy or fruit. Never were moral teachers
more necessary and never were they more unlikely to be found:
physicians are most in danger themselves in times when they are most
needed and many men are sick. For where are our modern physicians who
are strong and sure-footed enough to hold up another or lead him by
the hand? There lies a certain heavy gloom on the best men of our
time, an eternal loathing for the battle that is fought in their
hearts between honesty and lies, a wavering of trust in themselves,
which makes them quite incapable of showing to others the way they
must go.

So I was right in speaking of my "wandering in a world of wishes"
when I dreamt of finding a true philosopher who could lift me from
the slough of insufficiency, and teach me again simply and honestly
to be in my thoughts and life, in the deepest sense of the word, "out
of season"; simply and honestly--for men have now become such
complicated machines that they must be dishonest, if they speak at
all, or wish to act on their words.

With such needs and desires within me did I come to know
Schopenhauer.

I belong to those readers of Schopenhauer who know perfectly well,
after they have turned the first page, that they will read all the
others, and listen to every word that he has spoken. My trust in him
sprang to life at once, and has been the same for nine years. I
understood him as though he had written for me (this is the most
intelligible, though a rather foolish and conceited way of expressing
it). Hence I never found a paradox in him, though occasionally some
small errors: for paradoxes are only assertions that carry no
conviction, because the author has made them himself without any
conviction, wishing to appear brilliant, or to mislead, or, above
all, to pose. Schopenhauer never poses: he writes for himself, and no
one likes to be deceived--least of all a philosopher who has set this
up as his law: "deceive nobody, not even thyself," neither with the
"white lies" of all social intercourse, which writers almost
unconsciously imitate, still less with the more conscious deceits of
the platform, and the artificial methods of rhetoric. Schopenhauer's
speeches are to himself alone; or if you like to imagine an auditor,
let it be a son whom the father is instructing. It is a rough,
honest, good-humoured talk to one who "hears and loves." Such writers
are rare. His strength and sanity surround us at the first sound of
his voice: it is like entering the heights of the forest, where we
breathe deep and are well again. We feel a bracing air everywhere, a
certain candour and naturalness of his own, that belongs to men who
are at home with themselves, and masters of a very rich home indeed:
he is quite different from the writers who are surprised at
themselves if they have said something intelligent, and whose
pronouncements for that reason have something nervous and unnatural
about them. We are just as little reminded in Schopenhauer of the
professor with his stiff joints worse for want of exercise, his
narrow chest and scraggy figure, his slinking or strutting gait. And
again his rough and rather grim soul leads us not so much to miss as
to despise the suppleness and courtly grace of the excellent
Frenchmen; and no one will find in him the gilded imitations of
pseudo-gallicism that our German writers prize so highly. His style
in places reminds me a little of Goethe, but is not otherwise on any
German model. For he knows how to be profound with simplicity,
striking without rhetoric, and severely logical without pedantry: and
of what German could he have learnt that? He also keeps free from the
hair-splitting, jerky and (with all respect) rather un-German manner
of Lessing: no small merit in him, for Lessing is the most tempting
of all models for prose style. The highest praise I can give his
manner of presentation is to apply his own phrase to himself:--"A
philosopher must be very honest to avail himself of no aid from
poetry or rhetoric." That honesty is something, and even a virtue, is
one of those private opinions which are forbidden in this age of
public opinion; and so I shall not be praising Schopenhauer, but only
giving him a distinguishing mark, when I repeat that he is honest,
even as a writer; so few of them are, that we are apt to mistrust
every one who writes at all. I only know a single author that I can
rank with Schopenhauer, or even above him, in the matter of honesty;
and that is Montaigne. The joy of living on this earth is increased
by the existence of such a man. The effect on myself, at any rate,
since my first acquaintance with that strong and masterful spirit,
has been, that I can say of him as he of Plutarch--"As soon as I open
him, I seem to grow a pair of wings." If I had the task of making
myself at home on the earth, I would choose him as my companion.

Schopenhauer has a second characteristic in common with Montaigne,
besides honesty; a joy that really makes others joyful. "Aliis lætus,
sibi sapiens." There are two very different kinds of joyfulness. The
true thinker always communicates joy and life, whether he is showing
his serious or comic side, his human insight or his godlike
forbearance: without surly looks or trembling hands or watery eyes,
but simply and truly, with fearlessness and strength, a little
cavalierly perhaps, and sternly, but always as a conqueror: and it is
this that brings the deepest and intensest joy, to see the conquering
god with all the monsters that he has fought. But the joyfulness one
finds here and there in the mediocre writers and limited thinkers
makes some of us miserable; I felt this, for example, with the
"joyfulness" of David Strauss. We are generally ashamed of such a
quality in our contemporaries, because they show the nakedness of our
time, and of the men in it, to posterity. Such _fils de joie_ do not
see the sufferings and the monsters, that they pretend, as
philosophers, to see and fight; and so their joy deceives us, and we
hate it; it tempts to the false belief that they have gained some
victory. At bottom there is only joy where there is victory: and this
applies to true philosophy as much as to any work of art. The
contents may be forbidding and serious, as the problem of existence
always is; the work will only prove tiresome and oppressive, if the
slip-shod thinker and the dilettante have spread the mist of their
insufficiency over it: while nothing happier or better can come to
man's lot than to be near one of those conquering spirits whose
profound thought has made them love what is most vital, and whose
wisdom has found its goal in beauty. They really speak: they are no
stammerers or babblers; they live and move, and have no part in the
_danse macabre_ of the rest of humanity. And so in their company one
feels a natural man again, and could cry out with Goethe--"What a
wondrous and priceless thing is a living creature! How fitted to his
surroundings, how true, and real!"

I have been describing nothing but the first, almost physiological,
impression made upon me by Schopenhauer, the magical emanation of
inner force from one plant of Nature to another, that follows the
slightest contact. Analysing it, I find that this influence of
Schopenhauer has three elements, his honesty, his joy, and his
consistency. He is honest, as speaking and writing for himself alone;
joyful, because his thought has conquered the greatest difficulties;
consistent, because he cannot help being so. His strength rises like
a flame in the calm air, straight up, without a tremor or deviation.
He finds his way, without our noticing that he has been seeking it:
so surely and cleverly and inevitably does he run his course, as if
by some law of gravitation. If any one have felt what it means to
find, in our present world of Centaurs and Chimæras, a single-hearted
and unaffected child of nature who moves unconstrained on his own
road, he will understand my joy and surprise in discovering
Schopenhauer: I knew in him the educator and philosopher I had so
long desired. Only, however, in his writings: which was a great loss.
All the more did I exert myself to see behind the book the living man
whose testament it was, and who promised his inheritance to such as
could, and would, be more than his readers--his pupils and his sons.


III.

I get profit from a philosopher, just so far as he can be an example
to me. There is no doubt that a man can draw whole nations after him
by his example; as is shown by Indian history, which is practically
the history of Indian philosophy. But this example must exist in his
outward life, not merely in his books; it must follow the way of the
Grecian philosophers, whose doctrine was in their dress and bearing
and general manner of life rather than in their speech or writing. We
have nothing yet of this "breathing testimony" in German
philosophical life; the spirit has, apparently, long completed its
emancipation, while the flesh has hardly begun; yet it is foolish to
think that the spirit can be really free and independent when this
victory over limitation--which is ultimately a formative limiting of
one's self--is not embodied anew in every look and movement. Kant
held to his university, submitted to its regulations, and belonged,
as his colleagues and students thought, to a definite religious
faith: and naturally his example has produced, above all, University
professors of philosophy. Schopenhauer makes small account of the
learned tribe, keeps himself exclusive, and cultivates an
independence from state and society as his ideal, to escape the
chains of circumstance here: that is his value to us. Many steps in
the enfranchisement of the philosopher are unknown in Germany; they
cannot always remain so. Our artists live more bravely and honourably
than our philosophers; and Richard Wagner, the best example of all,
shows how genius need not fear a fight to the death with the
established forms and ordinances, if we wish to bring the higher
truth and order, that lives in him, to the light. The "truth,"
however, of which we hear so much from our professors, seems to be a
far more modest being, and no kind of disturbance is to be feared
from her; she is an easy-going and pleasant creature, who is
continually assuring the powers that be that no one need fear any
trouble from her quarter: for man is only "pure reason." And
therefore I will say, that philosophy in Germany has more and more to
learn not to be "pure reason": and it may well take as its model
"Schopenhauer the man."

It is no less than a marvel that he should have come to be this human
kind of example: for he was beset, within and without, by the most
frightful dangers, that would have crushed and broken a weaker
nature. I think there was a strong likelihood of Schopenhauer the man
going under, and leaving at best a residue of "pure reason": and only
"at best"--it was more probable that neither man nor reason would
survive.

A modern Englishman sketches the most usual danger to extraordinary
men who live in a society that worships the ordinary, in this
manner:--"Such uncommon characters are first cowed, then become sick
and melancholy, and then die. A Shelley could never have lived in
England: a race of Shelleys would have been impossible." Our
Hölderlins and Kleists were undone by their unconventionality, and
were not strong enough for the climate of the so-called German
culture; and only iron natures like Beethoven, Goethe, Schopenhauer
and Wagner could hold out against it. Even in them the effect of this
weary toiling and moiling is seen in many lines and wrinkles; their
breathing is harder and their voice is forced. The old diplomatist
who had only just seen and spoken to Goethe, said to a friend--"Voilà
un homme qui a eu de grands chagrins!" which Goethe translated to
mean "That is a man who has taken great pains in his life." And he
adds, "If the trace of the sorrow and activity we have gone through
cannot be wiped from our features, it is no wonder that all that
survives of us and our struggles should bear the same impress." And
this is the Goethe to whom our cultured Philistines point as the
happiest of Germans, that they may prove their thesis, that it must
be possible to be happy among them--with the unexpressed corollary
that no one can be pardoned for feeling unhappy and lonely among
them. Hence they push their doctrine, in practice, to its merciless
conclusion, that there is always a secret guilt in isolation. Poor
Schopenhauer had this secret guilt too in his heart, the guilt of
cherishing his philosophy more than his fellow-men; and he was so
unhappy as to have learnt from Goethe that he must defend his
philosophy at all costs from the neglect of his contemporaries, to
save its very existence: for there is a kind of Grand Inquisitor's
Censure in which the Germans, according to Goethe, are great adepts:
it is called--inviolable silence. This much at least was accomplished
by it;--the greater part of the first edition of Schopenhauer's
masterpiece had to be turned into waste paper. The imminent risk that
his great work would be undone, merely by neglect, bred in him a
state of unrest--perilous and uncontrollable;--for no single adherent
of any note presented himself. It is tragic to watch his search for
any evidence of recognition: and his piercing cry of triumph at last,
that he would now really be read (_legor et legar_), touches us with
a thrill of pain. All the traits in which we do not see the great
philosopher show us the suffering man, anxious for his noblest
possessions; he was tortured by the fear of losing his little
property, and perhaps of no longer being able to maintain in its
purity his truly antique attitude towards philosophy. He often chose
falsely in his desire to find real trust and compassion in men, only
to return with a heavy heart to his faithful dog again. He was
absolutely alone, with no single friend of his own kind to comfort
him; and between one and none there lies an infinity--as ever between
something and nothing. No one who has true friends knows what real
loneliness means, though he may have the whole world in antagonism
round him. Ah, I see well ye do not know what isolation is! Whenever
there are great societies with governments and religions and public
opinions--where there is a tyranny, in short, there will the lonely
philosopher be hated: for philosophy offers an asylum to mankind
where no tyranny can penetrate, the inner sanctuary, the centre of
the heart's labyrinth: and the tyrants are galled at it. Here do the
lonely men lie hid: but here too lurks their greatest danger. These
men who have saved their inner freedom, must also live and be seen in
the outer world: they stand in countless human relations by their
birth, position, education and country, their own circumstances and
the importunity of others: and so they are presumed to hold an
immense number of opinions, simply because these happen to prevail:
every look that is not a denial counts as an assent, every motion of
the hand that does not destroy is regarded as an aid. These free and
lonely men know that they perpetually seem other than they are. While
they wish for nothing but truth and honesty, they are in a net of
misunderstanding; and that ardent desire cannot prevent a mist of
false opinions, of adaptations and wrong conclusions, of partial
misapprehension and intentional reticence, from gathering round their
actions. And there settles a cloud of melancholy on their brows: for
such natures hate the necessity of pretence worse than death: and the
continual bitterness gives them a threatening and volcanic character.
They take revenge from time to time for their forced concealment and
self-restraint: they issue from their dens with lowering looks: their
words and deeds are explosive, and may lead to their own destruction.
Schopenhauer lived amid dangers of this sort. Such lonely men need
love, and friends, to whom they can be as open and sincere as to
themselves, and in whose presence the deadening silence and hypocrisy
may cease. Take their friends away, and there is left an increasing
peril; Heinrich von Kleist was broken by the lack of love, and the
most terrible weapon against unusual men is to drive them into
themselves; and then their issuing forth again is a volcanic
eruption. Yet there are always some demi-gods who can bear life under
these fearful conditions and can be their conquerors: and if you
would hear their lonely chant, listen to the music of Beethoven.

So the first danger in whose shadow Schopenhauer lived was--
isolation. The second is called--doubting of the truth. To this every
thinker is liable who sets out from the philosophy of Kant, provided
he be strong and sincere in his sorrows and his desires, and not a
mere tinkling thought-box or calculating machine. We all know the
shameful state of things implied by this last reservation, and I
believe it is only a very few men that Kant has so vitally affected
as to change the current of their blood. To judge from what one
reads, there must have been a revolution in every domain of thought
since the work of this unobtrusive professor: I cannot believe it
myself. For I see men, though darkly, as themselves needing to be
revolutionised, before any "domains of thought" can be so. In fact,
we find the first mark of any influence Kant may have had on the
popular mind, in a corrosive scepticism and relativity. But it is
only in noble and active spirits who could never rest in doubt that
the shattering despair of truth itself could take the place of doubt.
This was, for example, the effect of the Kantian philosophy on
Heinrich von Kleist. "It was only a short time ago," he writes in his
poignant way, "that I became acquainted with the Kantian philosophy;
and I will tell you my thought, though I cannot fear that it will
rack you to your inmost soul, as it did me.--We cannot decide,
whether what we call truth is really truth, or whether it only seems
so to us. If the latter, the truth that we amass here does not exist
after death, and all our struggle to gain a possession that may
follow us even to the grave is in vain. If the blade of this thought
do not cut your heart, yet laugh not at another who feels himself
wounded by it in his Holy of Holies. My one highest aim has vanished,
and I have no more." Yes, when will men feel again deeply as Kleist
did, and learn to measure a philosophy by what it means to the "Holy
of Holies"? And yet we must make this estimate of what Schopenhauer
can mean to us, after Kant, as the first pioneer to bring us from the
heights of sceptical disillusionment or "critical" renunciation, to
the greater height of tragic contemplation, the nocturnal heaven with
its endless crown of stars. His greatness is that he can stand
opposite the picture of life, and interpret it to us as a whole:
while all the clever people cannot escape the error of thinking one
comes nearer to the interpretation by a laborious analysis of the
colours and material of the picture; with the confession, probably,
that the texture of the canvas is very complicated, and the chemical
composition of the colours undiscoverable. Schopenhauer knew that one
must guess the painter in order to understand the picture. But now
the whole learned fraternity is engaged on understanding the colours
and canvas, and not the picture: and only he who has kept the
universal panorama of life and being firmly before his eyes, will use
the individual sciences without harm to himself; for, without this
general view as a norm, they are threads that lead nowhere and only
confuse still more the maze of our existence. Here we see, as I said,
the greatness of Schopenhauer, that he follows up every idea, as
Hamlet follows the Ghost, without allowing himself to turn aside for
a learned digression, or be drawn away by the scholastic abstractions
of a rabid dialectic. The study of the minute philosophers is only
interesting for the recognition that they have reached those stages
in the great edifice of philosophy where learned disquisitions for
and against, where hair-splitting objections and counter-objections
are the rule: and for that reason they evade the demand of every
great philosophy to speak _sub specie æternitatis_--"this is the
picture of the whole of life: learn thence the meaning of thine own
life." And the converse: "read thine own life, and understand thence
the hieroglyphs of the universal life." In this way must
Schopenhauer's philosophy always be interpreted; as an individualist
philosophy, starting from the single man, in his own nature, to gain
an insight into his personal miseries, and needs, and limitations,
and find out the remedies that will console them: namely, the
sacrifice of the ego, and its submission to the nobler ends,
especially those of justice and mercy. He teaches us to distinguish
between the true and the apparent furtherance of man's happiness: how
neither the attainment of riches, nor honour, nor learning, can raise
the individual from his deep despair at his unworthiness; and how the
quest for these good things can only have meaning through a universal
end that transcends and explains them;--the gaining of power to aid
our physical nature by them and, as far as may be, correct its folly
and awkwardness. For one's self only, in the first instance: and
finally, through one's self, for all. It is a task that leads to
scepticism: for there is so much to be made better yet, in one and
all!

Applying this to Schopenhauer himself, we come to the third and most
intimate danger in which he lived, and which lay deep in the marrow
of his being. Every one is apt to discover a limitation in himself,
in his gifts of intellect as well as his moral will, that fills him
with yearning and melancholy; and as he strives after holiness
through a consciousness of sin, so, as an intellectual being, he has
a deep longing after the "genius" in himself. This is the root of all
true culture; and if we say this means the aspiration of man to be
"born again" as saint and genius, I know that one need not be a
Buddhist to understand the myth. We feel a strong loathing when we
find talent without such aspiration, in the circle of the learned, or
among the so-called educated; for we see that such men, with all
their cleverness, are no aid but a hindrance to the beginnings of
culture, and the blossoming of genius, the aim of all culture. There
is a rigidity in them, parallel to the cold arrogance of conventional
virtue, which also remains at the opposite pole to true holiness.
Schopenhauer's nature contained an extraordinarily dangerous dualism.
Few thinkers have felt as he did the complete and unmistakable
certainty of genius within them; and his genius made him the highest
of all promises,--that there could be no deeper furrow than that
which he was ploughing in the ground of the modern world. He knew one
half of his being to be fulfilled according to its strength, with no
other need; and he followed with greatness and dignity his vocation
of consolidating his victory. In the other half there was a gnawing
aspiration, which we can understand, when we hear that he turned away
with a sad look from the picture of Rancé, the founder of the
Trappists, with the words: "That is a matter of grace." For genius
evermore yearns after holiness as it sees further and more clearly
from its watch-tower than other men, deep into the reconciliation of
Thought and Being, the kingdom of peace and the denial of the will,
and up to that other shore, of which the Indians speak. The wonder
is, that Schopenhauer's nature should have been so inconceivably
stable and unshakable that it could neither be destroyed nor
petrified by this yearning. Every one will understand this after the
measure of his own character and greatness: none of us will
understand it in the fulness of its meaning.

The more one considers these three dangers, the more extraordinary
will appear his vigour in opposing them and his safety after the
battle. True, he gained many scars and open wounds: and a cast of
mind that may seem somewhat too bitter and pugnacious. But his single
ideal transcends the highest humanity in him. Schopenhauer stands as
a pattern to men, in spite of all those scars and scratches. We may
even say, that what was imperfect and "all too human" in him, brings
us nearer to him as a man, for we see a sufferer and a kinsman to
suffering, not merely a dweller on the unattainable heights of
genius.

These three constitutional dangers that threatened Schopenhauer,
threaten us all. Each one of us bears a creative solitude within
himself and his consciousness of it forms an exotic aura of
strangeness round him. Most men cannot endure it, because they are
slothful, as I said, and because their solitude hangs round them a
chain of troubles and burdens. No doubt, for the man with this heavy
chain, life loses almost everything that one desires from it in
youth--joy, safety, honour: his fellow-men pay him his due
of--isolation! The wilderness and the cave are about him, wherever he
may live. He must look to it that he be not enslaved and oppressed,
and become melancholy thereby. And let him surround himself with the
pictures of good and brave fighters such as Schopenhauer.

The second danger, too, is not rare. Here and there we find one
dowered by nature with a keen vision; his thoughts dance gladly in
the witches' Sabbath of dialectic; and if he uncautiously give his
talent the rein, it is easy to lose all humanity and live a ghostly
life in the realm of "pure reason": or through the constant search
for the "pros and cons" of things, he may go astray from the truth
and live without courage or confidence, in doubt, denial and
discontent, and the slender hope that waits on disillusion: "No dog
could live long thus!"

The third danger is a moral or intellectual hardening: man breaks the
bond that united him to his ideal: he ceases to be fruitful and
reproduce himself in this or that province, and becomes an enemy or a
parasite of culture. The solitude of his being has become an
indivisible, unrelated atom, an icy stone. And one can perish of this
solitude as well as of the fear of it, of one's self as well as one's
self-sacrifice, of both aspiration and petrifaction: and to live is
ever to be in danger.

Beside these dangers to which Schopenhauer would have been
constitutionally liable, in whatever century he had lived, there were
also some produced by his own time; and it is essential to
distinguish between these two kinds, in order to grasp the typical
and formative elements in his nature. The philosopher casts his eye
over existence, and wishes to give it a new standard value; for it
has been the peculiar task of all great thinkers to be law-givers for
the weight and stamp in the mint of reality. And his task will be
hindered if the men he sees near him be a weakly and worm-eaten
growth. To be correct in his calculation of existence, the
unworthiness of the present time must be a very small item in the
addition. The study of ancient or foreign history is valuable, if at
all, for a correct judgment on the whole destiny of man; which must
be drawn not only from an average estimate but from a comparison of
the highest destinies that can befall individuals or nations. The
present is too much with us; it directs the vision even against the
philosopher's will: and it will inevitably be reckoned too high in
the final sum. And so he must put a low figure on his own time as
against others, and suppress the present in his picture of life, as
well as in himself; must put it into the background or paint it over;
a difficult, and almost impossible task. The judgment of the ancient
Greek philosophers on the value of existence means so much more than
our own, because they had the full bloom of life itself before them,
and their vision was untroubled by any felt dualism between their
wish for freedom and beauty on the grand scale, and their search
after truth, with its single question "What is the real _worth_ of
life?" Empedocles lived when Greek culture was full to overflowing
with the joy of life, and all ages may take profit from his words;
especially as no other great philosopher of that great time ventured
to contradict them. Empedocles is only the clearest voice among
them--they all say the same thing, if a man will but open his ears. A
modern thinker is always in the throes of an unfulfilled desire; he
is looking for life,--warm, red life,--that he may pass judgment on
it: at any rate he will think it necessary to be a living man
himself, before he can believe in his power of judging. And this is
the title of the modern philosophers to sit among the great aiders of
Life (or rather of the will to live), and the reason why they can
look from their own out-wearied time and aspire to a truer culture,
and a clearer explanation. Their yearning is, however, their danger;
the reformer in them struggles with the critical philosopher. And
whichever way the victory incline, it also implies a defeat. How was
Schopenhauer to escape this danger?

We like to consider the great man as the noble child of his age, who
feels its defects more strongly and intimately than the smaller men:
and therefore the struggle of the great man _against_ his age is
apparently nothing but a mad fight to the death with himself. Only
apparently, however: he only fights the elements in his time that
hinder his own greatness, in other words his own freedom and
sincerity. And so, at bottom, he is only an enemy to that element
which is not truly himself, the irreconcilable antagonism of the
temporal and eternal in him. The supposed "child of his age" proves
to be but a step-child. From boyhood Schopenhauer strove with his
time, a false and unworthy mother to him, and as soon as he had
banished her, he could bring back his being to its native health and
purity. For this very reason we can use his writings as mirrors of
his time; it is no fault of the mirror if everything contemporary
appear in it stricken by a ravaging disease, pale and thin, with
tired looks and hollow eyes,--the step-child's sorrow made visible.
The yearning for natural strength, for a healthy and simple humanity,
was a yearning for himself: and as soon as he had conquered his time
within him, he was face to face with his own genius. The secret of
nature's being and his own lay open, the step-mother's plot to
conceal his genius from him was foiled. And now he could turn a
fearless eye towards the question, "What is the real worth of life?"
without having any more to weigh a bloodless and chaotic age of doubt
and hypocrisy. He knew that there was something higher and purer to
be won on this earth than the life of his time, and a man does bitter
wrong to existence who only knows it and criticises it in this
hateful form. Genius, itself the highest product of life, is now
summoned to justify life, if it can: the noble creative soul must
answer the question:--"Dost thou in thy heart say 'Yea!' unto this
existence? Is it enough for thee? Wilt thou be its advocate and its
redeemer? One true 'Yea!' from thy lips, and the sorely accused life
shall go free." How shall he answer? In the words of Empedocles.


IV.

The last hint may well remain obscure for a time: I have something
more easy to explain, namely how Schopenhauer can help us to educate
ourselves _in opposition_ to our age, since we have the advantage of
really knowing our age, through him;--if it be an advantage! It may
be no longer possible in a couple of hundred years. I sometimes amuse
myself with the idea that men may soon grow tired of books and their
authors, and the savant of to-morrow come to leave directions in his
will that his body be burned in the midst of his books, including of
course his own writings. And in the gradual clearing of the forests,
might not our libraries be very reasonably used for straw and
brushwood? Most books are born from the smoke and vapour of the
brain: and to vapour and smoke may they well return. For having no
fire within themselves, they shall be visited with fire. And possibly
to a later century our own may count as the "Dark age," because our
productions heated the furnace hotter and more continuously than ever
before. We are anyhow happy that we can learn to know our time; and
if there be any sense in busying ourselves with our time at all, we
may as well do it as thoroughly as we can, so that no one may have
any doubt about it. The possibility of this we owe to Schopenhauer.

Our happiness would of course be infinitely greater, if our inquiry
showed that nothing so hopeful and splendid as our present epoch had
ever existed. There are simple people in some corner of the earth
to-day--perhaps in Germany--who are disposed to believe in all
seriousness that the world was put right two years ago,[1] and that
all stern and gloomy views of life are now contradicted by "facts."
The foundation of the New German Empire is, to them, the decisive
blow that annihilates all the "pessimistic" philosophisers,--no doubt
of it. To judge the philosopher's significance in our time, as an
educator, we must oppose a widespread view like this, especially
common in our universities. We must say, it is a shameful thing that
such abominable flattery of the Time-Fetish should be uttered by a
herd of so-called reflective and honourable men; it is a proof that
we no longer see how far the seriousness of philosophy is removed
from that of a newspaper. Such men have lost the last remnant of
feeling, not only for philosophy, but also for religion, and have put
in its place a spirit not so much of optimism as of journalism, the
evil spirit that broods over the day--and the daily paper. Every
philosophy that believes the problem of existence to be shelved, or
even solved, by a political event, is a sham philosophy. There have
been innumerable states founded since the beginning of the world;
that is an old story. How should a political innovation manage once
and for all to make a contented race of the dwellers on this earth?
If any one believe in his heart that this is possible, he should
report himself to our authorities: he really deserves to be Professor
of Philosophy in a German university, like Harms in Berlin, Jurgen
Meyer in Bonn, and Carrière in Munich.

    [1] This was written in 1873.--TR.

We are feeling the consequences of the doctrine, preached lately from
all the housetops, that the state is the highest end of man and there
is no higher duty than to serve it: I regard this not a relapse into
paganism, but into stupidity. A man who thinks state-service to be
his highest duty, very possibly knows no higher one; yet there are
both men and duties in a region beyond,--and one of these duties,
that seems to me at least of higher value than state-service, is to
destroy stupidity in all its forms--and this particular stupidity
among them. And I have to do with a class of men whose teleological
conceptions extend further than the well-being of a state, I mean
with philosophers--and only with them in their relation to the world
of culture, which is again almost independent of the "good of the
state." Of the many links that make up the twisted chain of humanity,
some are of gold and others of pewter.

How does the philosopher of our time regard culture? Quite
differently, I assure you, from the professors who are so content
with their new state. He seems to see the symptoms of an absolute
uprooting of culture in the increasing rush and hurry of life, and
the decay of all reflection and simplicity. The waters of religion
are ebbing, and leaving swamps or stagnant pools: the nations are
drawing away in enmity again, and long to tear each other in pieces.
The sciences, blindly driving along, on a _laisser faire_ system,
without a common standard, are splitting up, and losing hold of every
firm principle. The educated classes are being swept along in the
contemptible struggle for wealth. Never was the world more worldly,
never poorer in goodness and love. Men of learning are no longer
beacons or sanctuaries in the midst of this turmoil of worldliness;
they themselves are daily becoming more restless, thoughtless,
loveless. Everything bows before the coming barbarism, art and
science included. The educated men have degenerated into the greatest
foes of education, for they will deny the universal sickness and
hinder the physician. They become peevish, these poor nerveless
creatures, if one speak of their weakness and combat the shameful
spirit of lies in them. They would gladly make one believe that they
have outstripped all the centuries, and they walk with a pretence of
happiness which has something pathetic about it, because their
happiness is so inconceivable. One would not even ask them, as
Tannhäuser did Biterolf, "What hast thou, poor wretch, enjoyed!" For,
alas! we know far better ourselves, in another way. There is a wintry
sky over us, and we dwell on a high mountain, in danger and in need.
Short-lived is all our joy, and the sun's rays strike palely on our
white mountains. Music is heard; an old man grinds an organ, and the
dancers whirl round, and the heart of the wanderer is shaken within
him to see it: everything is so disordered, so drab, so hopeless.
Even now there is a sound of joy, of clear thoughtless joy! but soon
the mist of evening closes round, the note dies away, and the
wanderer's footsteps are heard on the gravel; as far as his eye can
reach there is nothing but the grim and desolate face of nature.

It may be one-sided, to insist only on the blurred lines and the dull
colours in the picture of modern life: yet the other side is no more
encouraging, it is only more disturbing. There is certainly strength
there, enormous strength; but it is wild, primitive and merciless.
One looks on with a chill expectancy, as though into the caldron of a
witch's kitchen; every moment there may arise sparks and vapour, to
herald some fearful apparition. For a century we have been ready for
a world-shaking convulsion; and though we have lately been trying to
set the conservative strength of the so-called national state against
the great modern tendency to volcanic destructiveness, it will only
be, for a long time yet, an aggravation of the universal unrest that
hangs over us. We need not be deceived by individuals behaving as if
they knew nothing of all this anxiety: their own restlessness shows
how well they know it. They think more exclusively of themselves than
men ever thought before; they plant and build for their little day,
and the chase for happiness is never greater than when the quarry
must be caught to-day or to-morrow: the next day perhaps there is no
more hunting. We live in the Atomic Age, or rather in the Atomic
Chaos. The opposing forces were practically held together in mediæval
times by the Church, and in some measure assimilated by the strong
pressure which she exerted. When the common tie broke and the
pressure relaxed, they rose once more against each other. The
Reformation taught that many things were "adiaphora"--departments
that needed no guidance from religion: this was the price paid for
its own existence. Christianity paid a similar one to guard itself
against the far more religious antiquity: and laid the seeds of
discord at once. Everything nowadays is directed by the fools and the
knaves, the selfishness of the money-makers and the brute forces of
militarism. The state in their hands makes a good show of
reorganising everything, and of becoming the bond that unites the
warring elements; in other words, it wishes for the same idolatry
from mankind as they showed to the Church.

And we shall yet feel the consequences. We are even now on the
ice-floes in the stream of the Middle Ages: they are thawing fast,
and their movement is ominous: the banks are flooded, and giving way.
The revolution, the atomistic revolution, is inevitable: but what
_are_ those smallest indivisible elements of human society?

There is surely far more danger to mankind in transitional periods
like these than in the actual time of revolution and chaos; they are
tortured by waiting, and snatch greedily at every moment; and this
breeds all kinds of cowardice and selfishness in them: whereas the
true feeling of a great and universal need ever inspires men, and
makes them better. In the midst of such dangers, who will provide the
guardians and champions for _Humanity_, for the holy and inviolate
treasure that has been laid up in the temples, little by little, by
countless generations? Who will set up again the _Image of Man_, when
men in their selfishness and terror see nothing but the trail of the
serpent or the cur in them, and have fallen from their high estate to
that of the brute or the automaton?

There are three Images of Man fashioned by our modern time, which for
a long while yet will urge mortal men to transfigure their own lives;
they are the men of Rousseau, Goethe, and Schopenhauer. The first has
the greatest fire, and is most calculated to impress the people: the
second is only for the few, for those contemplative natures "in the
grand style" who are misunderstood by the crowd. The third demands
the highest activity in those who will follow it: only such men will
look on that image without harm, for it breaks the spirit of that
merely contemplative man, and the rabble shudder at it. From the
first has come forth a strength that led and still leads to fearful
revolution: for in all socialistic upheavals it is ever Rousseau's
man who is the Typhoeus under the Etna. Oppressed and half crushed to
death by the pride of caste and the pitilessness of wealth, spoilt by
priests and bad education, a laughing-stock even to himself, man
cries in his need on "holy mother Nature," and feels suddenly that
she is as far from him as any god of the Epicureans. His prayers do
not reach her; so deeply sunk is he in the Chaos of the unnatural. He
contemptuously throws aside all the finery that seemed his truest
humanity a little while ago--all his arts and sciences, all the
refinements of his life,--he beats with his fists against the walls,
in whose shadow he has degenerated, and goes forth to seek the light
and the sun, the forest and the crag. And crying out, "Nature alone
is good, the natural man alone is human," he despises himself and
aspires beyond himself: a state wherein the soul is ready for a
fearful resolve, but calls the noble and the rare as well from their
utter depths.

Goethe's man is no such threatening force; in a certain sense he is a
corrective and a sedative to those dangerous agitations of which
Rousseau's man is a prey. Goethe himself in his youth followed the
"gospel of kindly Nature" with all the ardour of his soul: his Faust
was the highest and boldest picture of Rousseau's man, so far at any
rate as his hunger for life, his discontent and yearning, his
intercourse with the demons of the heart could be represented. But
what comes from these congregated storm-clouds? Not a single
lightning flash! And here begins the new Image of man--the man
according to Goethe. One might have thought that Faust would have
lived a continual life of suffering, as a revolutionary and a
deliverer, as the negative force that proceeds from goodness, as the
genius of ruin, alike religious and dæmonic, in opposition to his
utterly undæmonic companion; though of course he could not be free of
this companion, and had at once to use and despise his evil and
destructive scepticism--which is the tragic destiny of all
revolutionary deliverers. One is wrong, however, to expect anything
of the sort: Goethe's man here parts company with Rousseau's; for he
hates all violence, all sudden transition--that is, all action: and
the universal deliverer becomes merely the universal traveller. All
the riches of life and nature, all antiquity--arts, mythologies and
sciences--pass before his eager eyes, his deepest desires are aroused
and satisfied, Helen herself can hold him no more--and the moment
must come for which his mocking companion is waiting. At a fair spot
on the earth, his flight comes to an end: his pinions drop, and
Mephistopheles is at his side. When the German ceases to be Faust,
there is no danger greater than of becoming a Philistine and falling
into the hands of the devil--heavenly powers alone can save him.
Goethe's man is, as I said, the contemplative man in the grand style,
who is only kept from dying of ennui by feeding on all the great and
memorable things that have ever existed, and by living from desire to
desire. He is not the active man; and when he does take a place among
active men, as things are, you may be sure that no good will come of
it (think, for example, of the zeal with which Goethe wrote for the
stage!); and further, you may be sure that "things as they are" will
suffer no change. Goethe's man is a conciliatory and conservative
spirit, though in danger of degenerating into a Philistine, just as
Rousseau's man may easily become a Catiline. All his virtues would be
the better by the addition of a little brute force and elemental
passion. Goethe appears to have seen where the weakness and danger of
his creation lay, as is clear from Jarno's word to Wilhelm Meister:
"You are bitter and ill-tempered--which is quite an excellent thing:
if you could once become really angry, it would be still better."

To speak plainly, it is necessary to become really angry in order
that things may be better. The picture of Schopenhauer's man can help
us here. _Schopenhauer's man voluntarily takes upon himself the pain
of telling the truth:_ this pain serves to quench his individual will
and make him ready for the complete transformation of his being,
which it is the inner meaning of life to realise. This openness in
him appears to other men to be an effect of malice, for they think
the preservation of their shifts and pretences to be the first duty
of humanity, and any one who destroys their playthings to be merely
malicious. They are tempted to cry out to such a man, in Faust's
words to Mephistopheles:--

  "So to the active and eternal
  Creative force, in cold disdain
  You now oppose the fist infernal"--

and he who would live according to Schopenhauer would seem to be more
like a Mephistopheles than a Faust--that is, to our weak modern eyes,
which always discover signs of malice in any negation. But there is a
kind of denial and destruction that is the effect of that strong
aspiration after holiness and deliverance, which Schopenhauer was the
first philosopher to teach our profane and worldly generation.
Everything that can be denied, deserves to be denied; and real
sincerity means the belief in a state of things which cannot be
denied, or in which there is no lie. The sincere man feels that his
activity has a metaphysical meaning. It can only be explained by the
laws of a different and a higher life; it is in the deepest sense an
affirmation: even if everything that he does seem utterly opposed to
the laws of our present life. It must lead therefore to constant
suffering; but he knows, as Meister Eckhard did, that "the quickest
beast that will carry you to perfection is suffering." Every one, I
should think, who has such an ideal before him, must feel a wider
sympathy; and he will have a burning desire to become a "Schopenhauer
man";--pure and wonderfully patient, on his intellectual side full of
a devouring fire, and far removed from the cold and contemptuous
"neutrality" of the so-called scientific man; so high above any
warped and morose outlook on life as to offer himself as the first
victim of the truth he has won, with a deep consciousness of the
sufferings that must spring from his sincerity. His courage will
destroy his happiness on earth, he must be an enemy to the men he
loves and the institutions in which he grew up, he must spare neither
person nor thing, however it may hurt him, he will be misunderstood
and thought an ally of forces that he abhors, in his search for
righteousness he will seem unrighteous by human standards: but he
must comfort himself with the words that his teacher Schopenhauer
once used: "A happy life is impossible, the highest thing that man
can aspire to is a _heroic_ life; such as a man lives, who is always
fighting against unequal odds for the good of others; and wins in the
end without any thanks. After the battle is over, he stands like the
Prince in the _re corvo_ of Gozzi, with dignity and nobility in his
eyes, but turned to stone. His memory remains, and will be reverenced
as a hero's; his will, that has been mortified all his life by
toiling and struggling, by evil payment and ingratitude, is absorbed
into Nirvana." Such a heroic life, with its full "mortification"--
corresponds very little to the paltry ideas of the people who talk
most about it, and make festivals in memory of great men, in the
belief that a great man is great in the sense that they are small,
either through exercise of his gifts to please himself or by a blind
mechanical obedience to this inner force; so that the man who does
not possess the gift or feel the compulsion has the same right to be
small as the other to be great. But "gift" and "compulsion" are
contemptible words, mere means of escape from an inner voice, a
slander on him who has listened to the voice--the great man; he least
of all will allow himself to be given or compelled to anything: for
he knows as well as any smaller man how easily life can be taken and
how soft the bed whereon he might lie if he went the pleasant and
conventional way with himself and his fellow-creatures: all the
regulations of mankind are turned to the end that the intense feeling
of life may be lost in continual distractions. Now why will he so
strongly choose the opposite, and try to feel life, which is the same
as to suffer from life? Because he sees that men will tempt him to
betray himself, and that there is a kind of agreement to draw him
from his den. He will prick up his ears and gather himself together,
and say, "I will remain mine own." He gradually comes to understand
what a fearful decision it is. For he must go down into the depths of
being, with a string of curious questions on his lips--"Why am I
alive? what lesson have I to learn from life? how have I become what
I am, and why do I suffer in this existence?" He is troubled, and
sees that no one is troubled in the same way; but rather that the
hands of his fellow-men are passionately stretched out towards the
fantastic drama of the political theatre, or they themselves are
treading the boards under many disguises, youths, men and graybeards,
fathers, citizens, priests, merchants and officials,--busy with the
comedy they are all playing, and never thinking of their own selves.
To the question "To what end dost thou live?" they would all
immediately answer, with pride, "To _become_ a good citizen or
professor or statesman,"--and yet they _are_ something which can
never be changed: and why are they just--this? Ah, and why nothing
better? The man who only regards his life as a moment in the
evolution of a race or a state or a science, and will belong merely
to a history of "becoming," has not understood the lesson of
existence, and must learn it over again. This eternal "becoming
something" is a lying puppet-show, in which man has forgot himself;
it is the force that scatters individuality to the four winds, the
eternal childish game that the big baby time is playing in front of
us--and with us. The heroism of sincerity lies in ceasing to be the
plaything of time. Everything in the process of "becoming" is a
hollow sham, contemptible and shallow: man can only find the solution
of his riddle in "being" something definite and unchangeable. He
begins to test how deep both "becoming" and "being" are rooted in
him--and a fearful task is before his soul; to destroy the first, and
bring all the falsity of things to the light. He wishes to know
everything, not to feed a delicate taste, like Goethe's man, to take
delight, from a safe place in the multiplicity of existence: but he
himself is the first sacrifice that he brings. The heroic man does
not think of his happiness or misery, his virtues or his vices, or of
his being the measure of things; he has no further hopes of himself
and will accept the utter consequences of his hopelessness. His
strength lies in his self-forgetfulness: if he have a thought for
himself, it is only to measure the vast distance between himself and
his aim, and to view what he has left behind him as so much dross.
The old philosophers sought for happiness and truth, with all their
strength: and there is an evil principle in nature that not one shall
find that which he cannot help seeking. But the man who looks for a
lie in everything, and becomes a willing friend to unhappiness, shall
have a marvellous disillusioning: there hovers near him something
unutterable, of which truth and happiness are but idolatrous images
born of the night; the earth loses her dragging weight, the events
and powers of earth become as a dream, and a gradual clearness widens
round him like a summer evening. It is as though the beholder of
these things began to wake, and it had only been the clouds of a
passing dream that had been weaving about him. They will at some time
disappear: and then will it be day.


V.

But I have promised to speak of Schopenhauer, as far as my experience
goes, as an _educator_, and it is far from being sufficient to paint
the ideal humanity which is the "Platonic idea" in Schopenhauer;
especially as my representation is an imperfect one. The most
difficult task remains;--to say how a new circle of duties may spring
from this ideal, and how one can reconcile such a transcendent aim
with ordinary action; to prove, in short, that the ideal is
_educative_. One might otherwise think it to be merely the blissful
or intoxicating vision of a few rare moments, that leaves us
afterwards the prey of a deeper disappointment. It is certain that
the ideal begins to affect us in this way when we come suddenly to
distinguish light and darkness, bliss and abhorrence; this is an
experience that is as old as ideals themselves. But we ought not to
stand in the doorway for long; we should soon leave the first stages,
and ask the question, seriously and definitely, "Is it possible to
bring that incredibly high aim so near us, that it should educate us,
or 'lead us out,' as well as lead us upward?"--in order that the
great words of Goethe be not fulfilled in our case--"Man is born to a
state of limitation: he can understand ends that are simple, present
and definite, and is accustomed to make use of means that are near to
his hand; but as soon as he comes into the open, he knows neither
what he wishes nor what he ought to do, and it is all one whether he
be confused by the multitude of objects or set beside himself by
their greatness and importance. It is always his misfortune to be led
to strive after something which he cannot attain by any ordinary
activity of his own." The objection can be made with apparent reason
against Schopenhauer's man, that his greatness and dignity can only
turn our heads, and put us beyond all community with the active men
of the world: the common round of duties, the noiseless tenor of life
has disappeared. One man may possibly get accustomed to living in a
reluctant dualism, that is, in a contradiction with himself;--
becoming unstable, daily weaker and less productive:--while another
will renounce all action on principle, and scarcely endure to see
others active. The danger is always great when a man is too
heavy-laden, and cannot really _accomplish_ any duties. Stronger
natures may be broken by it; the weaker, which are the majority, sink
into a speculative laziness, and at last, from their laziness, lose
even the power of speculation.

With regard to such objections, I will admit that our work has hardly
begun, and so far as I know, I only see one thing clearly and
definitely--that it is possible for that ideal picture to provide you
and me with a chain of duties that may be accomplished; and some of
us already feel its pressure. In order, however, to be able to speak
in plain language of the formula under which I may gather the new
circle of duties, I must begin with the following considerations.

The deeper minds of all ages have had pity for animals, because they
suffer from life and have not the power to turn the sting of the
suffering against themselves, and understand their being
metaphysically. The sight of blind suffering is the spring of the
deepest emotion. And in many quarters of the earth men have supposed
that the souls of the guilty have entered into beasts, and that the
blind suffering which at first sight calls for such pity has a clear
meaning and purpose to the divine justice,--of punishment and
atonement: and a heavy punishment it is, to be condemned to live in
hunger and need, in the shape of a beast, and to reach no
consciousness of one's self in this life. I can think of no harder
lot than the wild beast's; he is driven to the forest by the fierce
pang of hunger, that seldom leaves him at peace; and peace is itself
a torment, the surfeit after horrid food, won, maybe, by a deadly
fight with other animals. To cling to life, blindly and madly, with
no other aim, to be ignorant of the reason, or even the fact, of
one's punishment, nay, to thirst after it as if it were a pleasure,
with all the perverted desire of a fool--this is what it means to be
an animal. If universal nature leads up to man, it is to show us that
he is necessary to redeem her from the curse of the beast's life, and
that in him existence can find a mirror of itself wherein life
appears, no longer blind, but in its real metaphysical significance.
But we should consider where the beast ends and the man begins--the
man, the one concern of Nature. As long as any one desires life as a
pleasure in itself, he has not raised his eyes above the horizon of
the beast; he only desires more consciously what the beast seeks by a
blind impulse. It is so with us all, for the greater part of our
lives. We do not shake off the beast, but are beasts ourselves,
suffering we know not what.

But there are moments when we do know; and then the clouds break, and
we see how, with the rest of nature, we are straining towards the
man, as to something that stands high above us. We look round and
behind us, and fear the sudden rush of light; the beasts are
transfigured, and ourselves with them. The enormous migrations of
mankind in the wildernesses of the world, the cities they found and
the wars they wage, their ceaseless gatherings and dispersions and
fusions, the doctrines they blindly follow, their mutual frauds and
deceits, the cry of distress, the shriek of victory--are all a
continuation of the beast in us: as if the education of man has been
intentionally set back, and his promise of self-consciousness
frustrated; as if, in fact, after yearning for man so long, and at
last reaching him by her labour, Nature should now recoil from him
and wish to return to a state of unconscious instinct. Ah! she has
need of knowledge, and shrinks before the very knowledge she needs:
the flame flickers unsteadily and fears its own brightness, and takes
hold of a thousand things before the one thing for which knowledge is
necessary. There are moments when we all know that our most elaborate
arrangements are only designed to give us refuge from our real task
in life; we wish to hide our heads somewhere as if our Argus-eyed
conscience could not find us out; we are quick to send our hearts on
state-service, or money-making, or social duties, or scientific work,
in order to possess them no longer ourselves; we are more willing and
instinctive slaves of the hard day's work than mere living requires,
because it seems to us more necessary not to be in a position to
think. The hurry is universal, because every one is fleeing before
himself; its concealment is just as universal, as we wish to seem
contented and hide our wretchedness from the keener eyes; and so
there is a common need for a new carillon of words to hang in the
temple of life, and peal for its noisy festival. We all know the
curious way in which unpleasant memories suddenly throng on us, and
how we do our best by loud talk and violent gestures to put them out
of our minds; but the gestures and the talk of our ordinary life make
one think we are all in this condition, frightened of any memory or
any inward gaze. What is it that is always troubling us? what is the
gnat that will not let us sleep? There are spirits all about us, each
moment of life has something to say to us, but we will not listen to
the spirit-voices. When we are quiet and alone, we fear that
something will be whispered in our ears, and so we hate the quiet,
and dull our senses in society.

We understand this sometimes, as I say, and stand amazed at the whirl
and the rush and the anxiety and all the dream that we call our life;
we seem to fear the awakening, and our dreams too become vivid and
restless, as the awakening draws near. But we feel as well that we
are too weak to endure long those intimate moments, and that we are
not the men to whom universal nature looks as her redeemers. It is
something to be able to raise our heads but for a moment and see the
stream in which we are sunk so deep. We cannot gain even this
transitory moment of awakening by our own strength; we must be lifted
up--and who are they that will uplift us?

The sincere men who have cast out the beast, the philosophers,
artists and saints. Nature--_quæ nunquam facit saltum_--has made her
one leap in creating them; a leap of joy, as she feels herself for
the first time at her goal, where she begins to see that she must
learn not to have goals above her, and that she has played the game
of transition too long. The knowledge transfigures her, and there
rests on her face the gentle weariness of evening that men call
"beauty." Her words after this transfiguration are as a great light
shed over existence: and the highest wish that mortals can reach is
to listen continually to her voice with ears that hear. If a man
think of all that Schopenhauer, for example, must have _heard_ in his
life, he may well say to himself--"The deaf ears, the feeble
understanding and shrunken heart, everything that I call mine,--how I
despise them! Not to be able to fly but only to flutter one's wings!
To look above one's self and have no power to rise! To know the road
that leads to the wide vision of the philosopher, and to reel back
after a few steps! Were there but one day when the great wish might
be fulfilled, how gladly would we pay for it with the rest of life!
To rise as high as any thinker yet into the pure icy air of the
mountain, where there are no mists and veils, and the inner
constitution of things is shown in a stark and piercing clarity! Even
by thinking of this the soul becomes infinitely alone; but were its
wish fulfilled, did its glance once fall straight as a ray of light
on the things below, were shame and anxiety and desire gone for
ever--one could find no words for its state then, for the mystic and
tranquil emotion with which, like the soul of Schopenhauer, it would
look down on the monstrous hieroglyphics of existence and the
petrified doctrines of "becoming"; not as the brooding night, but as
the red and glowing day that streams over the earth. And what a
destiny it is only to know enough of the fixity and happiness of the
philosopher to feel the complete unfixity and unhappiness of the
false philosopher, 'who without hope lives in desire': to know one's
self to be the fruit of a tree that is too much in the shade ever to
ripen, and to see a world of sunshine in front, where one may not
go!"

There were sorrow enough here, if ever, to make such a man envious
and spiteful: but he will turn aside, that he may not destroy his
soul by a vain aspiration; and will discover a new circle of duties.

I can now give an answer to the question whether it be possible to
approach the great ideal of Schopenhauer's man "by any ordinary
activity of our own." In the first place, the new duties are
certainly not those of a hermit; they imply rather a vast community,
held together not by external forms but by a fundamental idea, namely
that of _culture_; though only so far as it can put a single task
before each of us--to bring the philosopher, the artist and the
saint, within and without us, to the light, and to strive thereby for
the completion of Nature. For Nature needs the artist, as she needs
the philosopher, for a metaphysical end, the explanation of herself,
whereby she may have a clear and sharp picture of what she only saw
dimly in the troubled period of transition,--and so may reach
self-consciousness. Goethe, in an arrogant yet profound phrase,
showed how all Nature's attempts only have value in so far as the
artist interprets her stammering words, meets her half-way, and
speaks aloud what she really means. "I have often said, and will
often repeat," he exclaims in one place, "the _causa finalis_ of
natural and human activity is dramatic poetry. Otherwise the stuff is
of no use at all."

Finally, Nature needs the saint. In him the ego has melted away, and
the suffering of his life is, practically, no longer felt as
individual, but as the spring of the deepest sympathy and intimacy
with all living creatures: he sees the wonderful transformation scene
that the comedy of "becoming" never reaches, the attainment, at
length, of the high state of man after which all nature is striving,
that she may be delivered from herself. Without doubt, we all stand
in close relation to him, as well as to the philosopher and the
artist: there are moments, sparks from the clear fire of love, in
whose light we understand the word "I" no longer; there is something
beyond our being that comes, for those moments, to the hither side of
it: and this is why we long in our hearts for a bridge from here to
there. In our ordinary state we can do nothing towards the production
of the new redeemer, and so we hate ourselves in this state with a
hatred that is the root of the pessimism which Schopenhauer had to
teach again to our age, though it is as old as the aspiration after
culture.--Its root, not its flower; the foundation, not the summit;
the beginning of the road, not the end: for we have to learn at some
time to hate something else, more universal than our own personality
with its wretched limitation, its change and its unrest--and this
will be when we shall learn to love something else than we can love
now. When we are ourselves received into that high order of
philosophers, artists and saints, in this life or a reincarnation of
it, a new object for our love and hate will also rise before us. As
it is, we have our task and our circle of duties, our hates and our
loves. For we know that culture requires us to make ready for the
coming of the Schopenhauer man;--and this is the "use" we are to make
of him;--we must know what obstacles there are and strike them from
our path--in fact, wage unceasing war against everything that
hindered our fulfilment, and prevented us from becoming
Schopenhauer's men ourselves.


VI.

It is sometimes harder to agree to a thing than to understand it;
many will feel this when they consider the proposition--"Mankind must
toil unceasingly to bring forth individual great men: this and
nothing else is its task." One would like to apply to society and its
ends a fact that holds universally in the animal and vegetable world;
where progress depends only on the higher individual types, which are
rarer, yet more persistent, complex and productive. But traditional
notions of what the end of society is, absolutely bar the way. We can
easily understand how in the natural world, where one species passes
at some point into a higher one, the aim of their evolution cannot be
held to lie in the high level attained by the mass, or in the latest
types developed;--but rather in what seem accidental beings produced
here and there by favourable circumstances. It should be just as easy
to understand that it is the duty of mankind to provide the
circumstances favourable to the birth of the new redeemer, simply
because men can have a consciousness of their object. But there is
always something to prevent them. They find their ultimate aim in the
happiness of all, or the greatest number, or in the expansion of a
great commonwealth. A man will very readily decide to sacrifice his
life for the state; he will be much slower to respond if an
individual, and not a state, ask for the sacrifice. It seems to be
out of reason that one man should exist for the sake of another: "Let
it be rather for the sake of every other, or, at any rate, of as many
as possible!" O upright judge! As if it were more in reason to let
the majority decide a question of value and significance! For the
problem is--"In what way may your life, the individual life, retain
the highest value and the deepest significance? and how may it least
be squandered?" Only by your living for the good of the rarest and
most valuable types, not for that of the majority,--who are the most
worthless types, taken as individuals. This way of thinking should be
implanted and fostered in every young man's mind: he should regard
himself both as a failure of Nature's handiwork and a testimony to
her larger ideas. "She has succeeded badly," he should say; "but I
will do honour to her great idea by being a means to its better
success."

With these thoughts he will enter the circle of culture, which is the
child of every man's self-knowledge and dissatisfaction. He will
approach and say aloud: "I see something above me, higher and more
human than I: let all help me to reach it, as I will help all who
know and suffer as I do, that the man may arise at last who feels his
knowledge and love, vision and power, to be complete and boundless,
who in his universality is one with nature, the critic and judge of
existence." It is difficult to give any one this courageous
self-consciousness, because it is impossible to teach love; from love
alone the soul gains, not only the clear vision that leads to
self-contempt, but also the desire to look to a higher self which is
yet hidden, and strive upward to it with all its strength. And so he
who rests his hope on a future great man, receives his first
"initiation into culture." The sign of this is shame or vexation at
one's self, a hatred of one's own narrowness, a sympathy with the
genius that ever raises its head again from our misty wastes, a
feeling for all that is struggling into life, the conviction that
Nature must be helped in her hour of need to press forward to the
man, however ill she seem to prosper, whatever success may attend her
marvellous forms and projects: so that the men with whom we live are
like the débris of some precious sculptures, which cry out--"Come and
help us! Put us together, for we long to become complete."

I called this inward condition the "first initiation into culture." I
have now to describe the effects of the "second initiation," a task
of greater difficulty. It is the passage from the inner life to the
criticism of the outer life. The eye must be turned to find in the
great world of movement the desire for culture that is known from the
immediate experience of the individual; who must use his own
strivings and aspirations as the alphabet to interpret those of
humanity. He cannot rest here either, but must go higher. Culture
demands from him not only that inner experience, not only the
criticism of the outer world surrounding him, but action too to crown
them all, the fight for culture against the influences and
conventions and institutions where he cannot find his own aim,--the
production of genius.

Any one who can reach the second step, will see how extremely rare
and imperceptible the knowledge of that end is, though all men busy
themselves with culture and expend vast labour in her service. He
asks himself in amazement--"Is not such knowledge, after all,
absolutely necessary? Can Nature be said to attain her end, if men
have a false idea of the aim of their own labour?" And any one who
thinks a great deal of Nature's unconscious adaptation of means to
ends, will probably answer at once: "Yes, men may think and speak
what they like about their ultimate end, their blind instinct will
tell them the right road." It requires some experience of life to be
able to contradict this: but let a man be convinced of the real aim
of culture--the production of the true man and nothing else;--let him
consider that amid all the pageantry and ostentation of culture at
the present time the conditions for his production are nothing but a
continual "battle of the beasts": and he will see that there is great
need for a conscious will to take the place of that blind instinct.
There is another reason also;--to prevent the possibility of turning
this obscure impulse to quite different ends, in a direction where
our highest aim can no longer be attained. For we must beware of a
certain kind of misapplied and parasitical culture; the powers at
present most active in its propagation have other casts of thought
that prevent their relation to culture from being pure and
disinterested.

The first of these is the self-interest of the business men. This
needs the help of culture, and helps her in return, though at the
price of prescribing her ends and limits. And their favourite sorites
is: "We must have as much knowledge and education as possible; this
implies as great a need as possible for it, this again as much
production, this again as much material wealth and happiness as
possible."--This is the seductive formula. Its preachers would define
education as the insight that makes man through and through a "child
of his age" in his desires and their satisfaction, and gives him
command over the best means of making money. Its aim would be to make
"current" men, in the same sense as one speaks of the "currency" in
money; and in their view, the more "current" men there are, the
happier the people. The object of modern educational systems is
therefore to make each man as "current" as his nature will allow him,
and to give him the opportunity for the greatest amount of success
and happiness that can be got from his particular stock of knowledge.
He is required to have just so much idea of his own value (through
his liberal education) as to know what he can ask of life; and he is
assured that a natural and necessary connection between "intelligence
and property" not only exists, but is also a _moral_ necessity. All
education is detested that makes for loneliness, and has an aim above
money-making, and requires a long time: men look askance on such
serious education, as mere "refined egoism" or "immoral
Epicureanism." The converse of course holds, according to the
ordinary morality, that education must be soon over to allow the
pursuit of money to be soon begun, and should be just thorough enough
to allow of much money being made. The amount of education is
determined by commercial interests. In short, "man has a necessary
claim to worldly happiness; only for that reason is education
necessary."

There is, secondly, the self-interest of the state, which requires
the greatest possible breadth and universality of culture, and has
the most effective weapons to carry out its wishes. If it be firmly
enough established not only to initiate but control education and
bear its whole weight, such breadth will merely profit the
competition of the state with other states. A "highly civilised
state" generally implies, at the present time, the task of setting
free the spiritual forces of a generation just so far as they may be
of use to the existing institutions,--as a mountain stream is split
up by embankments and channels, and its diminished power made to
drive mill-wheels, its full strength being more dangerous than useful
to the mills. And thus "setting free" comes to mean rather "chaining
up." Compare, for example, what the self-interest of the state has
done for Christianity. Christianity is one of the purest
manifestations of the impulse towards culture and the production of
the saint: but being used in countless ways to turn the mills of the
state authorities, it gradually became sick at heart, hypocritical
and degenerate, and in antagonism with its original aim. Its last
phase, the German Reformation, would have been nothing but a sudden
flickering of its dying flame, had it not taken new strength and
light from the clash and conflagration of states.

In the third place, culture will be favoured by all those people who
know their own character to be offensive or tiresome, and wish to
draw a veil of so-called "good form" over them. Words, gestures,
dress, etiquette, and such external things, are meant to produce a
false impression, the inner side to be judged from the outer. I
sometimes think that modern men are eternally bored with each other
and look to the arts to make them interesting. They let their artists
make savoury and inviting dishes of them; they steep themselves in
the spices of the East and West, and have a very interesting aroma
after it all. They are ready to suit all palates: and every one will
be served, whether he want something with a good or bad taste,
something sublime or coarse, Greek or Chinese, tragedy or
gutter-drama. The most celebrated chefs among the moderns who wish to
interest and be interested at any price, are the French; the worst
are the Germans. This is really more comforting for the latter, and
we have no reason to mind the French despising us for our want of
interest, elegance and politeness, and being reminded of the Indian
who longs for a ring through his nose, and then proceeds to tattoo
himself.

Here I must digress a little. Many things in Germany have evidently
been altered since the late war with France, and new requirements for
German culture brought over. The war was for many their first venture
into the more elegant half of the world: and what an admirable
simplicity the conqueror shows in not scorning to learn something of
culture from the conquered! The applied arts especially will be
reformed to emulate our more refined neighbours, the German house
furnished like the French, a "sound taste" applied to the German
language by means of an Academy on the French model, to shake off the
doubtful influence of Goethe--this is the judgment of our new Berlin
Academician, Dubois-Raymond. Our theatres have been gradually moving,
in a dignified way, towards the same goal, even the elegant German
savant is now discovered--and we must now expect everything that does
not conform to this law of elegance, our music, tragedy and
philosophy, to be thrust aside as un-German. But there were no need
to raise a finger for German culture, did German culture (which the
Germans have yet to find) mean nothing but the little amenities that
make life more decorative--including the arts of the dancing-master
and the upholsterer;--or were they merely interested in academic
rules of language and a general atmosphere of politeness. The late
war and the self-comparison with the French do not seem to have
aroused any further desires, and I suspect that the German has a
strong wish for the moment to be free of the old obligations laid on
him by his wonderful gifts of seriousness and profundity. He would
much rather play the buffoon and the monkey, and learn the arts that
make life amusing. But the German spirit cannot be more dishonoured
than by being treated as wax for any elegant mould.

And if, unfortunately, a good many Germans will allow themselves to
be thus moulded, one must continually say to them, till at last they
listen:--"The old German way is no longer yours: it was hard, rough,
and full of resistance; but it is still the most valuable
material--one which only the greatest modellers can work with, for
they alone are worthy to use it. What you have in you now is a soft
pulpy stuff: make what you will out of it,--elegant dolls and
interesting idols--Richard Wagner's phrase will still hold good, 'The
German is awkward and ungainly when he wishes to be polite; he is
high above all others, when he begins to take fire.'" All the elegant
people have reason to beware of this German fire; it may one day
devour them with all their wax dolls and idols.--The prevailing love
of "good form" in Germany may have a deeper cause in the breathless
seizing at what the moment can give, the haste that plucks the fruit
too green, the race and the struggle that cut the furrows in men's
brows and stamp the same mark on all their actions. As if there were
a poison in them that would not let them breathe, they rush about in
disorder, anxious slaves of the "three m's," the moment, the mode and
the mob: they see too well their want of dignity and fitness, and
need a false elegance to hide their galloping consumption. The
fashionable desire of "good form" is bound up with a loathing of
man's inner nature: the one is to conceal, the other to be concealed.
Education means now the concealment of man's misery and wickedness,
his wild-beast quarrels, his eternal greed, his shamelessness in
fruition. In pointing out the absence of a German culture, I have
often had the reproach flung at me: "This absence is quite natural,
for the Germans have been too poor and modest up to now. Once rich
and conscious of themselves, our people will have a culture too."
Faith may often produce happiness, yet _this_ particular faith makes
me unhappy, for I feel that the culture whose future raises such
hopes--the culture of riches, politeness, and elegant concealments--
is the bitterest foe of that German culture in which I believe. Every
one who has to live among Germans suffers from the dreadful grayness
and apathy of their lives, their formlessness, torpor and clumsiness,
still more their envy, secretiveness and impurity: he is troubled by
their innate love of the false and the ignoble, their wretched
mimicry and translation of a good foreign thing into a bad German
one. But now that the feverish unrest, the quest of gain and success,
the intense prizing of the moment, is added to it all, it makes one
furious to think that all this sickness can never be cured, but only
painted over, by such a "cult of the interesting." And this among a
people that has produced a Schopenhauer and a Wagner! and will
produce others, unless we are blindly deceiving ourselves; for should
not their very existence be a guarantee that such forces are even now
potential in the German spirit? Or will they be exceptions, the last
inheritors of the qualities that were once called German? I can see
nothing to help me here, and return to my main argument again, from
which my doubts and anxieties have made me digress. I have not yet
enumerated all the forces that help culture without recognising its
end, the production of genius. Three have been named; the
self-interest of business, of the state, and of those who draw the
cloak of "good form" over them. There is fourthly the self-interest
of science, and the peculiar nature of her servants--the learned.

Science has the same relation to wisdom as current morality to
holiness: she is cold and dry, loveless, and ignorant of any deep
feeling of dissatisfaction and yearning. She injures her servants in
helping herself, for she impresses her own character on them and
dries up their humanity. As long as we actually mean by culture the
progress of science, she will pass by the great suffering man and
harden her heart, for science only sees the problems of knowledge,
and suffering is something alien and unintelligible to her
world--though no less a problem for that!

If one accustom himself to put down every experience in a dialectical
form of question and answer, and translate it into the language of
"pure reason," he will soon wither up and rattle his bones like a
skeleton. We all know it: and why is it that the young do not shudder
at these skeletons of men, but give themselves blindly to science
without motive or measure? It cannot be the so-called "impulse to
truth": for how could there be an impulse towards a pure, cold and
objectless knowledge? The unprejudiced eye can see the real driving
forces only too plainly. The vivisection of the professor has much to
recommend it, as he himself is accustomed to finger and analyse all
things--even the worthiest! To speak honestly, the savant is a
complex of very various impulses and attractive forces--he is a base
metal throughout.

Take first a strong and increasing desire for intellectual adventure,
the attraction of the new and rare as against the old and tedious.
Add to that a certain joy in nosing the trail of dialectic, and
beating the cover where the old fox, Thought, lies hid; the desire is
not so much for truth as the chase of truth, and the chief pleasure
is in surrounding and artistically killing it. Add thirdly a love of
contradiction whereby the personality is able to assert itself
against all others: the battle's the thing, and the personal victory
its aim,--truth only its pretext. The impulse to discover "particular
truths" plays a great part in the professor, coming from his
submission to definite ruling persons, classes, opinions, churches,
governments, for he feels it a profit to himself to bring truth to
their side.

The following characteristics of the savant are less common, but
still found.--Firstly, downrightness and a feeling for simplicity,
very valuable if more than a mere awkwardness and inability to
deceive, deception requiring some mother-wit.--(Actually, we may be
on our guard against too obvious cleverness and resource, and doubt
the man's sincerity.)--Otherwise this downrightness is generally of
little value, and rarely of any use to knowledge, as it follows
tradition and speaks the truth only in "adiaphora"; it being lazier
to speak the truth here than ignore it. Everything new means
something to be unlearnt, and your downright man will respect the
ancient dogmas and accuse the new evangelist of failing in the
_sensus recti_. There was a similar opposition, with probability and
custom on its side, to the theory of Copernicus. The professor's
frequent hatred of philosophy is principally a hatred of the long
trains of reasoning and artificiality of the proofs. Ultimately the
savants of every age have a fixed limit; beyond which ingenuity is
not allowed, and everything suspected as a conspirator against
honesty.

Secondly, a clear vision of near objects, combined with great
shortsightedness for the distant and universal. The professor's range
is generally very small, and his eye must be kept close to the
object. To pass from a point already considered to another, he has to
move his whole optical apparatus. He cuts a picture into small
sections, like a man using an opera-glass in the theatre, and sees
now a head, now a bit of the dress, but nothing as a whole. The
single sections are never combined for him, he only infers their
connection, and consequently has no strong general impression. He
judges a literary work, for example, by certain paragraphs or
sentences or errors, as he can do nothing more; he will be driven to
see in an oil painting nothing but a mass of daubs.

Thirdly, a sober conventionality in his likes and dislikes. Thus he
especially delights in history because he can put his own motives
into the actions of the past. A mole is most comfortable in a
mole-hill. He is on his guard against all ingenious and extravagant
hypotheses; but digs up industriously all the commonplace motives of
the past, because he feels in sympathy with them. He is generally
quite incapable of understanding and valuing the rare or the
uncommon, the great or the real.

Fourthly, a lack of feeling, which makes him capable of vivisection.
He knows nothing of the suffering that brings knowledge, and does not
fear to tread where other men shudder. He is cold and may easily
appear cruel. He is thought courageous, but he is not,--any more than
the mule who does not feel giddiness.

Fifthly, diffidence, or a low estimate of himself. Though he live in
a miserable alley of the world, he has no sense of sacrifice or
surrender; he appears often to know in his inmost heart that he is
not a flying but a crawling creature. And this makes him seem even
pathetic.

Sixthly, loyalty to his teachers and leaders. From his heart he
wishes to help them, and knows he can do it best with the truth. He
has a grateful disposition, for he has only gained admittance through
them to the high hall of science; he would never have entered by his
own road. Any man to-day who can throw open a new province where his
lesser disciples can work to some purpose, is famous at once; so
great is the crowd that presses after him. These grateful pupils are
certainly a misfortune to their teacher, as they all imitate him; his
faults are exaggerated in their small persons, his virtues
correspondingly diminished.

Seventhly, he will follow the usual road of all the professors, where
a feeling for truth springs from a lack of ideas, and the wheel once
started goes on. Such natures become compilers, commentators, makers
of indices and herbaria; they rummage about one special department
because they have never thought there are others. Their industry has
something of the monstrous stupidity of gravitation; and so they can
often bring their labours to an end.

Eighthly, a dread of ennui. While the true thinker desires nothing
more than leisure, the professor fears it, not knowing how it is to
be used. Books are his comfort; he listens to everybody's different
thoughts and keeps himself amused all day. He especially chooses
books with a personal relation to himself, that make him feel some
emotion of like or dislike; books that have to do with himself or his
position, his political, æsthetic, or even grammatical doctrines; if
he have mastered even one branch of knowledge, the means to flap away
the flies of ennui will not fail him.

Ninthly, the motive of the bread-winner, the "cry of the empty
stomach," in fact. Truth is used as a direct means of preferment,
when she can be attained; or as a way to the good graces of the
fountains of honour--and bread. Only, however, in the sense of the
"particular truth": there is a gulf between the profitable truths
that many serve, and the unprofitable truths to which only those few
people devote themselves whose motto is not _ingenii largitor
venter_.

Tenthly, a reverence for their fellow-professors and a fear of their
displeasure--a higher and rarer motive than the last, though not
uncommon. All the members of the guild are jealously on guard, that
the truth which means so much bread and honour and position may
really be baptized in the name of its discoverer. The one pays the
other reverence for the truth he has found, in order to exact the
toll again if he should find one himself. The Untruth, the Error is
loudly exploded, that the workers may not be too many; here and there
the real truth will be exploded to let a few bold and stiff-necked
errors be on show for a time; there is never a lack of "moral
idiosyncrasies,"--formerly called rascalities.

Eleventhly, the "savant for vanity," now rather rare. He will get a
department for himself somehow, and investigate curiosities,
especially if they demand unusual expenditure, travel, research, or
communication with all parts of the world. He is quite satisfied with
the honour of being regarded as a curiosity himself, and never dreams
of earning a living by his erudite studies.

Twelfthly, the "savant for amusement." He loves to look for knots in
knowledge and to untie them; not too energetically however, lest he
lose the spirit of the game. Thus he does not penetrate the depths,
though he often observes something that the microscopic eyes of the
bread-and-butter scientist never see.

If I speak, lastly, of the "impulse towards justice" as a further
motive of the savant, I may be answered that this noble impulse,
being metaphysical in its nature, is too indistinguishable from the
rest, and really incomprehensible to mortal mind; and so I leave the
thirteenth heading with the pious wish that the impulse may be less
rare in the professor than it seems. For a spark in his soul from the
fire of justice is sufficient to irradiate and purify it, so that he
can rest no more and is driven for ever from the cold or lukewarm
condition in which most of his fellows do their daily work.

All these elements, or a part of them, must be regarded as fused and
pounded together, to form the Servant of Truth. For the sake of an
absolutely inhuman thing--mere purposeless, and therefore motiveless,
knowledge--a mass of very human little motives have been chemically
combined, and as the result we have the professor,--so transfigured
in the light of that pure unearthly object that the mixing and
pounding which went to form him are all forgotten! It is very
curious. Yet there are moments when they must be remembered,--when we
have to think of the professor's significance to culture. Any one
with observation can see that he is in his essence and by his origin
unproductive, and has a natural hatred of the productive; and thus
there is an endless feud between the genius and the savant in idea
and practice. The latter wishes to kill Nature by analysing and
comprehending it, the former to increase it by a new living Nature.
The happy age does not need or know the savant; the sick and sluggish
time ranks him as its highest and worthiest.

Who were physician enough to know the health or sickness of our time?
It is clear that the professor is valued too highly, with evil
consequences for the future genius, for whom he has no compassion,
merely a cold, contemptuous criticism, a shrug of the shoulders, as
if at something strange and perverted for which he has neither time
nor inclination. And so he too knows nothing of the aim of culture.

In fact, all these considerations go to prove that the aim of culture
is most unknown precisely where the interest in it seems liveliest.
The state may trumpet as it will its services to culture, it merely
helps culture in order to help itself, and does not comprehend an aim
that stands higher than its own well-being or even existence. The
business men in their continual demand for education merely wish
for--business. When the pioneers of "good form" pretend to be the
real helpers of culture, imagining that all art, for example, is
merely to serve their own needs, they are clearly affirming
themselves in affirming culture. Of the savant enough has already
been said. All four are emulously thinking how they can benefit
_themselves_ with the help of culture, but have no thoughts at all
when their own interests are not engaged. And so they have done
nothing to improve the conditions for the birth of genius in modern
times; and the opposition to original men has grown so far that no
Socrates could ever live among us, and certainly could never reach
the age of seventy.

I remember saying in the third chapter that our whole modern world
was not so stable that one could prophesy an eternal life to its
conception of culture. It is likely that the next millennium may
reach two or three new ideas that might well make the hair of our
present generation stand on end. The belief in the metaphysical
significance of culture would not be such a horrifying thing, but its
effects on educational methods might be so.

It requires a totally new attitude of mind to be able to look away
from the present educational institutions to the strangely different
ones that will be necessary for the second or third generation. At
present the labours of higher education produce merely the savant or
the official or the business man or the Philistine or, more commonly,
a mixture of all four; and the future institutions will have a harder
task;--not in itself harder; as it is really more natural, and so
easier; and further, could anything be harder than to make a youth
into a savant against nature, as now happens?--But the difficulty
lies in unlearning what we know and setting up a new aim; it will be
an endless trouble to change the fundamental idea of our present
educational system, that has its roots in the Middle Ages and regards
the mediæval savant as the ideal type of culture. It is already time
to put these objects before us; for some generation must begin the
battle, of which a later generation will reap the victory. The
solitary man who has understood the new fundamental idea of culture
is at the parting of the ways; on the one he will be welcomed by his
age, laurels and rewards will be his, powerful parties will uphold
him, he will have as many in sympathy behind him as in front, and
when the leader speaks the word of deliverance, it will echo through
all the ranks. The first duty is to "fight in line," the second to
treat as foes all who will not "fall in." On the other way he will
find fewer companions; it is steeper and more tortuous. The
travellers on the first road laugh at him, as his way is the more
troublesome and dangerous; and they try to entice him over. If the
two ways cross, he is ill-treated, cast aside or left alone. What
significance has any particular form of culture for these several
travellers? The enormous throng that press to their end on the first
road, understand by it the laws and institutions that enable them to
go forward in regular fashion and rule out all the solitary and
obstinate people who look towards higher and remoter objects. To the
small company on the other road it has quite a different office: they
wish to guard themselves, by means of a strong organisation, from
being swept away by the throng, to prevent their individual members
from fainting on the way or turning in spirit from their great task.
These solitary men must finish their work; that is why they should
all hold together; and those who have their part in the scheme will
take thought to prepare themselves with ever-increasing purity of aim
for the birth of the genius, and ensure that the time be ripe for
him. Many are destined to help on the labour, even among the
second-rate talents, and it is only in submission to such a destiny
that they can feel they are living for a duty, and have a meaning and
an object in their lives. But at present these talents are being
turned from the road their instinct has chosen by the seductive tones
of the "fashionable culture," that plays on their selfish side, their
vanities and weaknesses; and the time-spirit ever whispers in their
ears its flattering counsel:--"Follow me and go not thither! There
you are only servants and tools, over-shadowed by higher natures with
no scope for your own, drawn by threads, hung with fetters, slaves
and automatons. With me you may enjoy your true personality, and be
masters, your talents may shine with their own light, and yourselves
stand in the front ranks with an immense following round you; and the
acclamation of public opinion will rejoice you more than a wandering
breath of approval sent down from the cold ethereal heights of
genius." Even the best men are snared by such allurements, and the
ultimate difference comes not so much from the rarity and power of
their talent, as the influence of a certain heroic disposition at the
base of them, and an inner feeling of kinship with genius. For there
are men who feel it as their own misery when they see the genius in
painful toil and struggle, in danger of self-destruction, or
neglected by the short-sighted selfishness of the state, the
superficiality of the business men, and the cold arrogance of the
professors; and I hope there may be some to understand what I mean by
my sketch of Schopenhauer's destiny, and to what end Schopenhauer can
really educate.


VII.

But setting aside all thoughts of any educational revolution in the
distant future;--what provision is required _now_, that our future
philosopher may have the best chance of opening his eyes to a life
like Schopenhauer's--hard as it is, yet still livable? What, further,
must be discovered that may make his influence on his contemporaries
more certain? And what obstacles must be removed before his example
can have its full effect and the philosopher train another
philosopher? Here we descend to be practical.

Nature always desires the greatest utility, but does not understand
how to find the best and handiest means to her end; that is her great
sorrow, and the cause of her melancholy. The impulse towards her own
redemption shows clearly her wish to give men a significant existence
by the generation of the philosopher and the artist: but how unclear
and weak is the effect she generally obtains with her artists and
philosophers, and how seldom is there any effect at all! She is
especially perplexed in her efforts to make the philosopher useful;
her methods are casual and tentative, her failures innumerable; most
of her philosophers never touch the common good of mankind at all.
Her actions seem those of a spendthrift; but the cause lies in no
prodigal luxury, but in her inexperience. Were she human, she would
probably never cease to be dissatisfied with herself and her
bungling. Nature shoots the philosopher at mankind like an arrow; she
does not aim, but hopes that the arrow will stick somewhere. She
makes countless mistakes that give her pain. She is as extravagant in
the sphere of culture as in her planting and sowing. She fulfils her
ends in a large and clumsy fashion, using up far too much of her
strength. The artist has the same relation to the connoisseurs and
lovers of his art as a piece of heavy artillery to a flock of
sparrows. It is a fool's part to use a great avalanche to sweep away
a little snow, to kill a man in order to strike the fly on his nose.
The artist and the philosopher are witnesses against Nature's
adaptation of her means, however well they may show the wisdom of her
ends. They only reach a few and should reach all--and even these few
are not struck with the strength they used when they shot. It is sad
to have to value art so differently as cause and effect; how huge in
its inception, how faint the echo afterwards! The artist does his
work as Nature bids him, for the benefit of other men--no doubt of
it; but he knows that none of those men will understand and love his
work as he understands and loves it himself. That lonely height of
love and understanding is necessary, by Nature's clumsy law, to
produce a lower type; the great and noble are used as the means to
the small and ignoble. Nature is a bad manager; her expenses are far
greater than her profits: for all her riches she must one day go
bankrupt. She would have acted more reasonably to make the rule of
her household--small expense and hundredfold profit; if there had
been, for example, only a few artists with moderate powers, but an
immense number of hearers to appreciate them, stronger and more
powerful characters than the artists themselves; then the effect of
the art-work, in comparison with the cause, might be a hundred-tongued
echo. One might at least expect cause and effect to be of equal power;
but Nature lags infinitely behind this consummation. An artist, and
especially a philosopher, seems often to have dropped by chance into his
age, as a wandering hermit or straggler cut off from the main body.
Think how utterly great Schopenhauer is, and what a small and absurd
effect he has had! An honest man can feel no greater shame at the
present time than at the thought of the casual treatment Schopenhauer
has received and the evil powers that have up to now killed his effect
among men. First there was the want of readers,--to the eternal shame of
our cultivated age;--then the inadequacy of his first public adherents,
as soon as he had any; further, I think, the crassness of the modern man
towards books, which he will no longer take seriously. As an outcome of
many attempts to adapt Schopenhauer to this enervated age, the new
danger has gradually arisen of regarding him as an odd kind of pungent
herb, of taking him in grains, as a sort of metaphysical pepper. In this
way he has gradually become famous, and I should think more have heard
his name than Hegel's; and, for all that, he is still a solitary being,
who has failed of his effect.--Though the honour of causing the failure
belongs least of all to the barking of his literary antagonists; first
because there are few men with the patience to read them, and secondly,
because any one who does, is sent immediately to Schopenhauer himself;
for who will let a donkey-driver prevent him from mounting a fine horse,
however much he praise his donkey?

Whoever has recognised Nature's unreason in our time, will have to
consider some means to help her; his task will be to bring the free
spirits and the sufferers from this age to know Schopenhauer; and
make them tributaries to the flood that is to overbear all the clumsy
uses to which Nature even now is accustomed to put her philosophers.
Such men will see that the identical obstacles hinder the effect of a
great philosophy and the production of the great philosopher; and so
will direct their aims to prepare the regeneration of Schopenhauer,
which means that of the philosophical genius. The real opposition to
the further spread of his doctrine in the past, and the regeneration
of the philosopher in the future, is the perversity of human nature
as it is; and all the great men that are to be must spend infinite
pains in freeing themselves from it. The world they enter is
plastered over with pretence,--including not merely religious dogmas,
but such juggling conceptions as "progress," "universal education,"
"nationalism," "the modern state"; practically all our general terms
have an artificial veneer over them that will bring a clearer-sighted
posterity to reproach our age bitterly for its warped and stunted
growth, however loudly we may boast of our "health." The beauty of
the antique vases, says Schopenhauer, lies in the simplicity with
which they express their meaning and object; it is so with all the
ancient implements; if Nature produced amphoræ, lamps, tables,
chairs, helmets, shields, breastplates and the like, they would
resemble these. And, as a corollary, whoever considers how we all
manage our art, politics, religion and education--to say nothing of
our vases!--will find in them a barbaric exaggeration and
arbitrariness of expression. Nothing is more unfavourable to
the rise of genius than such monstrosities. They are unseen and
undiscoverable, the leaden weights on his hand when he will set it to
the plough; the weights are only shaken off with violence, and his
highest work must to an extent always bear the mark of it.

In considering the conditions that, at best, keep the born
philosopher from being oppressed by the perversity of the age, I am
surprised to find they are partly those in which Schopenhauer himself
grew up. True, there was no lack of opposing influences; the evil
time drew perilously near him in the person of a vain and pretentious
mother. But the proud republican character of his father rescued him
from her and gave him the first quality of a philosopher--a rude and
strong virility. His father was neither an official nor a savant; he
travelled much abroad with his son,--a great help to one who must
know men rather than books, and worship truth before the state. In
time he got accustomed to national peculiarities: he made England,
France and Italy equally his home, and felt no little sympathy with
the Spanish character. On the whole, he did not think it an honour to
be born in Germany, and I am not sure that the new political
conditions would have made him change his mind. He held quite openly
the opinion that the state's one object was to give protection at
home and abroad, and even protection against its "protectors," and to
attribute any other object to it was to endanger its true end. And
so, to the consternation of all the so-called liberals, he left his
property to the survivors of the Prussian soldiers who fell in 1848
in the fight for order. To understand the state and its duties in
this single sense may seem more and more henceforth the sign of
intellectual superiority; for the man with the _furor philosophicus_
in him will no longer have time for the _furor politicus_, and will
wisely keep from reading the newspapers or serving a party; though he
will not hesitate a moment to take his place in the ranks if his
country be in real need. All states are badly managed, when other men
than politicians busy themselves with politics; and they deserve to
be ruined by their political amateurs.

Schopenhauer had another great advantage--that he had never been
educated for a professor, but worked for some time (though against
his will) as a merchant's clerk, and through all his early years
breathed the freer air of a great commercial house. A savant can
never become a philosopher: Kant himself could not, but remained in a
chrysalis stage to the end, in spite of the innate force of his
genius. Any one who thinks I do Kant wrong in saying this does not
know what a philosopher is--not only a great thinker, but also a real
man; and how could a real man have sprung from a savant? He who lets
conceptions, opinions, events, books come between himself and things,
and is born for history (in the widest sense), will never see
anything at once, and never be himself a thing to be "seen at once";
though both these powers should be in the philosopher, as he must
take most of his doctrine from himself and be himself the copy and
compendium of the whole world. If a man look at himself through a
veil of other people's opinions, no wonder he sees nothing but--those
opinions. And it is thus that the professors see and live. But
Schopenhauer had the rare happiness of seeing the genius not only in
himself, but also outside himself--in Goethe; and this double
reflection taught him everything about the aims and culture of the
learned. He knew by this experience how the free strong man, to whom
all artistic culture was looking, must come to be born; and could he,
after this vision, have much desire to busy himself with the
so-called "art," in the learned, hypocritical manner of the moderns?
He had seen something higher than that--an awful unearthly
judgment-scene in which all life, even the highest and completest,
was weighed and found too light; he had beheld the saint as the judge
of existence. We cannot tell how early Schopenhauer reached this view
of life, and came to hold it with such intensity as to make all his
writings an attempt to mirror it; we know that the youth had this
great vision, and can well believe it of the child. Everything that
he gained later from life and books, from all the realms of
knowledge, was only a means of colour and expression to him; the
Kantian philosophy itself was to him an extraordinary rhetorical
instrument for making the utterance of his vision, as he thought,
clearer; the Buddhist and Christian mythologies occasionally served
the same end. He had one task and a thousand means to execute it; one
meaning, and innumerable hieroglyphs to express it.

It was one of the high conditions of his existence that he really
could live for such a task--according to his motto _vitam impendere
vero_--and none of life's material needs could shake his resolution;
and we know the splendid return he made his father for this. The
contemplative man in Germany usually pursues his scientific studies
to the detriment of his sincerity, as a "considerate fool," in search
of place and honour, circumspect and obsequious, and fawning on his
influential superiors. Nothing offended the savants more than
Schopenhauer's unlikeness to them.


VIII.

These are a few of the conditions under which the philosophical
genius can at least come to light in our time, in spite of all
thwarting influences;--a virility of character, an early knowledge of
mankind, an absence of learned education and narrow patriotism, of
compulsion to earn his livelihood or depend on the state,--freedom in
fact, and again freedom; the same marvellous and dangerous element in
which the Greek philosophers grew up. The man who will reproach him,
as Niebuhr did Plato, with being a bad citizen, may do so, and be
himself a good one; so he and Plato will be right together! Another
may call this great freedom presumption; he is also right, as he
could not himself use the freedom properly if he desired it, and
would certainly presume too far with it. This freedom is really a
grave burden of guilt; and can only be expiated by great actions.
Every ordinary son of earth has the right of looking askance on such
endowments; and may Providence keep him from being so endowed--
burdened, that is, with such terrible duties! His freedom and his
loneliness would be his ruin, and ennui would turn him into a fool,
and a mischievous fool at that.

A father may possibly learn something from this that he may use for
his son's private education, though one must not expect fathers to
have only philosophers for their sons. It is possible that they will
always oppose their sons becoming philosophers, and call it mere
perversity; Socrates was sacrificed to the fathers' anger, for
"corrupting the youth," and Plato even thought a new ideal state
necessary to prevent the philosophers' growth from being dependent on
the fathers' folly. It looks at present as though Plato had really
accomplished something; for the modern state counts the encouragement
of philosophy as one of its duties and tries to secure for a number
of men at a time the sort of freedom that conditions the philosopher.
But, historically, Plato has been very unlucky; as soon as a
structure has risen corresponding actually to his proposals, it has
always turned, on a closer view, into a goblin-child, a monstrous
changeling; compare the ecclesiastical state of the Middle Ages with
the government of the "God-born king" of which Plato dreamed! The
modern state is furthest removed from the idea of the Philosopher-king
(Thank Heaven for that! the Christian will say); but we must
think whether it takes that very "encouragement of philosophy" in a
Platonic sense, I mean as seriously and honestly as if its highest
object were to produce more Platos. If the philosopher seem, as
usual, an accident of his time, does the state make it its conscious
business to turn the accidental into the necessary and help Nature
here also?

Experience teaches us a better way--or a worse: it says that nothing
so stands in the way of the birth and growth of Nature's philosopher
as the bad philosophers made "by order." A poor obstacle, isn't it?
and the same that Schopenhauer pointed out in his famous essay on
University philosophy. I return to this point, as men must be forced
to take it seriously, to be driven to activity by it; and I think all
writing is useless that does not contain such a stimulus to activity.
And anyhow it is a good thing to apply Schopenhauer's eternal
theories once more to our own contemporaries, as some kindly soul
might think that everything has changed for the better in Germany
since his fierce diatribes. Unfortunately his work is incomplete on
this side as well, unimportant as the side may be.

The "freedom" that the state, as I said, bestows on certain men for
the sake of philosophy is, properly speaking, no freedom at all, but
an office that maintains its holder. The "encouragement of
philosophy" means that there are to-day a number of men whom the
state enables to make their living out of philosophy; whereas the old
sages of Greece were not paid by the state, but at best were
presented, as Zeno was, with a golden crown and a monument in the
Ceramicus. I cannot say generally whether truth is served by showing
the way to live by her, since everything depends on the character of
the individual who shows the way. I can imagine a degree of pride in
a man saying to his fellow-men, "take care of me, as I have something
better to do--namely to take care of you." We should not be angry at
such a heightened mode of expression in Plato and Schopenhauer; and
so they might properly have been University philosophers,--as Plato,
for example, was a court philosopher for a while without lowering the
dignity of philosophy. But in Kant we have the usual submissive
professor, without any nobility in his relations with the state; and
thus he could not justify the University philosophy when it was once
assailed. If there be natures like Schopenhauer's and Plato's, which
can justify it, I fear they will never have the chance, as the state
would never venture to give such men these positions, for the simple
reason that every state fears them, and will only favour philosophers
it does not fear. The state obviously has a special fear of
philosophy, and will try to attract more philosophers, to create the
impression that it has philosophy on its side,--because it has those
men on its side who have the title without the power. But if there
should come one who really proposes to cut everything to the quick,
the state included, with the knife of truth, the state, that affirms
its own existence above all, is justified in banishing him as an
enemy, just as it bans a religion that exalts itself to be its judge.
The man who consents to be a state philosopher, must also consent to
be regarded as renouncing the search for truth in all its secret
retreats. At any rate, so long as he enjoys his position, he must
recognise something higher than truth--the state. And not only the
state, but everything required by it for existence--a definite form
of religion, a social system, a standing army; a _noli me tangere_ is
written above all these things. Can a University philosopher ever
keep clearly before him the whole round of these duties and
limitations? I do not know. The man who has done so and remains a
state-official, is a false friend to truth; if he has not,--I think
he is no friend to truth either.

But general considerations like these are always the weakest in their
influence on mankind. Most people will find it enough to shrug their
shoulders and say, "As if anything great and pure has ever been able
to maintain itself on this earth without some concession to human
vulgarity! Would you rather the state persecuted philosophers than
paid them for official services?" Without answering this last
question, I will merely say that these "concessions" of philosophy to
the state go rather far at present. In the first place, the state
chooses its own philosophical servants, as many as its institutions
require; it therefore pretends to be able to distinguish the good and
the bad philosophers, and even assumes there must be a sufficient
supply of good ones to fill all the chairs. The state is the
authority not only for their goodness but their numbers. Secondly, it
confines those it has chosen to a definite place and a definite
activity among particular men; they must instruct every undergraduate
who wants instruction, daily, at stated hours. The question is
whether a philosopher can bind himself, with a good conscience, to
have something to teach every day, to any one who wishes to listen.
Must he not appear to know more than he does, and speak, before an
unknown audience, of things that he could mention without risk only
to his most intimate friends? And above all, does he not surrender
the precious freedom of following his genius when and wherever it
call him, by the mere fact of being bound to think at stated times on
a fixed subject? And before young men, too! Is not such thinking in
its nature emasculate? And suppose he felt some day that he had no
ideas just then--and yet must be in his place and appear to be
thinking! What then?

"But," one will say, "he is not a thinker but mainly a depository of
thought, a man of great learning in all previous philosophies. Of
these he can always say something that his scholars do not know."
This is actually the third, and the most dangerous, concession made
by philosophy to the state, when it is compelled to appear in the
form of erudition, as the knowledge (more specifically) of the
history of philosophy. The genius looks purely and lovingly on
existence, like a poet, and cannot dive too deep into it;--and
nothing is more abhorrent to him than to burrow among the innumerable
strange and wrong-headed opinions. The learned history of the past
was never a true philosopher's business, in India or Greece; and a
professor of philosophy who busies himself with such matters must be,
at best, content to hear it said of him, "He is an able scholar,
antiquary, philologist, historian,"--but never, "He is a
philosopher." I said, "at best": for a scholar feels that most of the
learned works written by University philosophers are badly done,
without any real scientific power, and generally are dreadfully
tedious. Who will blow aside, for example, the Lethean vapour with
which the history of Greek philosophy has been enveloped by the dull
though not very scientific works of Ritter, Brandis and Zeller? I, at
any rate, would rather read Diogenes Laertius than Zeller, because at
least the spirit of the old philosophers lives in Diogenes, but
neither that nor any other spirit in Zeller. And, after all, what
does the history of philosophy matter to our young men? Are they to
be discouraged by the welter of opinions from having any of their
own; or taught to join the chorus that approves the vastness of our
progress? Are they to learn to hate or perhaps despise philosophy?
One might expect the last, knowing the torture the students endure
for their philosophical examinations, in having to get into their
unfortunate heads the maddest efforts of the human mind as well as
the greatest and profoundest. The only method of criticising a
philosophy that is possible and proves anything at all--namely to see
whether one can live by it--has never been taught at the
universities; only the criticism of words, and again words, is taught
there. Imagine a young head, without much experience of life, being
stuffed with fifty systems (in the form of words) and fifty
criticisms of them, all mixed up together,--what an overgrown
wilderness he will come to be, what contempt he will feel for a
philosophical education! It is, of course, not an education in
philosophy at all, but in the art of passing a philosophical
examination: the usual result being the pious ejaculation of the
wearied examinee, "Thank God I am no philosopher, but a Christian and
a good citizen!"

What if this cry were the ultimate object of the state, and the
"education" or leading to philosophy were merely a leading _from_
philosophy? We may well ask.--But if so, there is one thing to
fear--that the youth may some day find out to what end philosophy is
thus mis-handled. "Is the highest thing of all, the production of the
philosophical genius, nothing but a pretext, and the main object
perhaps to hinder his production? And is Reason turned to
Unreason?"--Then woe to the whole machinery of political and
professorial trickery!

Will it soon become notorious? I do not know; but anyhow university
philosophy has fallen into a general state of doubting and despair.
The cause lies partly in the feebleness of those who hold the chairs
at present: and if Schopenhauer had to write his treatise on
university philosophy to-day, he would find the club no longer
necessary, but could conquer with a bulrush. They are the heirs and
successors of those slip-shod thinkers whose crazy heads Schopenhauer
struck at: their childish natures and dwarfish frames remind one of
the Indian proverb: "men are born according to their deeds, deaf,
dumb, misshapen." Those fathers deserved such sons, "according to
their deeds," as the proverb says. Hence the students will, no doubt,
soon get on without the philosophy taught at their university, just
as those who are not university men manage to do without it already.
This can be tested from one's own experience: in my student-days, for
example, I found the university philosophers very ordinary men
indeed, who had collected together a few conclusions from the other
sciences, and in their leisure hours read the newspapers and went to
concerts; they were treated by their academic colleagues with
politely veiled contempt. They had the reputation of knowing very
little, but of never being at a loss for obscure expressions to
conceal their ignorance. They had a preference for those obscure
regions where a man could not walk long with clear vision. One said
of the natural sciences,--"Not one of them can fully explain to me
the origin of matter; then what do I care about them all?"--Another
said of history, "It tells nothing new to the man with ideas": in
fact, they always found reasons for its being more philosophical to
know nothing than to learn anything. If they let themselves be drawn
to learn, a secret instinct made them fly from the actual sciences
and found a dim kingdom amid their gaps and uncertainties. They "led
the way" in the sciences in the sense that the quarry "leads the way"
for the hunters who are behind him. Recently they have amused
themselves with asserting they are merely the watchers on the
frontier of the sciences. The Kantian doctrine is of use to them
here, and they industriously build up an empty scepticism on it, of
which in a short time nobody will take any more notice. Here and
there one will rise to a little metaphysic of his own, with the
general accompaniment of headaches and giddiness and bleeding at the
nose. After the usual ill-success of their voyages into the clouds
and the mist, some hard-headed young student of the real sciences
will pluck them down by the skirts, and their faces will assume the
expression now habitual to them, of offended dignity at being found
out. They have lost their happy confidence, and not one of them will
venture a step further for the sake of his philosophy. Some used to
believe they could find out new religions or reinstate old ones by
their systems. They have given up such pretensions now, and have
become mostly mild, muddled folk, with no Lucretian boldness, but
merely some spiteful complaints of the "dead weight that lies on the
intellects of mankind"! No one can even learn logic from them now,
and their obvious knowledge of their own powers has made them
discontinue the dialectical disputations common in the old days.
There is much more care and modesty, logic and inventiveness, in a
word, more philosophical method in the work of the special sciences
than in the so-called "philosophy," and every one will agree with the
temperate words of Bagehot[2] on the present system builders:
"Unproved abstract principles without number have been eagerly caught
up by sanguine men, and then carefully spun out into books and
theories, which were to explain the whole world. But the world goes
clear against these abstractions, and it must do so, as they require
it to go in antagonistic directions. The mass of a system attracts
the young and impresses the unwary; but cultivated people are very
dubious about it. They are ready to receive hints and suggestions,
and the smallest real truth is ever welcome. But a large book of
deductive philosophy is much to be suspected. Who is not almost sure
beforehand that the premises will contain a strange mixture of truth
and error, and therefore that it will not be worth while to spend
life in reasoning over their consequences?" The philosophers,
especially in Germany, used to sink into such a state of abstraction
that they were in continual danger of running their heads against a
beam; but there is a whole herd of Laputan flappers about them to
give them in time a gentle stroke on their eyes or anywhere else.
Sometimes the blows are too hard; and then these scorners of earth
forget themselves and strike back, but the victim always escapes
them. "Fool, you do not see the beam," says the flapper; and often
the philosopher does see the beam, and calms down. These flappers are
the natural sciences and history; little by little they have so
overawed the German dream-craft which has long taken the place of
philosophy, that the dreamer would be only too glad to give up the
attempt to run alone: but when they unexpectedly fall into the
others' arms, or try to put leading-strings on them that they may be
led themselves, those others flap as terribly as they can, as if they
would say, "This is all that is wanting,--that a philosophaster like
this should lay his impure hands on us, the natural sciences and
history! Away with him!" Then they start back, knowing not where to
turn or to ask the way. They wanted to have a little physical
knowledge at their back, possibly in the form of empirical psychology
(like the Herbartians), or perhaps a little history; and then they
could at least make a public show of behaving scientifically,
although in their hearts they may wish all philosophy and all science
at the devil.

    [2] _Physics and Politics_, chap. v. Nietzsche has altered the
    order of the sentences without any apparent benefit to his own
    argument, and to the disadvantage of Bagehot's. I have restored
    the original order.--TR.

But granted that this herd of bad philosophers is ridiculous--and who
will deny it?--how far are they also harmful? They are harmful just
because they make philosophy ridiculous. As long as this
imitation-thinking continues to be recognised by the state, the
lasting effect of a true philosophy will be destroyed, or at any rate
circumscribed; nothing does this so well as the curse of ridicule
that the representatives of the great cause have drawn on them, for
it attacks that cause itself. And so I think it will encourage
culture to deprive philosophy of its political and academic standing,
and relieve state and university of the task, impossible for them, of
deciding between true and false philosophy. Let the philosophers run
wild, forbid them any thoughts of office or civic position, hold them
out no more bribes,--nay, rather persecute them and treat them
ill,--you will see a wonderful result. They will flee in terror and
seek a roof where they can, these poor phantasms; one will become a
parson, another a schoolmaster, another will creep into an
editorship, another write school-books for young ladies' colleges,
the wisest of them will plough the fields, the vainest go to court.
Everything will be left suddenly empty, the birds flown: for it is
easy to get rid of bad philosophers,--one only has to cease paying
them. And that is a better plan than the open patronage of any
philosophy, whatever it be, for state reasons.

The state has never any concern with truth, but only with the truth
useful to it, or rather, with anything that is useful to it, be it
truth, half-truth, or error. A coalition between state and philosophy
has only meaning when the latter can promise to be unconditionally
useful to the state, to put its well-being higher than truth. It
would certainly be a noble thing for the state to have truth as a
paid servant; but it knows well enough that it is the essence of
truth to be paid nothing and serve nothing. So the state's servant
turns out to be merely "false truth," a masked actor who cannot
perform the office required from the real truth--the affirmation of
the state's worth and sanctity. When a mediæval prince wished to be
crowned by the Pope, but could not get him to consent, he appointed
an antipope to do the business for him. This may serve up to a
certain point; but not when the modern state appoints an
"anti-philosophy" to legitimise it; for it has true philosophy
against it just as much as before, or even more so. I believe in all
seriousness that it is to the state's advantage to have nothing
further to do with philosophy, to demand nothing from it, and let it
go its own way as much as possible. Without this indifferent
attitude, philosophy may become dangerous and oppressive, and will
have to be persecuted.--The only interest the state can have in the
university lies in the training of obedient and useful citizens; and
it should hesitate to put this obedience and usefulness in doubt by
demanding an examination in philosophy from the young men. To make a
bogey of philosophy may be an excellent way to frighten the idle and
incompetent from its study; but this advantage is not enough to
counterbalance the danger that this kind of compulsion may arouse
from the side of the more reckless and turbulent spirits. They learn
to know about forbidden books, begin to criticise their teachers, and
finally come to understand the object of university philosophy and
its examinations; not to speak of the doubts that may be fostered in
the minds of young theologians, as a consequence of which they are
beginning to be extinct in Germany, like the ibexes in the Tyrol.

I know the objections that the state could bring against all this, as
long as the lovely Hegel-corn was yellowing in all the fields; but
now that hail has destroyed the crop and all men's hopes of it, now
that nothing has been fulfilled and all the barns are empty,--there
are no more objections to be made, but rather rejections of
philosophy itself. The state has now the power of rejection; in
Hegel's time it only wished to have it--and that makes a great
difference. The state needs no more the sanction of philosophy, and
philosophy has thus become superfluous to it. It will find advantage
in ceasing to maintain its professors, or (as I think will soon
happen) in merely pretending to maintain them; but it is of still
greater importance that the university should see the benefit of this
as well. At least I believe the real sciences must see that their
interest lies in freeing themselves from all contact with sham
science. And further, the reputation of the universities hangs too
much in the balance for them not to welcome a severance from methods
that are thought little of even in academic circles. The outer world
has good reason for its widespread contempt of universities; they are
reproached with being cowardly, the small fearing the great, and the
great fearing public opinion; it is said that they do not lead the
higher thought of the age but hobble slowly behind it, and cleave no
longer to the fundamental ideas of the recognised sciences. Grammar,
for example, is studied more diligently than ever without any one
seeing the necessity of a rigorous training in speech and writing.
The gates of Indian antiquity are being opened, and the scholars have
no more idea of the most imperishable works of the Indians--their
philosophies--than a beast has of playing the harp; though
Schopenhauer thinks that the acquaintance with Indian philosophy is
one of the greatest advantages possessed by our century. Classical
antiquity is the favourite playground nowadays, and its effect is no
longer classical and formative; as is shown by the students, who are
certainly no models for imitation. Where is now the spirit of
Friedrich August Wolf to be found, of whom Franz Passow could say
that he seemed a loyal and humanistic spirit with force enough to set
half the world aflame? Instead of that a journalistic spirit is
arising in the university, often under the name of philosophy; the
smooth delivery--the very cosmetics of speech--with Faust and Nathan
the Wise for ever on the lips, the accent and the outlook of our
worst literary magazines and, more recently, much chatter about our
holy German music, and the demand for lectures on Schiller and
Goethe,--all this is a sign that the university spirit is beginning
to be confused with the Spirit of the Age. Thus the establishment of
a higher tribunal, outside the universities, to protect and criticise
them with regard to culture, would seem a most valuable thing, and as
soon as philosophy can sever itself from the universities and be
purified from every unworthy motive or hypocrisy, it will be able to
become such a tribunal. It will do its work without state help in
money or honours, free from the spirit of the age as well as from any
fear of it; being in fact the judge, as Schopenhauer was, of the
so-called culture surrounding it. And in this way the philosopher can
also be useful to the university, by refusing to be a part of it, but
criticising it from afar. Distance will lend dignity.

But, after all, what does the life of a state or the progress of
universities matter in comparison with the life of philosophy on
earth! For, to say quite frankly what I mean, it is infinitely more
important that a philosopher should arise on the earth than that a
state or a university should continue. The dignity of philosophy may
rise in proportion as the submission to public opinion and the danger
to liberty increase; it was at its highest during the convulsions
marking the fall of the Roman Republic, and in the time of the
Empire, when the names of both philosophy and history became _ingrata
principibus nomina_. Brutus shows its dignity better than Plato; his
was a time when ethics cease to have commonplaces. Philosophy is not
much regarded now, and we may well ask why no great soldier or
statesman has taken it up; and the answer is that a thin phantom has
met him under the name of philosophy, the cautious wisdom of the
learned professor; and philosophy has soon come to seem ridiculous to
him. It ought to have seemed terrible; and men who are called to
authority should know the heroic power that has its source there. An
American may tell them what a centre of mighty forces a great thinker
can prove on this earth. "Beware when the great God lets loose a
thinker on this planet," says Emerson.[3] "Then all things are at
risk. It is as when a conflagration has broken out in a great city,
and no man knows what is safe, or where it will end. There is not a
piece of science, but its flank may be turned to-morrow; there is not
any literary reputation, not the so-called eternal names of fame,
that may not be revised and condemned.... The things which are dear
to men at this hour are so on account of the ideas which have emerged
on their mental horizon, and which cause the present order of things
as a tree bears its apples. A new degree of culture would instantly
revolutionise the entire system of human pursuits." If such thinkers
are dangerous, it is clear why our university thinkers are not
dangerous; for their thoughts bloom as peacefully in the shade of
tradition "as ever tree bore its apples." They do not frighten; they
carry away no gates of Gaza; and to all their little contemplations
one can make the answer of Diogenes when a certain philosopher was
praised: "What great result has he to show, who has so long practised
philosophy and yet has _hurt_ nobody?" Yes, the university philosophy
should have on its monument, "It has hurt nobody." But this is rather
the praise one gives to an old woman than to a goddess of truth; and
it is not surprising that those who know the goddess only as an old
woman are the less men for that, and are naturally neglected by the
real men of power.

    [3] Essay on "Circles."

If this be the case in our time, the dignity of philosophy is trodden
in the mire; and she seems herself to have become ridiculous or
insignificant. All her true friends are bound to bear witness against
this transformation, at least to show that it is merely her false
servants in philosopher's clothing who are so. Or better, they must
prove by their own deed that the love of truth has itself awe and
power.

Schopenhauer proved this and will continue to prove it, more and
more.









*** END OF THE PROJECT GUTENBERG EBOOK THOUGHTS OUT OF SEASON, PART II ***


    

Updated editions will replace the previous one—the old editions will
be renamed.

Creating the works from print editions not protected by U.S. copyright
law means that no one owns a United States copyright in these works,
so the Foundation (and you!) can copy and distribute it in the United
States without permission and without paying copyright
royalties. Special rules, set forth in the General Terms of Use part
of this license, apply to copying and distributing Project
Gutenberg™ electronic works to protect the PROJECT GUTENBERG™
concept and trademark. Project Gutenberg is a registered trademark,
and may not be used if you charge for an eBook, except by following
the terms of the trademark license, including paying royalties for use
of the Project Gutenberg trademark. If you do not charge anything for
copies of this eBook, complying with the trademark license is very
easy. You may use this eBook for nearly any purpose such as creation
of derivative works, reports, performances and research. Project
Gutenberg eBooks may be modified and printed and given away—you may
do practically ANYTHING in the United States with eBooks not protected
by U.S. copyright law. Redistribution is subject to the trademark
license, especially commercial redistribution.


START: FULL LICENSE

THE FULL PROJECT GUTENBERG LICENSE

PLEASE READ THIS BEFORE YOU DISTRIBUTE OR USE THIS WORK

To protect the Project Gutenberg™ mission of promoting the free
distribution of electronic works, by using or distributing this work
(or any other work associated in any way with the phrase “Project
Gutenberg”), you agree to comply with all the terms of the Full
Project Gutenberg™ License available with this file or online at
www.gutenberg.org/license.

Section 1. General Terms of Use and Redistributing Project Gutenberg™
electronic works

1.A. By reading or using any part of this Project Gutenberg™
electronic work, you indicate that you have read, understand, agree to
and accept all the terms of this license and intellectual property
(trademark/copyright) agreement. If you do not agree to abide by all
the terms of this agreement, you must cease using and return or
destroy all copies of Project Gutenberg™ electronic works in your
possession. If you paid a fee for obtaining a copy of or access to a
Project Gutenberg™ electronic work and you do not agree to be bound
by the terms of this agreement, you may obtain a refund from the person
or entity to whom you paid the fee as set forth in paragraph 1.E.8.

1.B. “Project Gutenberg” is a registered trademark. It may only be
used on or associated in any way with an electronic work by people who
agree to be bound by the terms of this agreement. There are a few
things that you can do with most Project Gutenberg™ electronic works
even without complying with the full terms of this agreement. See
paragraph 1.C below. There are a lot of things you can do with Project
Gutenberg™ electronic works if you follow the terms of this
agreement and help preserve free future access to Project Gutenberg™
electronic works. See paragraph 1.E below.

1.C. The Project Gutenberg Literary Archive Foundation (“the
Foundation” or PGLAF), owns a compilation copyright in the collection
of Project Gutenberg™ electronic works. Nearly all the individual
works in the collection are in the public domain in the United
States. If an individual work is unprotected by copyright law in the
United States and you are located in the United States, we do not
claim a right to prevent you from copying, distributing, performing,
displaying or creating derivative works based on the work as long as
all references to Project Gutenberg are removed. Of course, we hope
that you will support the Project Gutenberg™ mission of promoting
free access to electronic works by freely sharing Project Gutenberg™
works in compliance with the terms of this agreement for keeping the
Project Gutenberg™ name associated with the work. You can easily
comply with the terms of this agreement by keeping this work in the
same format with its attached full Project Gutenberg™ License when
you share it without charge with others.

1.D. The copyright laws of the place where you are located also govern
what you can do with this work. Copyright laws in most countries are
in a constant state of change. If you are outside the United States,
check the laws of your country in addition to the terms of this
agreement before downloading, copying, displaying, performing,
distributing or creating derivative works based on this work or any
other Project Gutenberg™ work. The Foundation makes no
representations concerning the copyright status of any work in any
country other than the United States.

1.E. Unless you have removed all references to Project Gutenberg:

1.E.1. The following sentence, with active links to, or other
immediate access to, the full Project Gutenberg™ License must appear
prominently whenever any copy of a Project Gutenberg™ work (any work
on which the phrase “Project Gutenberg” appears, or with which the
phrase “Project Gutenberg” is associated) is accessed, displayed,
performed, viewed, copied or distributed:

    This eBook is for the use of anyone anywhere in the United States and most
    other parts of the world at no cost and with almost no restrictions
    whatsoever. You may copy it, give it away or re-use it under the terms
    of the Project Gutenberg License included with this eBook or online
    at www.gutenberg.org. If you
    are not located in the United States, you will have to check the laws
    of the country where you are located before using this eBook.
  
1.E.2. If an individual Project Gutenberg™ electronic work is
derived from texts not protected by U.S. copyright law (does not
contain a notice indicating that it is posted with permission of the
copyright holder), the work can be copied and distributed to anyone in
the United States without paying any fees or charges. If you are
redistributing or providing access to a work with the phrase “Project
Gutenberg” associated with or appearing on the work, you must comply
either with the requirements of paragraphs 1.E.1 through 1.E.7 or
obtain permission for the use of the work and the Project Gutenberg™
trademark as set forth in paragraphs 1.E.8 or 1.E.9.

1.E.3. If an individual Project Gutenberg™ electronic work is posted
with the permission of the copyright holder, your use and distribution
must comply with both paragraphs 1.E.1 through 1.E.7 and any
additional terms imposed by the copyright holder. Additional terms
will be linked to the Project Gutenberg™ License for all works
posted with the permission of the copyright holder found at the
beginning of this work.

1.E.4. Do not unlink or detach or remove the full Project Gutenberg™
License terms from this work, or any files containing a part of this
work or any other work associated with Project Gutenberg™.

1.E.5. Do not copy, display, perform, distribute or redistribute this
electronic work, or any part of this electronic work, without
prominently displaying the sentence set forth in paragraph 1.E.1 with
active links or immediate access to the full terms of the Project
Gutenberg™ License.

1.E.6. You may convert to and distribute this work in any binary,
compressed, marked up, nonproprietary or proprietary form, including
any word processing or hypertext form. However, if you provide access
to or distribute copies of a Project Gutenberg™ work in a format
other than “Plain Vanilla ASCII” or other format used in the official
version posted on the official Project Gutenberg™ website
(www.gutenberg.org), you must, at no additional cost, fee or expense
to the user, provide a copy, a means of exporting a copy, or a means
of obtaining a copy upon request, of the work in its original “Plain
Vanilla ASCII” or other form. Any alternate format must include the
full Project Gutenberg™ License as specified in paragraph 1.E.1.

1.E.7. Do not charge a fee for access to, viewing, displaying,
performing, copying or distributing any Project Gutenberg™ works
unless you comply with paragraph 1.E.8 or 1.E.9.

1.E.8. You may charge a reasonable fee for copies of or providing
access to or distributing Project Gutenberg™ electronic works
provided that:

    • You pay a royalty fee of 20% of the gross profits you derive from
        the use of Project Gutenberg™ works calculated using the method
        you already use to calculate your applicable taxes. The fee is owed
        to the owner of the Project Gutenberg™ trademark, but he has
        agreed to donate royalties under this paragraph to the Project
        Gutenberg Literary Archive Foundation. Royalty payments must be paid
        within 60 days following each date on which you prepare (or are
        legally required to prepare) your periodic tax returns. Royalty
        payments should be clearly marked as such and sent to the Project
        Gutenberg Literary Archive Foundation at the address specified in
        Section 4, “Information about donations to the Project Gutenberg
        Literary Archive Foundation.”
    
    • You provide a full refund of any money paid by a user who notifies
        you in writing (or by e-mail) within 30 days of receipt that s/he
        does not agree to the terms of the full Project Gutenberg™
        License. You must require such a user to return or destroy all
        copies of the works possessed in a physical medium and discontinue
        all use of and all access to other copies of Project Gutenberg™
        works.
    
    • You provide, in accordance with paragraph 1.F.3, a full refund of
        any money paid for a work or a replacement copy, if a defect in the
        electronic work is discovered and reported to you within 90 days of
        receipt of the work.
    
    • You comply with all other terms of this agreement for free
        distribution of Project Gutenberg™ works.
    

1.E.9. If you wish to charge a fee or distribute a Project
Gutenberg™ electronic work or group of works on different terms than
are set forth in this agreement, you must obtain permission in writing
from the Project Gutenberg Literary Archive Foundation, the manager of
the Project Gutenberg™ trademark. Contact the Foundation as set
forth in Section 3 below.

1.F.

1.F.1. Project Gutenberg volunteers and employees expend considerable
effort to identify, do copyright research on, transcribe and proofread
works not protected by U.S. copyright law in creating the Project
Gutenberg™ collection. Despite these efforts, Project Gutenberg™
electronic works, and the medium on which they may be stored, may
contain “Defects,” such as, but not limited to, incomplete, inaccurate
or corrupt data, transcription errors, a copyright or other
intellectual property infringement, a defective or damaged disk or
other medium, a computer virus, or computer codes that damage or
cannot be read by your equipment.

1.F.2. LIMITED WARRANTY, DISCLAIMER OF DAMAGES - Except for the “Right
of Replacement or Refund” described in paragraph 1.F.3, the Project
Gutenberg Literary Archive Foundation, the owner of the Project
Gutenberg™ trademark, and any other party distributing a Project
Gutenberg™ electronic work under this agreement, disclaim all
liability to you for damages, costs and expenses, including legal
fees. YOU AGREE THAT YOU HAVE NO REMEDIES FOR NEGLIGENCE, STRICT
LIABILITY, BREACH OF WARRANTY OR BREACH OF CONTRACT EXCEPT THOSE
PROVIDED IN PARAGRAPH 1.F.3. YOU AGREE THAT THE FOUNDATION, THE
TRADEMARK OWNER, AND ANY DISTRIBUTOR UNDER THIS AGREEMENT WILL NOT BE
LIABLE TO YOU FOR ACTUAL, DIRECT, INDIRECT, CONSEQUENTIAL, PUNITIVE OR
INCIDENTAL DAMAGES EVEN IF YOU GIVE NOTICE OF THE POSSIBILITY OF SUCH
DAMAGE.

1.F.3. LIMITED RIGHT OF REPLACEMENT OR REFUND - If you discover a
defect in this electronic work within 90 days of receiving it, you can
receive a refund of the money (if any) you paid for it by sending a
written explanation to the person you received the work from. If you
received the work on a physical medium, you must return the medium
with your written explanation. The person or entity that provided you
with the defective work may elect to provide a replacement copy in
lieu of a refund. If you received the work electronically, the person
or entity providing it to you may choose to give you a second
opportunity to receive the work electronically in lieu of a refund. If
the second copy is also defective, you may demand a refund in writing
without further opportunities to fix the problem.

1.F.4. Except for the limited right of replacement or refund set forth
in paragraph 1.F.3, this work is provided to you ‘AS-IS’, WITH NO
OTHER WARRANTIES OF ANY KIND, EXPRESS OR IMPLIED, INCLUDING BUT NOT
LIMITED TO WARRANTIES OF MERCHANTABILITY OR FITNESS FOR ANY PURPOSE.

1.F.5. Some states do not allow disclaimers of certain implied
warranties or the exclusion or limitation of certain types of
damages. If any disclaimer or limitation set forth in this agreement
violates the law of the state applicable to this agreement, the
agreement shall be interpreted to make the maximum disclaimer or
limitation permitted by the applicable state law. The invalidity or
unenforceability of any provision of this agreement shall not void the
remaining provisions.

1.F.6. INDEMNITY - You agree to indemnify and hold the Foundation, the
trademark owner, any agent or employee of the Foundation, anyone
providing copies of Project Gutenberg™ electronic works in
accordance with this agreement, and any volunteers associated with the
production, promotion and distribution of Project Gutenberg™
electronic works, harmless from all liability, costs and expenses,
including legal fees, that arise directly or indirectly from any of
the following which you do or cause to occur: (a) distribution of this
or any Project Gutenberg™ work, (b) alteration, modification, or
additions or deletions to any Project Gutenberg™ work, and (c) any
Defect you cause.

Section 2. Information about the Mission of Project Gutenberg™

Project Gutenberg™ is synonymous with the free distribution of
electronic works in formats readable by the widest variety of
computers including obsolete, old, middle-aged and new computers. It
exists because of the efforts of hundreds of volunteers and donations
from people in all walks of life.

Volunteers and financial support to provide volunteers with the
assistance they need are critical to reaching Project Gutenberg™’s
goals and ensuring that the Project Gutenberg™ collection will
remain freely available for generations to come. In 2001, the Project
Gutenberg Literary Archive Foundation was created to provide a secure
and permanent future for Project Gutenberg™ and future
generations. To learn more about the Project Gutenberg Literary
Archive Foundation and how your efforts and donations can help, see
Sections 3 and 4 and the Foundation information page at www.gutenberg.org.

Section 3. Information about the Project Gutenberg Literary Archive Foundation

The Project Gutenberg Literary Archive Foundation is a non-profit
501(c)(3) educational corporation organized under the laws of the
state of Mississippi and granted tax exempt status by the Internal
Revenue Service. The Foundation’s EIN or federal tax identification
number is 64-6221541. Contributions to the Project Gutenberg Literary
Archive Foundation are tax deductible to the full extent permitted by
U.S. federal laws and your state’s laws.

The Foundation’s business office is located at 809 North 1500 West,
Salt Lake City, UT 84116, (801) 596-1887. Email contact links and up
to date contact information can be found at the Foundation’s website
and official page at www.gutenberg.org/contact

Section 4. Information about Donations to the Project Gutenberg
Literary Archive Foundation

Project Gutenberg™ depends upon and cannot survive without widespread
public support and donations to carry out its mission of
increasing the number of public domain and licensed works that can be
freely distributed in machine-readable form accessible by the widest
array of equipment including outdated equipment. Many small donations
($1 to $5,000) are particularly important to maintaining tax exempt
status with the IRS.

The Foundation is committed to complying with the laws regulating
charities and charitable donations in all 50 states of the United
States. Compliance requirements are not uniform and it takes a
considerable effort, much paperwork and many fees to meet and keep up
with these requirements. We do not solicit donations in locations
where we have not received written confirmation of compliance. To SEND
DONATIONS or determine the status of compliance for any particular state
visit www.gutenberg.org/donate.

While we cannot and do not solicit contributions from states where we
have not met the solicitation requirements, we know of no prohibition
against accepting unsolicited donations from donors in such states who
approach us with offers to donate.

International donations are gratefully accepted, but we cannot make
any statements concerning tax treatment of donations received from
outside the United States. U.S. laws alone swamp our small staff.

Please check the Project Gutenberg web pages for current donation
methods and addresses. Donations are accepted in a number of other
ways including checks, online payments and credit card donations. To
donate, please visit: www.gutenberg.org/donate.

Section 5. General Information About Project Gutenberg™ electronic works

Professor Michael S. Hart was the originator of the Project
Gutenberg™ concept of a library of electronic works that could be
freely shared with anyone. For forty years, he produced and
distributed Project Gutenberg™ eBooks with only a loose network of
volunteer support.

Project Gutenberg™ eBooks are often created from several printed
editions, all of which are confirmed as not protected by copyright in
the U.S. unless a copyright notice is included. Thus, we do not
necessarily keep eBooks in compliance with any particular paper
edition.

Most people start at our website which has the main PG search
facility: www.gutenberg.org.

This website includes information about Project Gutenberg™,
including how to make donations to the Project Gutenberg Literary
Archive Foundation, how to help produce our new eBooks, and how to
subscribe to our email newsletter to hear about new eBooks.

